\documentclass[bibother]{asl}
%\documentclass{book}[12pt]

\usepackage{amssymb}
\usepackage{comment}

\usepackage{color}
%\usepackage{geometry}
\usepackage{marginnote}


\usepackage[english]{babel}
\usepackage{graphicx}

%\markright{Dissertation}


\newtheorem{thm}{Theorem}[section]
\newtheorem{cor}[thm]{Corollary}
\newtheorem{lem}[thm]{Lemma}
\newtheorem{prop}[thm]{Proposition}


\theoremstyle{definition}
\newtheorem{defn}[thm]{Definition}

\theoremstyle{remark}
\newtheorem{rem}[thm]{Remark}
\newtheorem*{examples}{Examples}
\newtheorem*{example}{Example}
\newtheorem*{claim}{Claim}
\newtheorem{fact}[thm]{Fact} 







\newcommand{\ad}{\textsf{AD}}
\newcommand{\zf}{\textsf{ZF}}
\newcommand{\dc}{\textsf{DC}}
\newcommand{\ac}{\textsf{AC}}

\newcommand{\bd}{\boldsymbol{\delta}}
\newcommand{\bD}{\boldsymbol{\Delta}}
\newcommand{\bP}{\boldsymbol{\Pi}}
\newcommand{\bS}{\boldsymbol{\Sigma}}

\newcommand{\ww}{\omega^\omega}
\newcommand{\on}{\text{ON}}
\newcommand{\dom}{\text{dom}}


\newcommand{\cD}{\mathcal{D}}
\newcommand{\cL}{\mathcal{L}}
\newcommand{\cO}{\mathcal{O}}

\newcommand{\defeq}{:=}
\newcommand{\res}{\restriction}

\newcommand{\nl}{\hfill \break}

\newcommand{\od}[2]{{\operatorname{ oseq}_{#2}(#1)}}
\newcommand{\ods}[2]{ {\operatorname{oseq}}^*_{#2}(#1) }
\newcommand{\dep}[1]{ {\operatorname{depth}(#1)} }
\newcommand{\lv}[2]{\mathit{lev}_{#2}(#1)}
\newcommand{\bl}[2]{{\operatorname{B}_{#2}(#1)}}
\renewcommand{\v}[1]{\operatorname{\cdot}_{#1}}
\newcommand{\id}{ {\operatorname{id}} }
\newcommand{\LR}{{L(\mathbb{R})}}
\newcommand{\R}{{\mathbb{R}}}
\newcommand{\cal}{\mathcal}
\newcommand{\Cal}{\mathcal}

\newcommand{\al}[1]{{\aleph_{#1}}}
\newcommand{\cf}{\operatorname{cof}}
\newcommand{\cof}{\operatorname{cof}}
\newcommand{\lh}{\operatorname{lh}}

\newcommand{\ran}{\text{ran}}
\newcommand{\lex}{{\text{lex}}}

\newcommand{\sP}{\mathcal{P}}
\newcommand{\sS}{\mathcal{S}}
\newcommand{\sW}{\mathcal{W}}
\newcommand{\sK}{\mathcal{K}}
\newcommand{\sL}{\mathcal{L}}


\newcommand{\sG}{\mathcal{G}}
\newcommand{\sH}{\mathcal{H}}
\newcommand{\sA}{\mathcal{A}}
\newcommand{\sF}{\mathcal{F}}


\newcommand{\ml}{\ll}
\newcommand{\conc}{{}^\smallfrown}				   

% referee's comments
\newcommand{\rc}[2]{{\color{red}#2}\marginnote{{\color{red}#1}}}
% negative spacing
\newcommand{\ns}{{\mkern-7mu}}

% drawing related
\newcommand{\ket}[1]{$\left|#1\right\rangle$}
\newcommand{\Om}[1]{\small $\omega_{#1}$}
\newcommand{\De}[1]{$\Delta_{#1}$}
\newcommand{\Ga}[1]{$\Gamma_{#1}$}


\begin{document}

\section{Basic Definitions}


\begin{defn}
A linear ordering $<$ of $P$ is a \textbf{wellordering} if every subset of $P$ has a $<$-least element. 
\end{defn}



\begin{defn}
A \textbf{filter} over a set $S$  is a collection $F$ of subsets of $S$ s.t.
\begin{itemize}
\item[1)] $S\in F$;
\item[2)] if $X\in F$ and $Y\in F$, then $X\cap Y\in F$;
\item[3)] if $X \subset Y \subset S$ and $X\in F$, then $Y\in F$.
\end{itemize} 
A filter $F$ is a \textbf{proper filter} if
\begin{itemize}
\item[4)] $\emptyset\notin F$;
\end{itemize} 
\end{defn}

Examples:
\begin{itemize}
\item[(e1)] A trivial filter $F=\{S\}$
\item[(e2)] The Fr\'{e}chet filter $F=\{X\subseteq S\, :\, S-X \textrm{ is finite } \}$, where $S$ is infinite.
\end{itemize} 


\begin{defn}$F$ is a \textbf{principle filter} if $\exists X_0\subset  S$ s.t. 
$\forall X\in F, X_0\subset X$.
\end{defn}

\begin{defn}
A proper filter $U$ over $S$ is an \textbf{ultrafilter} if $\forall X\subseteq  S$,
either  $X\in U$ or  $S-X\in U$. (But not both since $X\cap (S-X)=\emptyset\notin U$)
\end{defn}

\begin{thm}[Tarski]
(AC) Every filter can be extended to an ultrafilter.
\end{thm}


\begin{defn}
A relation $\preceq$ on $A$ is a \textbf{prewellordering} (of $A$) iff it is reflexive, transitive, and connected. (I.e., $\forall a,b,c\in A, \quad (a\preceq a)$;\quad  $(a\preceq b) \land (b \preceq c) \to (a\preceq c); \quad   (a\preceq b) \lor (b\preceq a)$.)
\end{defn}


Example: If $U$ is an ultrafilter over $S$, and
$\forall f,g:S\to\on$ we define 
$$f \preceq_U g \iff \{i\in S: f(i)\le g(i)\}\in U,$$ 
then $\preceq_U $ is a prewellordering. 

\begin{defn}
A \textbf{prewellordering} is a wellordering of equivalence classes. 
\end{defn}




\begin{defn}
A set $H\subseteq A$ is \textbf{homogeneous for a partition} $\{X_i:i\in I\}$ of $A^n$, if $\exists i, H^n\subseteq X_i$.
\end{defn}


\begin{defn}
$k \to (\lambda)^n_m$ means every $F:k^n\to m$ is constant on $H^n$ for some $H\subseteq k$ with $|H|=\lambda$.
\end{defn}


\begin{thm}[Ramsey]
$\omega\to(\omega)^n_k$, $\, \forall n,k\in\omega$. 

I.e., 
$\forall F:\omega^n\to\{1,\dots,k\},\, \exists H\subseteq\omega$ s.t. $|H|=\omega$  and $F$ is constant on $H^n$.

Equivalently, 
every partition $\{X_1,\dots,X_k\}$ of $\omega^n$ into $K$ pieces has an infinite homogeneous set.
\end{thm}


\begin{lem}
For all $\kappa$ and $\lambda$,
\begin{itemize}
\item[(a)] $2^\kappa \not\to (\lambda)^2_\kappa$
\item[(b)] $2^\kappa \not\to (\kappa^+)^2_2$
\item[(c)] $\aleph_1 \not\to (\aleph_1)^2_2$
\end{itemize}

\end{lem}





\newpage
Chart of Cardinals in A.Kanamori's book on page 472.
\begin{itemize}
\item 0=1
\item I0-I3
\item $n$-huge
\item superhuge
\item huge
\item almost huge
\item Vopenka's principle
\item extendible
\item supercompact
\item superstrong
\item Woodin
\item strong
\item $0^\dagger$ exists
\item measurable
\item Ramsey
\item Rowbottom
\item Jonsson
\item $0^{\textrm{\#}}$ exists
\item $\kappa\longrightarrow (\omega)^{<\omega}_2$ 
\item indescribable
\item weakly compact
\item $\alpha$-Mahlo
\item Mahlo
\item $\alpha$-inaccessible
\item inaccessible
\item weekly inaccessible

\end{itemize}

\section{Notes from Jech's book}
[page 398]
Let $A$ be a set of size at least $\kappa$. By $P_\kappa(A)$, we denote the set 
$$ P_\kappa(A):=\{P\subset A \, :\, |P|<\kappa     \}
$$ 

\begin{defn}[p.408]
A fine measure $U$ on $P_\kappa(A)$ is {\em normal} if whenever $f:P_\kappa(A)\to A$
is such that $f(P)\in P$ for almost all $P$, then $f$ is constant on a set in $U$.
\end{defn}

\begin{defn}[p.408]
A cardinal $\kappa$ is {\em supercompact} if for every $A$ such that $|A|\ge\kappa$, there exists
a normal measure on $P_\kappa(A)$.
\end{defn}

\begin{example}[Exercise 33.3]
If $\kappa$ is measurable, then there is a normal measure on $P_\kappa(\kappa)$.
\end{example}












\newpage

\section{Definitions and the general picture}  \label{introduction}

\begin{defn}
For each $n\ge 1$ we define a projective ordinal $\bd^1_n$ as follows
$$\bd^1_n:=\sup\{\, \xi \, : \, \xi \textrm{ is the length of }    \bD^1_n \textrm{ prewellordering of } \R (=\omega^\omega)
\}$$
\end{defn}




\section{For all $n$, $\bd^1_n$ is a cardinal}


\section{For all $n$, $\bd^1_n$ is a successor cardinal}
\begin{itemize}
\item[a) ]
$\bd^1_{2n+2}=(\bd^1_{2n+1})^+$
\item[b) ]
$\bd^1_{2n+1}=(\kappa_{2n+1})^+$, where $\kappa_{2n+1}$ is a cardinal of cofinality $\omega$
\item[c) ]
$\bd^1_{2n+1}=\aleph_{e(2n-1)+1}$, where $e(1)=\omega$ and $e(n+1)=\omega^{e(n)}$ (ordinal exponentiation)
\end{itemize}

Concretely,
\begin{align*}
\bd^1_{1}&=\aleph_1 \qquad\qquad  & \bd^1_{2}&=\aleph_2 \\
\bd^1_{3}&=\aleph_{\omega+1}
\qquad\qquad
& \bd^1_{4}&=\aleph_{\omega+2}\\
%
\bd^1_{5}&=\aleph_{\omega^{\omega^\omega}+1} & 
\bd^1_{6}&=\aleph_{\omega^{\omega^\omega}+2}\\
%
\bd^1_{7}&=\aleph_{\omega^{\omega^{\omega^{\omega^\omega}}}+1} & 
\bd^1_{8}&=\aleph_{\omega^{\omega^{\omega^{\omega^\omega}}}+2}
\end{align*}


\section{For all $n$, $\bd^1_n$ is a regular cardinal}
\section{For all $n$, $\bd^1_n$ is a measurable cardinal}

\section{Calculating $\bd^1_n$ for $n\le 4$}
\section{The closed unbounded measure on $\aleph_1$}
\section{Uniform indiscernibles and the $\aleph_n$'s for $n\le\omega$}
\section{Back to the real world}
What was developed so far was in the context of ZF + DC + AD. Now we review the results under ZF + AC and \textbf{Projective Determinacy} only.
\begin{thm}  Under ZF + AC and \textbf{Projective Determinacy} only,
\begin{itemize}
\item[1.] $\bd^1_1=\aleph_1$
\item[2.] $\bd^1_2\le \aleph_2$\nl
$[\forall \alpha (\alpha^{\#} \mbox{ \textrm{exists} })] \, \bd^1_2=u_2$
\end{itemize}
\end{thm}



\section{Infinite exponent partition relations and the singular measures $\mu_\lambda$}
\begin{defn}
If $\alpha, \beta, \gamma$ are ordinals with $\gamma \le \beta \le \alpha$, we write $\alpha \to \beta^\gamma$ iff $\forall X\subset \alpha^\gamma \uparrow, \exists H\subset\alpha \textrm{ with } o.t.(H)=\beta 
\textrm{ such that either } H^\gamma\uparrow \subset X 
\textrm{ or } H^\gamma\uparrow \subset (\alpha-X)  $
\end{defn}

\begin{rem}
Assuming ZFC, there is no $\kappa$ such that $\kappa \to \omega^\omega$ 
\end{rem}

\begin{defn}
$A\subseteq \kappa$ is $\lambda$-closed if every increasing $\lambda$-sequence from $A$ has its limit in $A$.
\end{defn}

\begin{defn}
If $\lambda, \kappa$ are regular cardinals, and $\lambda<\kappa$. $\mu_\lambda$ is defined to be  the set of all subsets of $\kappa$ which contain a $\lambda$-closed unbounded set.
\end{defn}


\begin{thm}
Let $\lambda, \kappa$ be regular cardinals, $\kappa$ uncountable,  $\lambda<\kappa$, and $\kappa\to\kappa^{\lambda+ \lambda}$. Then $\mu_\lambda$ is a normal measure.
\end{thm}





\section{Countable exponent partition relations for $\bd^1_n$, $n$ odd.}

\section{$\omega_1 \rightarrow \omega_1^{\omega_1}$}

\section{The Martin-Paris Theorem}

\section{The measure $\mu_\omega$ on $\bd^1_n$, $n$ odd.}

\section{The measure $\mu_\lambda$, with $\lambda > \omega$, on $\bd^1_n$, $n$ odd.}

\section{Countable exponent partition relations on $\bd^1_n$, $n$ even.}

\section{The measure $\mu_\omega$ on $\bd^1_n$, $n$ even.}

\section{Some singular cardinals}

\newpage


\section{Infinity}

"The Mathematics of Infinity" by Theodore G. Faticoni (QA 248.F29 2006)

"Roads to Infinity" by John Stillwell. (QA 248.S778 2010 )

\begin{enumerate}
\item The Diagonal Argument
\item Ordinals
\item Computability and Proof
\item Logic
\item Arithmetic
\item Natural Unprovable Sentences
\item Axioms of Infinity

"By the 1930s, mathematicians were contemplating sets large enough to model the universe itself: the co-called \emph{inaccessible} and \emph{measurable} cardinals."

"a better understanding of finite objects depends on a better understanding of infinity"
\begin{enumerate}
\item Set Theory without Infinity
\item Inaccessible Cardinals

Let $V_0=\emptyset$ is the empty set, $V_1=\{\emptyset \}$, $V_\alpha$ is consists of sets built by $<\alpha$ applications of the power set operation. $V_\lambda$, for limit $\lambda$, is the union of the $V_\alpha$ for $\alpha<\lambda$.

The least $\alpha$ for which a set $X$ is a subset of $V_\alpha$ is called the \emph{rank} of $X$. The axiom of foundation implies that every set has a rank.

A set $V_\alpha$ that is closed under the power set and replacement operations is necessarily of the form $V_\kappa$ for some cardinal $\kappa$. The cardinal $\kappa$ is then called \emph{inaccessible}, because in some sense $V_\kappa$ cannot be reached from below.
Note, in ZF we cannot prove the existence of an inaccessible $\kappa>\omega$. Otherwise we could use $V_\kappa$ to model all the axioms of ZF, and hence prove in ZF the sentence Con(ZF), which is impossible by Godel's second incompleteness theorem (ZF includes PA).

Thus "inaccessible $\kappa>\omega$ exists" is a new kind of axiom of infinity.
\end{enumerate}
\end{enumerate}


\end{document}














\section{Introduction}  \label{introduction}

We work throughout in the theory $\zf+\ad+\dc$. The projective ordinals 
play an important role in the descriptive set theory of the projective sets. 
They are defined by:
$$
\bd^1_n=\mbox{the supremum of the lengths of the $\bD^1_{n}$ prewellordings of $\ww$.}
$$

\noindent
For example, every $\bP^1_{2n+1}$ set admits a $\bP^1_{2n+1}$-scale 
onto $\bd^1_{2n+1}$. In \cite{Mo} a basic theory of the projective sets
(assuming $\ad$) is given, and it presented largely in terms of these ordinals. 
The work of the descriptive set theorists of the late 60's and 70's 
established some of the fundamental properties of these ordinals, 
such as: they are all measurable cardinals (Moschovakis, Kunen), 
$\bd^1_{2n+2}=(\bd^1_{2n+1})^+$ (Kunen, Martin),
and $\bd^1_{2n+1}=(\lambda_{2n+1})^+$, where $\cf(\lambda_{2n+1})=\omega$ (Kechris). 
The projective ordinals through $\bd^1_4$ were also computed (Kunen, Martin, Solovay). 
We refer the reader to \cite{Mo}, \cite{Ke1} for an accounting of these results
along with more detailed  history and credits. 
In the 80's, building on work of Kunen and Martin, Jackson 
computed the values of the projective
ordinals $\bd^1_n$ and showed that all the odd projective ordinals
$\bd^1_{2n+1}$ have the strong partition property (defined below),
a crucial element in the analysis. The result 
was that $$\bd^1_{2n+1}= \aleph_{e(2n-1)+1},$$ where 
$e(1)=\omega$ and $e(n+1)=\omega^{e(n)}$ (ordinal exponentiation).



 
The upper bound in the general case appears in
\cite{J2}, and the complete argument for $\bd^1_5$ appears in
\cite{J1}. The reader can also consult \cite{J3}
for an introduction to this theory. 

A key part of the projective ordinal analysis is the concept of a
{\it description}.
Intuitively, a description is a finitary  object ``describing'' how to build
an equivalence class of a function $f\colon \bd^1_3 \to \bd^1_3$
with respect to certain canonical measures $W^m_3$ which we
define below. The proof of the upper bound for the $\bd^1_{2n+3}$
proceeds by showing that every successor cardinal less than $\bd^1_{2n+3}$ is
represented by a description, and then counting the number of descriptions.
The lower bound for $\bd^1_{2n+3}$ was obtained by embedding
enough ultrapowers of $\bd^1_{2n+1}$ (by various measures
on $\bd^1_{2n+1}$) into $\bd^1_{2n+3}$. A theorem of Martin gives
that these ultrapowers are all cardinals, and the lower bound follows.
A question left open, however, was whether every description
actually represents a cardinal. The main result of this paper
is to show, below $\bd^1_5$, that this is the case. Thus, the descriptions
below $\bd^1_5$ exactly correspond to the cardinals below $\bd^1_5$.
Aside from rounding out the theory of descriptions, the results
presented here also serve to simplify some of the ordinal computations
of \cite{J1}. In fact, implicit in our results is a simple (in principle)
algorithm for determining the cardinal represented by a given
description. This, in itself, could prove useful in addressing
certain questions about the cardinals below the projective ordinals.
Also, the analysis presented here gives an alternate way to describe the cardinal 
structure without mentioning descriptions (although descriptions are
used heavily in the proofs). As an application of this we 
give in \S\ref{applications} a formula
which computes the cofinality of an arbitrary cardinal below $\bd^1_5$. 



We have attempted to make this paper as self-contained  as possible modulo basic $\ad$ facts
about $\bd^1_1$ and  $\bd^1_3$.  The main facts that we require are that 
$\bd^1_1=\omega_1$, $\bd^1_3=\omega_{\omega+1}$, and that 
$\bd^1_1$ and $\bd^1_3$ have the strong partition property (defined below). 
The earlier (pre-description) theory suffices to establish 
the facts mentioned  at the beginning as well to show the strong partition
property on $\bd^1_1$ and the weak partition property on $\bd^1_3$. 
The last two facts are due to Martin and Kunen respectively. 
Proofs of these facts and further background on the projective ordinals
can be found in \cite{Mo}, \cite{Ke1}, and 
%\cite{So}
\rc{mc01?}{\cite{So}}.  
Also, 
proofs of these facts from the modern perspective of descriptions 
(using only ``trivial' descriptions) can be found in \cite{J3}. 
The proof of the strong partition property on $\bd^1_3$ requires the theory
of descriptions. This can be found in \cite{J1}. The proofs of these results,
however, are not \rc{mc}{important} for the current paper, and the reader can 
simply take these partition results as given. 
 





Since we are not assuming familiarity with \cite{J1}, we present in
the next section the definition of description (below $\bd^1_5$) and some related concepts.
The reader familiar with \cite{J1} will note that we use here slightly
simplified versions of these notions as compared to \cite{J1}. 
Although the concepts are not changed in any significant way, 
we have discarded some notation which is not necessary here 
(such as  the notion of ``type'' of a description). 
Since the notion of a description is necessarily somewhat technical, 
we carry along through the paper some specific examples to help the
reader. 





\section{Preliminaries} \label{preliminaries}

By a {\em measure} on a set $X$ we mean a countably additive ultrafilter on $X$. 
Under $\ad$, every ultrafilter on a set is countably additive, so the notions
of measure and ultrafilter coincide. If $\mu$ is a measure on an ordinal 
$\lambda$ and $f \colon \lambda \to \on$, we write $[f]_\mu$
for the equivalence class of $f$ in the ultrapower by the measure $\mu$. 
Although we don't have Los' theorem without $\ac$, it nevertheless
makes sense to identify $[f]_\mu$ with an ordinal as usual. If $\mu$
is a measure on an ordinal and $\alpha \in \on$, we let $j_\mu(\alpha)$
denote the image of $\alpha$ in the ultrapower embedding by the measure $\mu$. 
That is, $j_\mu(\alpha)$ is the ordinal corresponding to the equivalence
class of the constant function $\alpha$. 


If $\mu$ is a measure on a set $X$, we write ``$\forall^*_\mu x \in X$''
to mean ``for $\mu$ almost all $x \in X$.'' If $\mu_1,\dots \mu_t$
are measures on $X_1,\dots,X_t$, we write $\forall^*_{\mu_1} x_1 \cdots
\forall^*_{\mu_t} x_t$'' to mean: for $\mu_1$ almost all $x_1 \in X_1$ 
it is the case that for $\mu_2$ almost all $x_2 \in X_2$ it is the case that,
\dots, for $\mu_t$ almost all $x_t \in X_t$.'' That is, we are referring
to the iterated product measure of the $\mu_i$. 



We  recall (a special case of) the Erd\H{o}s-Rado partition notation and
the definitions of the weak and strong partition properties.
We let throughout $(A)^\alpha$ denote the set of increasing functions
from $\alpha$ to $A$, where $\alpha \in \on$ and $A \subseteq \on$. 


\begin{defn}
For $\lambda \leq \kappa  \in \on$, we write $\kappa \rightarrow (\kappa)^\lambda$ 
to denote that for any partition $\sP \colon (\kappa)^\lambda \to 
\{0,1\}$ of the increasing functions from $\lambda$ to $\kappa$ into 
two pieces, there is a {\em homogeneous} set $H \subseteq \kappa$
of size $\kappa$. That is, $\sP \res (H)^\lambda$ is constant. 
We say $\kappa$ has the {\em weak} partition property if 
$\forall \lambda <\kappa\ \kappa \rightarrow (\kappa)^\lambda$ and say $\kappa$
has the {\em strong} partition property if $\kappa \rightarrow (\kappa)^\kappa$.
\end{defn}



It is convenient to reformulate the partition properties in a way that uses 
c.u.b.\ homogeneous sets. To do this, we must restrict the ``type'' of the functions
being partitioned. For our purposes we require only the simplest type,
which we define next. 



\begin{defn}
For $\alpha \in \on$, we say a function $f \colon \alpha \to \on$ has {\em uniform cofinality} 
$\omega$ if there is a function $g \colon \alpha \times \omega \to 
\on$ which is increasing in the second argument and 
for all $\beta < \alpha$ we have $f(\beta)= \sup_{n \in \omega} g(\beta,n)$. 
If $f\colon \alpha \to  \on$, we say $f$ has the {\em correct type}
if is strictly increasing, everywhere discontinuous (that is, 
for all limit $\beta$ we have $f(\beta)> \sup_{\gamma< \beta} f(\gamma$)), and of
uniform cofinality $\omega$.

We say $f \colon \alpha \to \on$ is of {\em continuous type} 
if $f$ is continuous (that is, for all limit $\beta$ we have 
$f(\beta)=\sup_{\gamma<\beta} f(\gamma)$) and $f$ restricted to the 
successor ordinals has uniform cofinality $\omega$ (that is, the function 
$g$ above has  domain the set of $(\beta,n)$ such that $\beta <\alpha$,
$\beta$ is a successor ordinal, and $n \in \omega$). 
\end{defn}




The reformulation of the partition property using c.u.b.\ sets can now be stated. 
We say $\kappa \overset{c.u.b}{\longrightarrow}  (\kappa)^\lambda$
if for every partition $\sP$ of the functions $f \colon \lambda \to \kappa$ of the correct type
there is a c.u.b.\ $C \subseteq \kappa$ which is homogeneous for $\sP$ 
(that is, $\sP$ takes a constant value on the functions $f \colon \lambda \to C$
of the correct type). For any infinite cardinal $\lambda$, the two versions of the partition
property using exponent $\lambda$ are equivalent. This follows from the following fact.
For the sake of completeness we sketch the proof (see also 
%\cite{So}
\rc{mc02?}{\cite{So}}
).

\begin{fact}
For all $\lambda \leq \kappa$ we have:
\begin{enumerate}
\item
$\kappa \overset{c.u.b.}{\longrightarrow} (\kappa)^\lambda$ implies 
$\kappa \rightarrow (\kappa)^\lambda$.
\item 
$\kappa \rightarrow (\kappa)^{\omega \cdot \lambda}$ implies 
$\kappa \overset{c.u.b.}{\longrightarrow} (\kappa)^\lambda$
\end{enumerate}
\end{fact}


\begin{proof}
Suppose first $\kappa \overset{c.u.b.}{\longrightarrow} (\kappa)^\lambda$
and let $\sP \colon (\kappa)^\lambda \to \{ 0,1\}$. $\sP$ induces by restriction a 
%
%parttion
\rc{mc}{partition}
%
of the functions of the correct type, and by assumption there is a c.u.b.\ $C \subseteq
\kappa$ which is homogeneous for the restricted partition. Let $H=\ran(h)$,
where $h \colon \kappa \to \kappa$ is given by $h(\alpha)= \omega \cdot (\alpha+1)$st
element of $C$. Note that if $f \in (H)^\lambda$, then $f$ is necessarily of the correct type. 
It follows that $H$ is homogeneous for $\sP$. 

Suppose next that $\kappa \rightarrow (\kappa)^{\omega \cdot \lambda}$,
and let $\sP$ be a partition of the functions $f \colon \lambda \to \kappa$
of the correct type. Let $\sP'$ be the partition of the increasing
functions $f' \colon \omega \cdot \lambda \to \kappa$   defined by:
$\sP'(f')=\sP(f)$ where $f$ is the function induced by $f'$, that is, 
$f(\alpha)=\sup_{\beta <\omega \cdot(\alpha+1)} f'(\beta)$.  Let $H\subseteq \kappa$
be homogeneous for $\sP'$. Let $C$ be the set of limit points of $H$. 
Then if $f\colon \lambda \to C$ is of the correct type, there is an 
increasing $f' \colon \omega \cdot \lambda \to H$ which induces $f$. It follows that $C$
is homogeneous for $\sP$.
\end{proof}





We henceforth officially adopt the c.u.b.\ versions of the 
%
%partiton 
\rc{mc}{partition} 
%
relations,
and we just write $\kappa \rightarrow (\kappa)^\lambda$ for the c.u.b.\ version
from now on.   
Also, if $\prec$ is a 
%
%wellorder 
\rc{mc}{well-order} 
%
of length $\mid \prec\mid =\lambda$, and 
$\kappa \rightarrow (\kappa)^\lambda$, then by identifying the domain of 
$\prec$ with $\lambda$ we have a partition relation for functions 
$f \colon \dom(\prec) \to \kappa$ of the correct type (with obvious meaning).




If $\kappa$ has the weak partition property, and $\lambda <\kappa$ is regular, 
then the c.u.b.\ filter restricted to points of cofinality $\lambda$ 
gives  a normal measure on $\kappa$. We call this measure the 
$\lambda$-cofinal normal measure on $\kappa$. In particular, we have the 
$\omega$-cofinal normal measure. The $n$-fold product of this normal measure can
also be described as the 
%
%mesure 
\rc{mc}{measure} 
%
induced from the partition relation 
$\kappa \rightarrow (\kappa)^n$. That is, $A \subseteq \kappa^n$ has measure one
iff there is a c.u.b.\ $C \subseteq \kappa$ such that $(C)^n \subseteq A$. 


More generally, if $\kappa \rightarrow (\kappa)^\lambda$, then the partition relation
induces a measure of the functions $f \colon \lambda \to \kappa$ of the correct type. 
Namely, $A$ has measure one iff there is a c.u.b.\ $C \subseteq \kappa$ such that
for all $f \colon \lambda \to C$ of the correct type we have $f \in A$. 
If $\mu$ is a measure on 
$\lambda$, then we may also speak of the measure induced by the partition relation 
$\kappa \rightarrow (\kappa)^\lambda$ and the measure $\mu$.  This measure is defined 
as follows. 
A set $A\subseteq \on $ has measure one iff 
there is a c.u.b.\ $C \subseteq \kappa$ such that for all 
$f \colon \lambda \to C$ of the correct type we have $[f]_\mu \in A$. The first 
(function space) measure induces the second (ordinal) measure via the map
$f \mapsto [f]_\mu$. 










We define next  three families of canonical measures, These measures play an important role in
the theory of descriptions below $\bd^1_5$. Although we are confining out attention to the 
cardinals below $\bd^1_5$ in this paper, these measures have generalizations
that play a similar role below the general projective ordinal (the reader can find
the general definitions in \cite{J2}).




For $r \in \omega$, let $<_r$ be the well-ordering of $(\omega_1)^r$
defined by: 
$$(\alpha_1,\dots,\alpha_r) <_r (\beta_1, \dots, \beta_r)
\Leftrightarrow  (\alpha_r,\alpha_1,\dots,\alpha_{r-1}) <_{\lex}
(\beta_r,\beta_1,\dots,\beta_{r-1}),$$
where $<_\lex$ denotes lexicographic ordering. Note that $<_r$ has order-type $\omega_1$. 
Thus, granting the strong partition relation on $\omega_1$, it makes sense to 
consider partitions of the functions $f \colon \dom(<_r) \to \omega_1$ of the correct type. 
We will use this in the following definition.  We will also implicitly
use two $\ad$ facts. The first is that the ultrapower of $\omega_1$ by the
$m$-fold product of the normal measure on $\omega_1$ (which we 
define below to be $W_1^m$) is equal to $\omega_{m+1}$. A proof can be found in 
\cite{Ke1}. The second fact is that for any 
%
%meaure 
\rc{mc}{measure} 
%
$\mu$ on $\lambda <\bd^1_3$,
and any $\theta<\bd^1_3$, we have $j_\mu(\theta)<\bd^1_3$. This follows, for example,
from the arguments of 
%\cite{So}
\rc{mc03?}{\cite{So}}
. Actually, we only need this fact for certain 
measures $\mu$ we define on the $\omega_k$, and in this case this fact can also
be proved directly [Since $\bd^1_3$ is a cardinal, it suffices to show that 
$j_\mu(\omega_\omega)<\bd^1_3$.  In fact, $j_\mu(\omega_\omega)=\omega_\omega$.
By countable additivity, this follows from $j_\mu(\omega_k)<\omega_\omega$
for all $k$. This last fact, for the measures $\mu$ we consider in this paper, 
can be shown directly.]



\begin{defn}[Canonical Measures]
We define the ordinal measures $W_1^r$, $S_1^r$, $W_3^r$, and the function space
measures $\sW_1^r$, $\sS_1^r$, $\sW_3^r$  as follows. 

\begin{enumerate}
\item
$\sW_1^r=W^r_1$ is the $r$-fold product of the normal measure
on $\omega_1$.
\item
$\sS^r_1$ is the measure on functions $f \colon \dom(<_r) \to
\omega_1$ of the correct type induced by the strong partition relation on $\omega_1$. 
$S^r_1$ is the measure on $\omega_{r+1}$ induced from $\sS^r_1$ and the measure 
$W_1^r$ on $(\omega_1)^r$. 
That is,  $A \subseteq \omega_{r+1}$ has $S^r_1$ measure one iff
$\exists \text{ c.u.b. } C\subseteq \omega_1$ $\forall f\colon \dom(<_r)
\to C$ of the correct type, $[f]_{W^r_1} \in A$.


\item
$\sW_3^r$ is the measure on functions $f \colon \omega_{r+1} \to \bd^1_3$ 
of the correct type induced by the weak partition relation on $\bd^1_3$. 
$W^r_3$ is the measure on $\bd^1_3$ induced from $\sW_3^r$ and the measure 
$S_1^r$ on $\omega_{r+1}$. 
That is, 
$A \subseteq \bd^1_3$ has $W_3^r$ measure one iff
$\exists \text{ c.u.b. } C\subseteq \bd^1_3$ $\forall f\colon 
\omega_{r+1} \to C$ of the correct type, $[f]_{S^r_1}
\in A$.
\end{enumerate}
\end{defn}






As the reader has probably guessed, the ``W'' in these definitions stands for 
``weak'' and the ``S'' for strong. Also, the subscript denotes which 
projective ordinal the partition property is being  applied to. 
The strong partition property of $\omega_1$ and the weak partition property
of $\bd^1_3$ respectively suffice to show that $S_1^r$, $W_3^r$ are measures. 



For our purposes, it is convenient to introduce also  variations
of these measures.  For each of the $(r-1)!$ permutations
$\pi=(r,i_1,\dots,i_{r-1})$ of $\{1,2,\dots,r\}$ beginning with $r$, let
$<^\pi$ be the corresponding well-ordering of $(\omega_1)^r$; that is,
$(\alpha_1,\dots,\alpha_r) <^\pi (\beta_1, \dots, \beta_r)$ iff
$(\alpha_r,\alpha_{i_1}, \dots, \alpha_{i_{r-1}}) <_{\lex}
(\beta_r, \beta_{i_1}, \dots, \beta_{i_{r-1}})$.
We say $h \colon (\omega_1)^r \to \omega_1$ is of type $\pi$ 
if $h$ is order-preserving with respect to $<^\pi$, discontinuous at
points of limit rank, and has uniform cofinality $\omega$. 
Let $S^\pi_1$ denote the corresponding measure on $\omega_{r+1}$ (as
in the definition of $S^r_1$), using functions $h$ of type $\pi$. 
Of course, $S_1^r$ is also a measure of the form $S_1^\pi$, using
the particular permutation $(r,1,2,\dots,r-1)$. 
We also define the measure $W^r$ as follows. 
$W^r$ is the
measure on $(r-1)!$ tuples $(\dots, \alpha_\pi, \dots)$ of
ordinals $< \bd^1_3$ defined by: $A$ has measure one iff
there is a c.u.b.\ $C \subseteq \bd^1_3$ such that for all 
%$f \colon \omega_{r-1}\to C$ 
$f \colon \omega_{r{\rc{mc04?}{+}}1}\to C$ 
which are of continuous type,
$(\dots,\alpha_\pi,\dots)\in A$, where $\alpha_\pi=[f]_{S_1^\pi}$.
We let $\sS_1^\pi$ and $\sW^r$ denote the corresponding function space measures. 






\begin{defn} \label{definvariants}
If $h\colon \dom(<_r) \to  \omega_1$
is of the correct type, we define the {\em invariants} of
%$f$
$\rc{mc05}{h}$
as follows: for $1 \leq j \leq r-1$, we define 
\begin{equation*}
\begin{split}
h(j)(\alpha_1,\dots, \alpha_{j})= 
\sup \{ & h(\alpha_1, \dots, \alpha_{j-1}, \beta_{j}, \dots, \beta_{r-1},
\alpha_{j})\colon \\ & \alpha_{j-1}< \beta_{j} < \dots < \beta_{r-1}<
\alpha_{j} \}.
\end{split}
\end{equation*}
 


We also define $h(r)=h$. Similarly, for
$1 \leq j \leq r$ we define 
\begin{equation*}
\begin{split}
h^s(j)(\alpha_1, \dots,
\alpha_{j})= \sup \{ & h(\alpha_1, \dots, \alpha_{j-2}, \beta_{j-1},
\beta_{j}, \dots, \beta_{r-1}, \alpha_{j}) \colon \\ & 
\beta_{j-1}<\alpha_{j-1},
\beta_{j-1}< \beta_{j} < \dots < \beta_{r-1}<
\alpha_{j} \}
\end{split}
\end{equation*}
\end{defn}



Note that $h^s(j)$ is 
%ontained
\rc{mc06}{obtained} 
from $h(j)$ by applying a supremum to the 
least significant variable, that is, 
$$
h^s(j)(\alpha_1,\dots,\alpha_{j-1},\alpha_j)=
\sup_{\beta_{j-1}<\alpha_{j-1}} h(j)(\alpha_1,\dots,\alpha_{j-2},
%\beta,\alpha_{j-1}).
\beta_{\rc{}{j-1}},\alpha_{\rc{mc?mc07?}{j}}).
$$
The superscript ``$s$'' in fact stands for ``supremum.''



If $\alpha = [h]_{W^r_1}$, where $h\colon \dom(<_r) \to \omega_1$ is of the
correct type, 
let 
%$\alpha(j)=[h(j)]_{W^{j+1}_1}$ 
$\alpha(j)=[h(j)]_{W^{\rc{mc08?}{j}}_1}$ 
for $1 \leq
j \leq r$. This is easily well-defined (that is, does not depend on the choice
of $h$ of the correct type representing $\alpha$). 













We turn now to the definition of descriptions. We will follow the 
definition in \cite{J1}, simplified somewhat (there is no need to
consider ``type $-1$'' descriptions).


A description is a
finitary object, and has an index associated with it.
An index is of the form $(f_m)$, where $m\geq 1$ is an integer, 
and is written as a superscript of the description. The index is a purely
syntactic object, it is merely a formal symbol. The form of the notation suggests
a function, and for higher level descriptions (which we do consider in this paper)
the intuitive meaning becomes more significant. Here, the reader can think of the index $(f_m)$
as a reminder that the description will take as input a function $f \colon m \to
\omega_1$ (i.e., an $m$-tuple of countable ordinals), and ultimately
return a countable ordinal value. 
Descriptions indexed as $d^{(f_m)}$ will be called level--$m$ descriptions,
We frequently suppress writing the index when it is understood or irrelevant.
The descriptions defined directly will be also referred at as {\em basic}
descriptions, and the ones defined in terms of the other descriptions
will be called {\em non--basic}.


In \cite{J1} we defined descriptions and then defined a certain 
``well-definedness'' condition which was called ``Condition C.'' 
Following a suggestion of the referee, we call the descriptions of 
\cite{J1} here ``pre-descriptions,'' and incorporate Condition C
into the definition of description. In fact, after we have defined
descriptions there will be no need to ever refer back to the 
pre-descriptions.





Let $t \in \omega$ and fix a sequence of measures $K_1,\dots, K_t$
with each $K_i$ of the form $K_i=S_1^r$ or $W_1^r$ for some $r=r(i)$
which depends on $i$. Fix also $m \in \omega$. 
A set of level--$m$ pre-descriptions,
$\cD'_m=\cD'_m(K_1,\dots,K_t)$,
is defined relative to this sequence of measures.
Along with $\cD'_m$ is also defined
a numerical function $k\colon \cD'_m \to \{1,\dots,t\}\cup\{\infty\}$.
It will be apparent from the definition that the function $k$ 
does not 
%depent
\rc{mc09}{depend} 
on the value of $m$ or the sequence of measures 
$K_1,\dots, K_t$, but only on the syntactic object $d$. Thus,
we are justified in simply writing ``$k$'' for the function. 
In the last case of the following definition another formal
symbol appears. The symbol ``s'' is again a purely syntactic
object; the symbol stands for ``sup'' and its meaning will become clearer
when we define the interpretation of a description below (it plays the same role
as in definition~\ref{definvariants}).


\begin{defn}[Pre-Descriptions]   \label{def_pf}
The set of pre-descriptions 
$\cD'_m(K_1,\dots,K_t)$ and
the function $k_m\colon \cD'_m \to \{1,\dots,t\}\cup\{\infty\}$
are defined  by reverse induction on $k_m(d)$ through the following cases:



\underline{Basic Pre-Descriptions:} 
We allow the following objects.
\begin{itemize}
\item[1.] \label{pdb1}
$d=(k;p)$ where $1\le k\le t$, $K_k=W^r_1$,
and $1\le p\le r$. We set $k(d) =k$.

\item[2.] \label{pdb2}
$d =(p)$ where $1\le p\le m$. We set $k(d)= \infty$.
\end{itemize}

\underline{Non--Basic Pre-Descriptions:} 
We allow the following objects.
\begin{itemize}
\item[1.] \label{pdnb1}
 $d=(k;d_0,d_1,d_2,\dots,d_l)$
where $1\le k\le t$, $K_k=S^r_1$, $l\le r-1$, each $d_i \in \cD'_m$, and 
$k(d_0), k(d_1),\dots,k(d_l)>k$ (we allow $l=0$ in which case $d=(k;d_0)$).
We set $k(d)= k$.
\item[2.]  \label{pdnb2}
$d=(k;d_0,d_1,d_2,\dots,d_l)^s$,
where $r\ge 2$, $1\le k\le t$,
$K_k=S^r_1$, $1 \leq l\leq r-1$, each $d_i \in \cD'_m$, and 
$k(d_0), k(d_1),\dots,k(d_l)>k$. We set $k(d)= k$.
\end{itemize}
\end{defn}


Now let $\cD'(K_1,\dots,K_t)\defeq \bigcup_m \cD'_m(K_1,\dots,K_t)$
to be the set of pre-descriptions relative to $K_1,\dots,K_t$.
We will frequently suppress the background sequence of measures simply writing
$\cD'$ or $\cD'_m$. We will write $\bar{K}$ to denote a sequence of measures. 
Note that if $\bar{K}$ is a subsequence of $\bar{K}'$, then
$\cD'_m(\bar{K}) \subseteq \cD'_m(\bar{K}')$.


In writing descriptions, we adopt the notation convention 
of writing the symbol $s$ in parentheses, that is, we write $(k;d_r, d_1,\dots,d_l)^{(s)}$,
to indicate that the symbol $s$ may or may not appear. 


Next we give the definition of the interpretation of a pre-description.
Fix $d\in\cD'_m$, let $h_1,\dots,h_t$ be functions of type
$K_1,\dots,K_t$, that is, if $K_i=W^r_1$, then $h_i\colon r\to \omega_1$,
and if $K_i=S^r_1$, then $h_i\colon \dom(<_r) \to \omega_1$ is of the correct type.
We define the ordinal $(d;\bar h)=(d; h_1,\dots,h_t)$
through cases by reverse induction on $k(d)$. If $d=d^{(f_m)}$ then
$(d; h_1,\dots,h_t) < \omega_{m+1}$ and is represented with respect to
$W^m_1$ by a function which is denoted by $(\alpha_1,\dots,\alpha_m) \to
(d; h_1,\dots,h_t)(\alpha_1,\dots,\alpha_m)$. The ordinal 
$(d; h_1,\dots,h_t)(\alpha_1,\dots,\alpha_m)< \omega_1$ is defined 
as follows. 


\begin{defn}[Interpretation of Pre-Descriptions in $\cD'_m$]  \label{interp}
\nl

\underline{Basic:}
\begin{itemize}
\item[1.] \label{ib1}
If $d=(k;p)$, then
$(d; \bar h)(\alpha_1,\dots,\alpha_m)=h_k(p)$.
\item[2.] \label{ib2}
If $d=(p), 1\le p\le m$, then
$(d; \bar h)(\alpha_1,\dots,\alpha_m)=\alpha_p$.
\end{itemize}

\underline{Non--Basic:}
\begin{itemize}
\item[1.] \label{inb1}
If $d=(k; d_0, d_1, d_2, \dots,d_l)$, then 
$$(d; \bar h)(\bar \alpha)= 
h_k(l+1)((d_1;\bar h)(\bar \alpha), \dots, (d_l;\bar h)(\bar \alpha), (d_0;\bar h)(\bar \alpha)).$$

\item[2.] \label{inb2}
If $d=(k; d_0, d_1,d_2, \dots,d_l)^s$, then 
$$(d;\bar h)(\bar \alpha)=
h^s_k(l+1)((d_1;\bar h)(\bar \alpha), \dots, 
(d_l;\bar h)(\bar \alpha),(d_0;\bar h)(\bar \alpha)).$$ 
\end{itemize}
\end{defn}




Next we define a relation $<$ on $\cD'_m(K_1,\dots,K_t)$ which will well-order 
this (finite) set. 
For ordinal measure $K_i$, we let $\sK_i$ denote the corresponding function
space measure. 




\begin{defn}[Order $<$ on $\cD'_m(K_1,\dots,K_t)$]
If $d_1, d_2 \in\cD'(K_1,\dots,K_t)$ then
$d_1<d_2$ iff $\forall^*_{\sK_1} h_1 \cdots \forall^*_{\sK_t} h_t\  
(d_1,\bar h)< (d_2,\bar h)$.
\end{defn}






Recall that ``$\forall^*_{\sK_1} h_1 \cdots \forall^*_{\sK_t} h_t$''
refers to the iterated product measure, in this case of the function space measures
$\sK_i$. 


The ordering 
%$<$
$\rc{mc10!}{<}$ 
on $\cD'_m(\bar K)$ can be checked to be a well-ordering on
$\cD'_m(K_1, \dots, K_t)$. The only non-trivial part to check is that 
if $d_1 \neq d_2$ then $d_1<d_2$ or $d_2<d_1$. The reader may note that this 
is why we required that $r \geq 2$ and $\ell \geq 1$ in case~2
of the definition for non-basic pre-descriptions 
\rc{MajorC2.1!}{(if we had allowed $d=(k;d_1)^s$ as a description, it would have the same interpretation as $d_1$)}. 



We next introduce the set of descriptions $\cD_m(K_1,\dots,K_t)$ by adding one extra condition
to the notion of pre-description. This extra condition ensures that 
the interpretation of the description is well-defined with respect 
%
\rc{mc11}{to} 
the
iteration of the ordinal measures $K_1,\dots,K_t$. This we state
precisely in lemma~\ref{welldef} below.




\begin{defn}[Descriptions]
Inductively, we say $d\in \cD'_m(K_1,\dots,K_t)$ is a {\em description} if either $d$ is basic or else
$d$ is non--basic, say of the form $d=(k;d_0,d_1,\dots,d_l)^{(s)}$, 
and $d_1<d_2<\dots<d_l<d_0$,
and $d_0, d_1,\dots,d_l$ are also descriptions. We let $\cD_m(K_1,\dots,K_t)$
be the set of level--$m$ descriptions defined relative to $K_1,\dots,K_t$. 
\end{defn}


\begin{rem}
If $m<m'$ and $d \in \cD_m(K_1,\dots,K_t)$, then inspecting the definition
of $\cD_m(K_1,\dots,K_t)$ shows that $d$ may be regarded as an element
of $\cD_{m'}(K_1,\dots,K_t)$. That is, if we remove the superscript 
$(f_m)$ from $d^{(f_m)} \in \cD_m(K_1,\dots,K_t)$ and then add the superscript 
$(f_{m'})$, we will have an element of $\cD_{m'}(K_1,\dots,K_t)$. 
The only purpose of this superscript is to tell us for which $m$
we regard $d$ as an element of $\cD_m(K_1,\dots,K_t)$. With the slight
abuse of ignoring the superscripts, we may therefore 
write  $\cD_m(K_1,\dots,K_t) \subseteq \cD_{m'}(K_1,\dots,K_t)$. 
\end{rem}






In the following lemma, when we write ``$h_i=h'_i$ almost everywhere,'' we mean 
$[h_i]_{W_1^r}=[h'_i]_{W_1^r}$ if $K_i=S_i^r$, and mean simply $h_i=h'_i$
if $K_i=W_1^i$. 




\begin{lem} \label{welldef}
Suppose $d\in \cD_m(K_1,\dots,K_t)$. Then for $\sK_1$ almost all $h_1$, if $h_1=h_1'$ a.e.,
then for $\sK_2$ almost all $h_2$, if $h_2=h_2'$ a.e., $\dots$, then for $\sK_t$ almost
all $h_t$, if $h_t=h_t'$ a.e., then $(d;\bar h)=(d;\bar{h'})$.
\end{lem}




The lemma is proved by a straightforward induction on the
definition of description. We omit the details. 


In view of lemma~\ref{welldef},
if $d \in \cD_m(K_1,\dots,K_t)$ is a description, then we may write 
$\forall_{K_1}^* [h_1] \cdots \forall_{K_t}^* [h_t] \ P((d;[h_1],\dots,[h_t]))$
for any set $P \subseteq \on$. That is, we may use the iteration of the ordinal measures $K_i$
instead of the function space measures $\sK_i$. To ease notation, we frequently
write $$\forall_{K_1}^* h_1 \cdots \forall_{K_t}^* h_t \ P((d;h_1,\dots,h_t)),$$ 
that is, we write $h_i$
in place of $[h_i]$ even when using the ordinal measure $K_i$. In view of 
lemma~\ref{welldef} this should cause no confusion. The reader should be
warned not to mis-interpret lemma~\ref{welldef}, however. The lemma 
does not say (and it 
%
\rc{mc12}{is} 
not in general true) that there is a well-defined map 
$([h_1], \dots, [h_t]) \mapsto (d;h_1,\dots,h_t)$. That is, the interpretation 
function (for most descriptions) is not a well-defined function on the 
product space $K_1 \times \cdots \times K_t$ (although it is well-defined on the 
function space product $\sK_1 \times \cdots \times \sK_t$). When dealing with the ordinal measure
spaces $K_1,\dots,K_t$, the notation $(d;h_1,\dots, h_t)$ only makes sense
inside a string of quantifiers $\forall_{K_1}^* h_1 \cdots \forall_{K_t}^* h_t$. 






Having formally defined descriptions and their interpretations, we
introduce now a somewhat less formal notation to represent them, which
we refer to as the {\em functional representation} of the
description. In the 
%
%functonal 
\rc{mc}{functional} 
%
representation, the notation more
closely identifies the description with its interpretation.
The functional representation of a description can be viewed as a
term in the language with function symbols $h_i(j)$,
$h^s_i(j)$, and variables $\alpha_{i,j}$, $\cdot_r$.
A basic description of the form $(k;p)$ will be
represented as $\alpha_{k,p}$. The basic description
$(p)$ will be represented as $\cdot_p$. A non-basic description of the
form $d= (k; d_0, d_1, d_2, \dots,
d_l)$ will then be represented as
$h_k(l+1)(g_1,\dots,g_l,g_0)$, where $g_0, g_1, \dots, g_l$ are the
representations of $d_0,d_1, \dots, d_l$. Similarly,
$d= (k; d_0, d_1, d_2, \dots,d_l)^s$ is represented as
%$h^s_k(l_1)(g_1,\dots,g_l,g_0)$. 
$h^s_k(\rc{mc13}{l+1})(g_1,\dots,g_l,g_0)$. 
Recall in this case we must have $l \geq 1$. Note that in the 
functional representation, the arguments are written in 
increasing order since $d_1<d_2 \cdots <d_l<d_0$ (in the original description notation, they 
are 
%wrutten
\rc{mc14}{written} 
in their  order of significance in determining the size of the 
output value). 


In using the functional notation, we will (as in the original notation)
write $h^{(s)}_k(l)$ to denote either $h_k(l)$ or $h^s_k(l)$, i.e., the 
symbol $s$ may or may not appear. 



Note that the variable  $\alpha_{i,j}$ is identified with the description $d=(i;j)$ whose
interpretation relative to $h_1,\dots$, $h_t$ is the ordinal
$\alpha_{i,j}$, where $h_i=(\alpha_{i,1}, \dots, \alpha_{i,j}, \dots)$.
Also, the variable $\cdot_p$ corresponds to the description $d=(p)$ whose interpretation
is represented by the function $(\alpha_1, \dots, \alpha_m)
\rightarrow \alpha_p$.



\begin{examples}
For the sequence of measures $K_1=S^4_1$, $K_2=S^4_1$,
$K_3=S^3_1$, $K_4=W^4_1$, some descriptions 
in $\cD_4(\bar{K})$ are:
$$d_1= h_1(3)(\alpha_{4,2}, h_2(2)(\alpha_{4,1}, \cdot_3), \cdot_4),$$
and $$d_2= h_1(1)(h_2(2)(\alpha_{4,4},h_3(1)(\cdot_4))).$$
For the first of these, and for fixed $h_1, \dots, h_4=(
\alpha_{4,1}, \dots, \alpha_{4,4})$, the interpretation of
$d$ is the ordinal represented with 
%repect
\rc{mc}{respect} 
to $W^4_1$ by the
function $(\beta_1, \dots, \beta_4)\to 
h_1(3)(\alpha_{4,2}, h_2(2)(\alpha_{4,1}, \beta_3), \beta_4)$.
\end{examples}




The next technical definition will be useful in some arguments. 



\begin{defn}  \label{genpos}
Let $\bar{K}=K_1,\dots,K_t$ be a sequence of measures with 
each $K_i$ of the form $W_1^{r_i}$ or $S_1^{r_i}$. Let $\bar{h}=h_1,\dots,h_t$
be a sequence of functions where $h_i \colon \dom(<_{r_i}) \to \omega_1$
is of the correct type if $K_i=S_1^{r_i}$ and $h_i \colon 
r_i \to \omega_1$ if $K_i=W_1^{r_i}$. Then we say the sequence 
$\bar{h}$ is in {\em general position} if it satisfies the following.

\begin{enumerate}
\item \label{genpos1}
If $i<j$ and $K_i=S_1^{r_i}$, then $h_j$ has range 
%is
\rc{mc15}{in} 
a set closed under 
the function $h_i(1) \colon \omega_1 \to \omega_1$. 
\item \label{genpos2}
If $i<j$ and $K_i=W_1^{r_i}$, then $\sup(\ran(h_i))<\min(\ran(h_j))$.
\end{enumerate}
\end{defn}




Note that (\ref{genpos1}) of definition~\ref{genpos} \
rc{mc}{implies} 
that if 
$i<j$ and both $K_i$, $K_j$ are of the form $S_1^r$, then 
$[h_i(1)]_{W_1^1} <[h_j(1)]_{W_1^1}$. In fact, $h_j(\alpha_1,\dots,\alpha_{r_j})
> h_i(1)(\alpha_{r_j})$ for all $\bar{\alpha} \in (\omega_1)^{r_j}$
(this is because $\alpha_{r_i}< h_j(\alpha_1,\dots,\alpha_{r_j})$ 
as $h_j$ is of the correct type). It is clear that the set of $\bar{h}$
in general position has measure one in the function space
product measure $\sK_1 \times \cdots \times \sK_t$.



We next reformulate the ordering $<$ on the descriptions in $\cD_m(\bar{K})$
in a purely syntactic manner. This is the content of the next lemma.


\begin{lem} \label{lemreform}
Fix $m$ and the measure sequence $\bar{K}=K_1,\dots, K_t$. 
Let $d, d' \in \cD_m(\bar{K})$. Then $d' < d$ iff $d' \ml d$
where $\ml$ is defined inductively through the 
\rc{mc}{following} 
cases.

\begin{itemize}


\item[I.]
Suppose $k'=k(d')<k(d)=k$.

\begin{itemize}

\item[1.]
$K_{k'}=W_1^{r'}$. In this case we set $d' \ml d$.
\item[2.]
$K_{k'}=S_1^{r'}$. In this case 
%$d'=h^{(s)}_{k'}(l+1)(d'_1,\dots,d'_{l'},d'_0)$.
$d'=h^{(s)}_{k'}(l+1)(d'_1,\dots,
%d'_{l'},
\rc{mc16?:$l'\to l$}{d'_{l}},
d'_0)$.
We define $d' \ml d$ to hold iff $d'_0 \ml d$. 

\end{itemize}


\item[II.]
Suppose $k'=k(d')>k(d)=k$. 


\begin{itemize}

\item[1.]
$K_k=W_1^r$. In this case we do not set $d' \ml d$.

\item[2.]
$K_k=S_1^r$. In this case $d= h^{(s)}_k(l+1)(d_1,\dots,d_l,d_0)$. We define  
$d' \ml d$ to hold iff $d' \ml d_0$ or $d'=d_0$.
\end{itemize}


\item[III.]  Suppose $k(d')= k(d)=k<\infty$. 

\begin{itemize}
\item[1.] $K_k=W_1^r$. In this case $d=\alpha_{k,p}$, $d'=\alpha_{k,p'}$.
We set $d' \ml d$ iff $p' <p$.


\item[2.] $K_k=S_1^r$. In this case $d=h^{(s)}_k(l+1)(d_1,\dots,d_l,d_0)$
and \newline $d'=h^{(s)}_k(l'+1)(d'_1,\dots,d'_{l'},d'_0)$.

\begin{itemize}

\item[a.] Suppose there is a least $j$ with $0 \leq j \leq l$ such that 
$d'_j \neq d_j$. Then we define $d' \ml d$ iff $d'_j \ml d_j$. In the remaining
cases assume there is no such $j$. 

\item[b.] If $l' <l$, then $d' \ml d$ iff $d'$ has the symbol $s$.

\item[c.] If $l' >l$ then $d' \ml d$ iff $d$ does not have the symbol $s$.

\item[d.] If $l'=l$, then $d' \ml d$ iff $d'$ has the symbol $s$ and $d$ does not.
\end{itemize}




\end{itemize}


\item[IV.] $k(d')=k(d)=\infty$. In this case $d'=\v{r'}$ and $d=\v{r}$.
We set $d' \ml d$ iff $r' <r$.

\end{itemize}


\end{lem}





\begin{proof}
It is clear by inspection that if $d' \neq d$, then either $d' \ml d$
or $d \ml d'$. So, it 
\rc{mc}{suffices} 
to show that that if $d' \ml d$ then $d' <d$. 
We show in fact that if $d' \ml d$ and if $\bar{h}=h_1,\dots, h_t$ are in general position,
then $(d';\bar{h}) <(d;\bar{h})$. That is, for $W_1^m$ almost all 
$\bar{\alpha}=(\alpha_1,\dots,\alpha_m)$ we have 
$(d';\bar{h})(\bar{\alpha})< (d;\bar{h})(\bar{\alpha})$. 
We prove this claim by reverse induction on 
$\min \{ k(d'), k(d)\}$. We suppose $d'\ml d$, and we use the functional representation for 
these descriptions in the following argument. 
If $k(d')=k(d)=\infty$, then $d'=\v{r'}$ and $d=\v{r}$. From IV of \ref{lemreform}
we have $r'<r $ and so $(d';\bar{h})(\bar{\alpha})=\alpha_{r'}<\alpha_r=
(d;\bar{h})(\bar{\alpha})$. In the remaining cases we assume  $\min\{ k(d'),k(d)\}<\infty$.


Assume next that $k'=k(d')<k(d)=k$. Suppose first that $K_{k'}=W_1^{r'}$. 
So, $d'$ is of the form $d'=\alpha_{k',p}$. 
Recall that this means that $(d';\bar{h})(\bar{\alpha})=h_{k'}(p)=
\alpha_{k',p}$ if we let $h_{k'}=(\alpha_{k',1},\dots, \alpha_{k',r'})$.
If $k<\infty$, then $\alpha_{k',p} < \min (\ran(h_k))$ as $\bar{h}$
is in general position, and since $(d;\bar{h})(\bar{\alpha})$ is in the closure 
of the range of $h_k$ (this is clear from the definition of $(d;\bar{h})(\bar{\alpha})$),
we have $\alpha_{k',p}=(d';\bar{h})(\bar{\alpha})<(d;\bar{h})(\bar{\alpha})$. 
Suppose next that $K_{k'}=S_1^{r'}$. So, $d'$ is of the form 
$d'=h_{k'}^{(s)}(l+1)(d'_1,\dots,d'_l,d'_0)$. From I.2.\ of lemma~\ref{lemreform}
we have $d'_0 \ml d$. By induction, for almost all $\bar{\alpha}$
we have $(d'_0;\bar{h})(\bar{\alpha})<(d;\bar{h})(\bar{\alpha})$. 
Since $k>k'$, $h_k$ has range in a set closed under $h_{k'}(1)$. So, 
$(d';\bar{h})(\bar{\alpha}) \leq h_{k'}(1)((d'_0;\bar{h})(\bar{\alpha})) 
<(d;\bar{h})(\bar{\alpha})$ (the first inequality follows from the definition of
$(d';\bar{h})(\bar{\alpha})$). 


Suppose next that $k'=k(d')>k(d)=k$. 
In this case we must have $K_k=S_1^r$. So, $d$ is of the form 
$d=h_k^{(s)}(l+1)(d_1,\dots,d_l,d_0)$. From II.2.\ of \ref{lemreform}
we have $d' \ml d_0$ or $d'=d_0$. If $d' \ml d_0$ then by induction 
$(d';\bar{h})(\bar{\alpha})< (d_0;\bar{h})(\bar{\alpha})$ for almost all 
$\bar{\alpha}$. But $ (d_0;\bar{h})(\bar{\alpha}) \leq  (d;\bar{h})(\bar{\alpha})$
from the definition of $(d;\bar{h})(\bar{\alpha})$. 
If $d'=d_0$, the result follows from $(d_0;\bar{h})(\bar{\alpha}) 
< (d;\bar{h})(\bar{\alpha})$. This follows from the definition of 
$(d;\bar{h})(\bar{\alpha})$ and the fact that if $d$ has the symbol $s$, then 
$l \geq 1$ (from the definition of pre-description). 
We are using here the fact that for almost all $\alpha_1<\cdots<\alpha_l<\alpha_0$ that 
$h_k^s(l+1)(\alpha_1,\dots,\alpha_l,\alpha_0) > \alpha_0$. Note that if we had allowed 
$l=0$ in this case (that is, if had allowed $d=h_k^s(1)(d_0)$ as a description),
then we would not have strict inequality here. That is
$h_k^s(1)(\alpha)= \sup_{\beta<\alpha}(h_k(1)(\beta)) =\alpha$ almost everywhere. 



Suppose next that $k'=k(d')=k(d)=k$. 
If $K_k=W_1^r$, the result follows easily. So, assume $K_k=S_1^r$. 
Say $d'=h_k^{(s)}(l'+1)(d'_1,\dots,d'_{l'},d'_0)$ and 
$d=h_k^{(s)}(l+1)(d_1,\dots,d_l,d_0)$. We are in case III.2.\ of \ref{lemreform}. 
If III.2.a.\ of \ref{lemreform} holds, let $j \leq \min\{ l, l'\}$ be least such that 
$d'_j \neq d_j$, so $d'_j \ml d_j$. By induction, for almost all $\bar{\alpha}$
we have $(d'_j;\bar{h})(\bar{\alpha}) < (d_j;\bar{h})(\bar{\alpha})$. The result now
follws from the fact that for almost all $\alpha_1< \cdots< \alpha_{j-1}
<\alpha'_j <\alpha_j<\alpha_0$ and any $\alpha'_j<\alpha'_{j+1}<\cdots < \alpha'_{l'}<\alpha_0$
and $\alpha_j<\alpha_{j+1}<\cdots <\alpha_l<\alpha_0$ that 

\begin{equation*}
\begin{split}
&h_k^{(s)}(l'+1)(\alpha_1,\dots,\alpha_{j-1},\alpha'_j,\dots,\alpha'_{j+1},\dots,\alpha'_l,
\alpha_0)  \\ & \qquad 
< h_k^{(s)}(l+1)(\alpha_1,\dots,\alpha_{j-1},\alpha_j,\alpha_{j+1},
\dots,\alpha_l,\alpha_0),
\end{split}
\end{equation*}


\noindent
where here either side may or may not have the symbol $s$. This inequality
uses the fact that $h_k$ is order-preserving from $\dom(<_r)$ to $\omega_1$.

In case III.2.b.\ of \ref{lemreform}, $l'<l$ and $d'$ has the symbol $s$. 
The result follows from the fact that for almost all 
$\alpha_1< \cdots <\alpha_l <\alpha_0$ that 
$$
h_k^s(l'+1)(\alpha_1,\dots,\alpha_{l'},\alpha_0)
< h_k^{(s)}(l+1)(\alpha_1,\dots, \alpha_{l'},\dots,\alpha_l,\alpha_0).
$$

\noindent
In III.2.c.\ we have $l'>l$ and $d$ does not have the symbol $s$. 
The result follows from the fact that for almost all 
$\alpha_1< \cdots <\alpha_{l'} <\alpha_0$ that

$$
h_k^{(s)}(l'+1)(\alpha_1,\dots,\alpha_l,\dots,\alpha_{l'},\alpha_0)
< h_k(l+1)(\alpha_1,\dots, \alpha_l,\alpha_0).
$$

\noindent
Finally, in case III.2.d.\ we have $l=l'$ and $d'$ has the symbol $s$
while $d$ does not. The result follows from the fact that
for almost all $\alpha_1<\cdots <\alpha_l <\alpha_0$ that 
%
\rc{mc17!: $l>0$}{}

$$
h_k^{(s)}(l+1)(\alpha_1,\dots,\alpha_l,\alpha_0)
< h_k(l+1)(\alpha_1,\dots, \alpha_l,\alpha_0).
$$
\end{proof}








In \cite{J1}, the set of descriptions $\cD$ was extended
to a set $\overline{\cD}$, and a property called
``condition D'' was introduced. Here, we have no need of
$\overline{\cD}$, and condition D 
\rc{mc}{simplifies} 
to a
fairly trivial condition. Nevertheless, to maintain consistency with
\cite{J1} we define:



\begin{defn}[Condition D]
If $d=d^{(f_m)} \in \cD_m(K_1,\dots,K_t)$, then
 we say $d$ satisfies condition D if $d > \cdot_m$.
\end{defn}



If $d$ satisfies condition D, then $\forall^\ast h_1, \dots
\forall^\ast h_t\ (d;h_1,\dots,h_t)> \omega_{m}$, that is,
$\forall^\ast h_1,\dots$, $h_t$ $\forall^\ast\alpha_1,\dots,\alpha_m$
$(d;h_1,\dots,h_t)(\alpha_1,\dots,\alpha_m) > \alpha_m$.
The significance of this is explained in remark~\ref{sig} below.




Next, we show how to use descriptions to generate equivalence classes
of functions from $\bd^1_3$ to $\bd^1_3$ with respect to the
measures $W^m$ (in \cite{J1}, the measures $W_3^m$ were used).

\begin{defn}[Ordinal represented by description]\label{drepr}
Fix $m\in\omega$, and let $d
%=\in 
\rc{mc18}{\in} 
{\cD}_m(K_1,\dots,K_t)$
satisfy condition D. Let $g\colon \bd^1_3 \to \bd^1_3$
be given.
\begin{itemize}
\item
We define  $(g;d;W^m;K_1,\dots,K_t)$ to be the ordinal
represented  w.r.t.\ $W^m$ by the function which assigns to
the tuple $(\dots, [f]_{S^\pi_1}, \dots)$ represented by 
$f \colon \omega_{m+1} \to \bd^1_3$ of continuous  type 
the value  $(g;d;f;\bar{K})$. 

\item
$(g;d;f;\bar K)$ is represented w.r.t.\ $K_1$ by the function which
assigns to $[h_1]$ the ordinal $(g;d;f;h_1,K_2,\dots,K_t)$.
\item
In general, $(g;d;f;h_1,\dots,h_{i-1},K_i,\dots,K_t)$ is represented
w.r.t.\ $K_i$ by the function which assigns to $[h_i]$ the ordinal
$$(g;d;f;h_1,\dots,h_{i-1},h_i,K_{i+1},\dots,K_t).$$
\item
Finally, $(g;d;f;h_1,\dots,h_t)= g(f((d;h_1,\dots,h_t)))$.
\end{itemize}
\end{defn}





\begin{rem} \label{sig}
If $d$ satisfies condition D, then $(g; d; W^m;\bar K)$ is well defined.
To see this, let $f,f'\colon \omega_{m+1} \to \bd^1_3$ be strictly
increasing, continuous, and represent the same tuple of ordinals, that is, 
$(\dots, [f]_{S^\pi_1}, \dots)=
(\dots, [f']_{S^\pi_1}, \dots)$. Then there is a c.u.b.\ 
$C \subseteq \omega_1$ such that for all permutations $\pi=
(m,i_1,\dots,i_m)$ and all functions $h\colon \dom(<^\pi) \to C$ of the
correct type, $f([h])=f'([h])$. Now, $$\forall^\ast h_1, \dots,
h_t \ \forall^\ast\alpha_1,\dots,\alpha_m \ (d;h_1,\dots,h_t)(\alpha_1,
\dots, \alpha_m) \in C.$$
This, in fact, holds for all $\bar{h}$ having range in $C$. 
Since $$\forall^\ast h_1,\dots,h_t\ 
(d;h_1,\dots,h_t)> \omega_{m},$$ it follows there is a permutation
$\overline{\pi}$ such that $\forall^\ast h_1,\dots,h_t$
$(d;h_1,\dots,h_t)$ can be represented by a function $h$ such that
either $h\colon \dom(<^{\bar{\pi}}) \to C$ is of the correct type, or
$[h]$ is the supremum of ordinals represented by
such functions (see the following remark). Since $f$, $f'$ are
continuous, in either case we have $f([h])=f'([h])$.
\end{rem}




\begin{rem}
We have used the following fact. If $h \colon (\omega_1)^m \to \omega_1$ 
is such that $[h]_{W_1^m}>\omega_m$ (i.e., $\forall_{W_1^m}^* \alpha_1,\dots,\alpha_m\ 
h(\alpha_1,\dots,\alpha_m)>\alpha_m$), then there is a {\em partial permutation}
$\pi$ beginning with $m$ which describes the ordering given by   $h$ on a c.u.b.\ set. 
That is, $\pi$ is of the form $\pi=(m,i_2,\dots,i_k)$ where $k \leq m$
and there is a c.u.b.\ $D \subseteq \omega_1$ such that for all 
$\vec \alpha, \vec \beta \in D^m$, $h(\vec \alpha) <h(\vec \beta)$ iff 
$(\alpha_m, \alpha_{i_2},\dots, \alpha_{i_k})<_\lex 
(\beta_m. \beta_{i_2},\dots, \beta_{i_m})$. The complete ``type''' of $h$ 
is determined  by $\pi$ and the 
\rc{mc}{specification} 
that $h$ is either continuous (on a c.u.b.\ set),
of uniform cofinality $\omega$, or $h(\vec \alpha)$ has uniform cofinality 
$\alpha_m$, $\alpha_{i_2}, \dots,$ or $\alpha_{i_k}$. 
%In all cases, if $h$ has range in $C'$ (the limit points of $C$), then $[h]$ is a limit of $[h']$ where $h' \colon \dom(<^{\pi'}) \to C$ is of type $\pi'$ for some permutation $\pi'$ extending $\pi$. 
\rc{Quesn1.1 !}{In all cases, if $h$ has range in $C'$ (the limit points of $C$),} 
%
then $[h]$ is a limit of $[h']$ where $h' \colon \dom(<^{\pi'}) \to C$ is of type $\pi'$ for some permutation $\pi'$ extending $\pi$. 
The reader can consult 
%\cite{} 
\rc{mc19?}{\cite{}} 
for more details.
\end{rem}










Next we introduce the lowering operator
$\sL$ on $\cD$.
For every description $d\in \cD_m(K_1,\dots,K_t)$,
$\sL$ applied to $d$ gives  the largest  description $\sL(d) \in
\cD_m(\bar{K})$ below $d$, except when $d$ is the (unique)
minimal description in $\cD_m(\bar{K})$ in which case $\sL(d)$
will be undefined. $\sL(d)$ depends on the measure sequence $\bar{K}$
as well, so we will write $\sL(d;\bar{K})$ when there is danger of confusion.



First, given measures $K_1,\dots,K_t$ and an integer $k$ (where $1\le k\le t$ or
$k=\infty$), an operator $\sL^k$ is defined on those $d$ satisfying
$k(d)\ge k$, except for a unique $d^k_{\min}$ which is called the {\em minimal}
description with respect to $\sL^k$. We then define $\sL = \sL^1$.
$\sL^k$ is defined by reverse induction on $k$ as follows (we write 
$\sL^k(d)$ throughout this definition on place of $\sL^k(d;\bar{K})$):

\begin{defn}[Operator $\sL^k$] Let $d \in \cD_m(K_1,\dots,K_t)$ with $k(d) \geq k$
where $1 \leq k \leq t$ or $k=\infty$.  We define $\sL^k(d)$ through the following cases. \label{defl}
\begin{itemize}
\item[I.] $k=\infty$.
So, $d$ is basic with $d=\cdot_p$
for $1\le p\le m$. If $p>1$, then set $\sL^{\infty}(d) = \cdot_{p-1}$.
If $p=1$, $d$ is minimal with respect to $\sL^{\infty}$.
\item[II.] $1\le k\le t$.
\begin{itemize}
\item[1.]      $k=k(d)$
\begin{itemize}
 \item[a.] $d$ is basic, so $d=\alpha_{k,p}$.
If $p>1$, then $\sL^k (d)=  \alpha_{k,p-1}$. If $p=1$, $d$ is minimal
with respect to $\sL^k$. 

\item[b.] $d=h_k(l+1)(d_1,\dots,d_l,d_0)$, with $l= r-1$
and  $K_k=S^r_1$.
Then $$\sL^k(d) = h^s_k(l+1)(d_1,\dots,
d_l,d_0)$$  if $l \geq 1$, and if $l=0$, that is, $d=h_k(1)(d_0)$,
then $\sL^k(d) =  d_0$.

\item[c.] $d$ as in (b), but $l< r-1$. If $\sL^{k+1}(d_0)$
is defined, and also  is $> d_l$ in case $l \geq 1$, then
$$\sL^{k}(d) =  h_k(l+2)(d_1, \dots, d_l, 
%\sL^{k+1}(d_l), 
\rc{mc20?: $d_l\to d_0$}{\sL^{k+1}(d_0)}, 
d_0).$$
If 
%\rc{mc20?: $d_l\to d_0$}{$\sL^{k+1}(d_l)$} 
\rc{mc20?: $d_l\to d_0$}{$\sL^{k+1}(d_0)$} 
is not defined, or is $\leq d_l$ (and $l \geq 1$),
then we set $\sL^{k}(d)= h^s_k(l+1)(d_1, \dots,d_l,d_0)$
if $l \geq 1$; otherwise $\sL^{k}(d) =  d_0$.

\item[d.] $d=h^s_k(l+1)(d_1, \dots,d_l,d_0)$.
If $\sL^{k+1}(d_l)$ is defined and also satisfies
%$\sL^{k+1}(d_{l-1})> d_{l}$ 
\rc{mc21}{$\sL^{k+1}(d_{l})> d_{l-1}$} 
if $l \geq 2$, set $$\sL^{k}(d) =
h_k(l+1)(d_1,\dots,d_{l-1},\sL^{k+1}(d_l),d_0).$$ Otherwise, set
$\sL^{k}(d) =  h^s_k(l)(d_1, \dots, d_{l-1},d_0)$
if $l \geq 2$, and for $l=1$, $\sL^{k}(d) =  d_0$.
\end{itemize}
\item[2.] $k<k(d), K_k=W^{r}_1$.

\begin{itemize}
\item[a.] $d$ is not minimal with respect to $\sL^{k+1}$.
Then $\sL^k(d) =  \sL^{k+1}(d)$. 
%
\rc{mc22?!: $k=t$?}{}
\item[b.] $d$ is minimal with respect to $\sL^{k+1}$.
Then $\sL^k(d)=  \alpha_{k,r}$.
\end{itemize}

\item[3.] $k<k(d), K_k=S_1^{r}$

\begin{itemize}
\item[a.] $d$ is not minimal with respect to $\sL^{k+1}$.
Then $\sL^k(d) =  h_k(1)(\sL^{k+1}(d))$.
\item[b.] $d$ is minimal with respect to $\sL^{k+1}$.
Then $d$ is minimal with respect to $\sL^k$.
\end{itemize}
\end{itemize}

\end{itemize}
\end{defn}


\begin{rem} \label{remonl}
$\sL(d)$ for $d=d^{(f_m)} \in \cD_m(\bar{K})$ depends only on 
$d$ and not on the superscript $(f_m)$. That is, if we regard 
$\cD_m(\bar{K}) \subseteq \cD_{m'}(\bar{K})$ for $m<m'$, then 
the $\sL$ operations agree on $\cD_m(\bar{K})$. 
So, we may unambiguously write $\sL(d;\bar{K})$.
\end{rem}



\begin{example}
For the sequence of measures $K_1=S^4_1$, $K_2=S^4_1$,
$K_3=S^3_1$, $K_4=W^4_1$, and $d^{(f_4)}=
h_1(3)(\alpha_{4,2}, h_2(2)(\alpha_{4,1}, \cdot_3), \cdot_4)$,
$$\sL(d)= h_1(4)(\alpha_{4,2}, h_2(2)(\alpha_{4,1}, \cdot_3),
h_2(1)(h_3(1)(\cdot_3)), \cdot_4).$$
\end{example}





\begin{lem}  \label{llemma}
If $d \in \cD_m(K_1,\dots,K_t)$ and $k(d)\geq k$, then $\sL^k(d)$ is defined except when $d$
is the unique minimal (with respect to $\sL^k$) description $d^k_{\min}$ 
in $\cD_m(K_1,\dots,K_t)$. Furthermore,
$\sL^k(d) \in \cD_m(K_1,\dots,K_t)$ (i.e., $\sL^k(d)$ is a description, not
just a pre-description). Also, $\sL^k(d)$ (if defined) is the largest
description $d'$ in $\cD_m(K_1,\dots,K_t)$ with $k(d') \geq k$ satisfying $d' <d$.
\end{lem}



\begin{proof}
The fact that $\sL^k(d)$ is a description, not just a pre-description,
follows immediately from definition~\ref{defl} in cases 
%\rc{mc23}{III}.1.b,c,d. 
\rc{mc23}{II}.1.b,c,d. 


Suppose $\sL^k(d')$ is not defined, where $k(d')\geq k$. We show that $d'$ is the 
minimal description $d''$ in $\cD_m(\bar{K})$ with $k(d'') \geq k$. 
Suppose first $K_k=W_1^r$. 
From II.1.a.\ and II.2.\ of definition~\ref{defl} we see that $\sL^k(d')$ is defined unless 
$d'=\alpha_{k,1}$. If $k(d)\geq k$ and $d \neq d'$, then $d' < d$
from cases I.1.\ and III.1.\ of lemma~\ref{lemreform}. 
Suppose next that $K_k=S_1^r$. 
In cases II.1.b,c,d of definition~\ref{defl} we see that $\sL^k(d')$
is always defined, so we must be in case~3 of \ref{defl}. Further, 
$d'$ must be minimal with respect to $\sL^{k+1}$ (note that $k(d') \geq k+1$ now). 
By induction $d'$ is $<$-minimal among those $d$ with $k(d) \geq k+1$. 
Suppose now $k(d) \geq k$ and $d \neq d'$. We show $d' < d$. 
If $k(d) \geq k+1$, then the result follows as $d'$ is $<$-minimal among
those $d$ with $k(d) \geq k+1$. So, 
\rc{mc}{suppose} 
$k(d)=k$. Say $d=h_k^{(s)}(l+1)(d_1,\dots,d_l,d_0)$. 
Since $k(d_0)\geq k+1$, we have $d' \leq d_0$. 
It follows that $d' < d$ from III.2.\ of lemma~\ref{lemreform}.



For the remainder of the proof we show that $\sL^k(d)$ is maximal among those $d' \in \cD_m(\bar{K})$
with $k(d') \geq k$ and $d' < d$. 
So, suppose $d' < d$, and $k(d') \geq k$. We show that 
$d' \leq  \sL^k(d)$. 





\rc{MajorC2.2!}{Suppose first that $k(d)=k$. If $K_k=W_1^r$, then $d$ is of the form} 
$d=\alpha_{k,p}$ where $p>1$. In this case $\sL^k(d)=\alpha_{k,p-1}$
from II.1.a.\ of definition~\ref{defl}. Since $d' <d$ we must have 
$k(d')=k$ from II.1.\ of lemma~\ref{lemreform} and so $d'=\alpha_{k,p'}$
where $p' <p$. So, from III.1.\ of lemma~\ref{lemreform} we have $d' \leq \sL^k(d)$. 
Assume next that $K_k=S_1^r$. So, $d=h_k^{(s)}(l+1)(d_1,\dots,d_l,d_0)$. 
Consider first the case $k(d')>k$. Since $d' < d$, from II.2.\ of
lemma~\ref{lemreform} we have $d' \leq d_0$. 
From cases II.1.b,c,d of definition~\ref{defl} we see that $\sL^k(d)$
is either of the form $d_0$, and we are done, or else of the form 
$h_k^{(s)}(
%l+1
\rc{mc24?}{l'+1}
)(d_1,\dots,d'_{l'},d_0)$ for some $l' \geq 1$. Since $k(d_0)>k$
in the second case, 
we have from II.2.\ of lemma~\ref{lemreform} that $d_0 < \sL^k(d)$
and so $d' \leq d_0 <\sL^k(d)$. Consider next the case $k(d')=k$. So now 
$d'=h_k^{(s)}(l'+1)(d'_1,\dots,d'_{l'}, d'_0)$. Suppose first that there is a least 
$j \leq \min \{ l, l'\}$ such that $d'_j \neq d_j$, so $d'_j <d_j$ from 
III.2.a.\ of lemma~\ref{lemreform}. If $j=0$ then 
by induction $d'_0 \leq \sL^{k+1}(d_0)<d_0 \leq \sL^k(d)$ (the last inequality 
holds since $\sL^k(d)$ is either $d_0$ or of the form $h_k^{(s)}(a+1)(\cdots, d_0)$). 
So, assume $j>0$. 
%If $l>j$ or $l=j$ and $d$ does not have the symbol $s$,
\rc{mc25!?}{If $l>j$ or $l=j$ and $d$ does not have the symbol $s$},
then $\sL^k(d)$ is of the form $h_k^{(s)}(a+1)(d_1,\dots,d_j,f_{j+1},\dots,f_a,d_0)$
for some $a \geq j$. Then $d' < \sL^k(d)$ follows from 
III.2.a.\ of lemma~\ref{lemreform}. If $l=j$ and $d$ has the symbol $s$, 
then since $d'_j<d_j$ we must have that $\sL^{k+1}(d_j)$ is defined and 
$d'_j \leq \sL^{k+1}(d_j)$. If $j \geq 2$ note that this also implies that 
%
\rc{mc26!}{}
$d_{j-1} =d'_{j-1}<d'_j \leq \sL^{k+1}(d_j)$.  So, $\sL^k(d)= 
h_k(j+1)(d_1,\dots, d_{j-1},\sL^{k+1}(d_j),d_0)$ from II.1.d.\ of definition~\ref{defl}. 
If $d'_j < \sL^{k+1}(d_j)$, then $d' < \sL^k(d)$ from III.2.a.\ of lemma~\ref{lemreform}. 
If  $d'_j = \sL^{k+1}(d_j)$, then $d' \leq \sL^k(d)$
follows from cases III.2.c,d of lemma~\ref{lemreform}. 
Suppose next that for all $j \leq \min \{ l,l'\}$ that $d'_j=d_j$. 
If $l' <l$, then from III.2.b.\ of lemma~\ref{lemreform} we have that 
$d'$ has the symbol $s$ (in this case $l'\geq 1$). 
Since 
%$l'>l$,  
\rc{mc27}{$l'<l$},  
$\sL^k(d)$ is of the form 
$h_k^{(s)}(a+1)(d_1,\dots, d_{l'}, f_{l'+1},\dots,f_a,d_0)$ for some $a \geq l'$. 
Then $d' \leq \sL^k(d)$ follows from III.2.b.\ of lemma~\ref{lemreform}.
If $l'>l$, then from III.2.c.\ of \ref{lemreform}, $d$ does not have the symbol $s$. 
Also, $d'_{l+1} <d'_0=d_0$ and $d'_{l+1} >d_l$ if $l>0$. Thus by induction 
$\sL^{k+1}(d_0)$ is defined and $\sL^{k+1}(d_0)\geq d'_{l+1}> d_l$ if $l>0$. 
%
\rc{mc28!$l=0$}{}
So, from 
II.1.c.\ of definition~\ref{defl} we have $\sL^k(d)=h_k(l+2)(d_1,\dots,d_l,\sL^{k+1}(d_0),d_0)$. 
By induction $d'_{l+1} \leq \sL^{k+1}(d_0)$ and so from cases III.2.a,d, of
lemma~\ref{lemreform} we have $d' \leq \sL^k(d)$. If $l'=l$ then 
from III.2.d.\ of \ref{lemreform}, $d'$ has the symbol $s$ and $d$ does not. 
We must have $l>0$ as otherwise $d'$ is not a description. 
Then $\sL^k(d)$ is of the form $h_k^{(s)}(a+1)(d_1,\dots,d_l,f_{l+1},\dots, f_a,d_1)$
for some $a \geq l$. 
From cases III.2.b,d of lemma~\ref{lemreform} we have $d' \leq \sL^k(d)$. 



Suppose next that $k(d)>k$. 
Assume first that $k(d')=k$. If $K_k=W_1^r$, then either $\sL^{k+1}(d)$
is defined 
%
\rc{mc29!$k=t$}{}
(and $\sL^k(d)=\sL^{k+1}(d)$) 
or else $\sL^k(d)=\alpha_{k,r}$ from II.2.b.\ of definition~\ref{defl}.
Since $d'=\alpha_{k,p}$ for some $p \leq r$, we have from either 
I.1.\ or III.1.\ of lemma~\ref{lemreform} that $d' \leq \sL^k(d)$. 
So, assume $K_k=S_1^r$. Thus, $d'$ is of the form 
$d'=h_k^{(s)}(l+1)(d'_1,\dots,d'_l,d'_0)$. From I.2.\ of \ref{lemreform}
we have $d'_0<d$. By induction, $d'_0 \leq \sL^{k+1}(d)$. 
From II.3.a.\ of definition~\ref{defl} we have $\sL^k(d)=
h_k(1)(\sL^{k+1}(d))$. We then from cases III.2.a,c,d of lemma~\ref{lemreform}
that $d' \leq \sL^k(d)$. 
Finally, assume $k(d')>k$. By induction, $d' \leq \sL^{k+1}(d)$. If 
$K_k=W_1^r$, then $\sL^k(d)=\sL^{k+1}(d)$ and we are done. If $K_k=S_1^r$
then $\sL^k(d)=h_k(1)(\sL^{k+1}(d))$ and from II.2.\ of lemma~\ref{lemreform}
we have 
%$\sL^{k+1}(d) <\sL^{k+1}(d)$ 
\rc{mc30}{$\sL^{k+1}(d) <\sL^{k}(d)$} 
in this case, so $d' <\sL^k(d)$.
\end{proof}


























\begin{defn}[Sup of a description] \label{defsup}
If $d \in \cD_m(K_1,\dots,K_t)$,
and $1\leq n\leq t$, then by $\sup_{K_n,\dots, K_t} (d)$ we
mean the description $d' \in \cD_m(K_1,\dots,K_t)$
defined as follows. 

\begin{enumerate}
\item \label{sup1}
If $d= \alpha_{i,j}$, then $d'= \alpha_{i,j}$ if $i <n$, and
$d'= \cdot_1$ if $i \geq n$.

\item \label{sup2}
If $d=\cdot_r$, then $d'=d$. 


\item \label{sup5}
If $d= h_k^{(s)}(l+1)(f_1, \dots, f_l,f_0)$, where
$k \geq n$, then
$d'= \cdot_{r+1}$ if $f_0=\cdot_r$, and otherwise $d'= f'_0=\sup_{K_n,\dots,K_t} (f_0)$.

\item \label{sup4}
If $d=h_k^{(s)}(l+1)(f_1, \dots,f_l,f_0)$ where $k<n$, then $d'=
f'_0=\sup_{K_n,\dots,K_t} (f_0)$ if $f'_0 \neq f_0$. Otherwise, let $i>0$ be least such
that $f'_i=\sup_{K_n,\dots,K_t} (f_i) \neq f_i$ (if such an $i$ does not exist, set 
$d'=d$).  If $f'_i < f_0$, set 
$$d'= h_k^s(i+1)(f_1, \dots, f_{i-1},f'_i, f_0).$$ If $f'_i\geq f_0$, set
$d'= h_k(i)(f_1, \dots, f_{i-1}, f_0)$ if $i>1$, and for
$i=1$, $d'= h_k(1)(f_0)$.
\end{enumerate}
\end{defn}



The following lemma gives the properties of the supremum of a description. 
Property~\ref{lemonsup5} in particular justifies the use of the term ``sup.''


\begin{lem} \label{supprop}
Let  $d \in \cD_m(K_1,\dots,K_t)$, and $1\leq n\leq t$. 
Then $d'=\sup_{K_n,\dots,K_t} (d)$ satisfies the following.

\begin{enumerate}
\item \label{lemonsup1}
$d' \in \cD_m(K_1,\dots, K_{n-1})$.
\item \label{lemonsup2}
If $d \in \cD(K_1,\dots,K_{n-1})$ then $d' =d$. 
\item \label{lemonsup2b}
$k(d')\geq k(d)$.
\item \label{lemonsup3}
$d' \geq d$.
\item \label{lemonsup4}
If $d$ satisfies condition D then so does $d'$.
\item \label{lemonsup5}
$\forall^*  h_1 \dots h_{n-1} \ 
\forall \alpha < (d'; h_1,\dots, h_{n-1})\ 
\forall^* h_n \dots h_t\  (\alpha< (d; h_1,\dots,h_t)).$
\end{enumerate}
\end{lem}


\begin{proof}
We prove properties (\ref{lemonsup1})-(\ref{lemonsup3}) by reverse induction on $k(d)$.
We let $d'$ abbreviate $\sup_{K_n,\dots,K_y}(d)$. 
We first prove property~(\ref{lemonsup1}). If $k(d)=\infty$ then 
$d=\v{r}$ in which case $d' = \v{r} \in \cD_m(K_1,\dots,K_{n-1})$. 
If $d=\alpha_{i,j}$ and $i<n$ then $d'=\alpha_{i,j} \in \cD_m(K_1,\dots,K_{n-1})$. 
If $i \geq n$ then $d'=\v{1} \in \cD_m(K_1,\dots,K_{n-1})$. 
If $d$ is of the form $d=h_k^{(s)}(l+1)(f_1,\dots,f_l,f_0)$ where $k\geq n$, then either 
$d'=\v{r+1} \in \cD_m(K_1,\dots,K_{n-1})$ or else $d'=f'_0=
\sup_{K_n,\dots,K_{t}}(f_0)$ which is in $\cD_m(K_1,\dots,K_{n-1})$
by induction. Suppose then that  $d=h_k^{(s)}(l+1)(f_1,\dots,f_l,f_0)$ where $k < n$. 
If $f'_0 >f_0$ (where again $f'_0$ denotes $\sup_{K_n,\dots,K_t}(f_0)$) then 
$d'=f'_0$ which is in $\cD_m(K_1,\dots,K_{n-1})$ by induction. 
Otherwise, as in  (\ref{sup4}) of definition~\ref{defsup} let $i>0$ 
be least such that $f'_i>f_i$. For $j<i$ we have 
$f_j=f'_j=\sup_{K_n,\dots,K_y}(f_j) $ which is in $\cD_m(K_1,\dots,K_{n-1})$
by induction. We either have $d'=h_k^s(i+1)(f_1,\dots,f_{i-1},f'_i,f_0)$
or $d'=h_k(i)(f_1,\dots,f_{i-1},f_0)$. or $d'=h_k(1)(f_0)$. So, in all cases 
$d' \in \cD_m(K_1,\dots,K_{n-1})$.




We next prove property~(\ref{lemonsup2}). Suppose $d \in \cD_m(K_1,\dots,K_{n-1})$. 
If $k(d)=\infty$ then $d=\v{r}$, and $d'=d$ from (\ref{sup2}) of definition~\ref{defsup}.
If $d=\alpha_{i,j}$ then $i <n$ and so $d'=d$ from (\ref{sup1}) of definition~\ref{defsup}.
So, suppose $d=h_k^{(s)}(l+1)(f_1,\dots,f_l,f_0)$. We must have $k<n$
since $d \in \cD_m(K_1,\dots,K_{n-1})$. Also, all the $f_i$ are in 
$\cD_m(K_1,\dots,K_{n-1})$ and so by induction $f'_i=f_i$ for all $0 \leq i \leq l$. 
From (\ref{sup4}) of definition~\ref{defsup} we have $d'=d$. 



We next prove property~(\ref{lemonsup2b}). If $k(d)=\infty$, than $d'=d$
so $k(d')=k(d)$. If $d=\alpha_{i,j}$ then $d'$ is either $\alpha_{i,j}$
so $k(d')=k(d)$ or $d'=\v{1}$ in which case $k(d')=\infty>k(d)$. 
Assume next that $d=h_k^{(s)}(l+1)(f_1,\dots,f_l,f_0)$. If $k\geq n$ 
then $d'$ is either $\v{r+1}$ (where $f_0=\v{r}$) and so $k(d')=\infty >k(d)$
or $d'=f'_0$. By induction, $k(f'_0) \geq k(f_0)$. So, $k(d')=k(f'_0)
\geq k(f_0) >k=k(d)$. Suppose next that $k<n$. If $f'_0 \neq f_0$ then 
$d'=f'_0$ and so $k(d')=k(f'_0) \geq k(f_0) >k=k(d)$ by induction. 
If $d'=d$ the result is trivial, so we may assume there is a least $i>0$
such that $f'_i \neq f_i$. In all cases of (\ref{sup4}) of definition~\ref{defsup}
we have that $d'$ is of the form $h^{(s)}_k(a)(\cdots)$ and so $k(d')=k=k(d)$. 







We next prove property~(\ref{lemonsup3}). If $k(d)=\infty$, then $d'=d$. 
If $d=\alpha_{i,j}$ and $i<n$ then $d'=d$. If $i\geq n$ then $d'=\v{1}>d$. 
So, assume $d=h_k^{(s)}(l+1)(f_1,\dots,f_l,f_0)$. First suppose $k\geq n$. 
If $f_0=\v{r}$, then $d'=\v{r+1}$ and $\v{r+1}> d$ from I.2.\ of 
lemma~\ref{lemreform}. Otherwise from (\ref{sup5}) of definition~\ref{defsup}
we have $d'=f'_0$. By induction $f'_0 \geq f_0$. Moreover, since 
$f_0 \notin \cD_m(K_1,\dots,K_{n-1})$ (as $k(f_0) >k(d)\geq n$ and 
$k(f_0)<\infty$) but $f'_0 \in \cD_m(K_1,\dots,K_{n-1})$ (by (\ref{lemonsup1})
of the lemma), we must have $f'_0\neq f_0$ and so $f'_0>f_0$. 
Also, $k(f'_0)\geq k(f_0) >k(d)$ by property~(\ref{lemonsup2b}) of the lemma. 
%If 
\rc{mc31,mc32!}{It} 
then follows from I.2.\ of lemma~\ref{lemreform} that $d <d'$. 










Property (\ref{lemonsup4}) is clear from 
(\ref{lemonsup3}). We do not give the complete details of the proof of 
(\ref{lemonsup5}), but rather illustrate the proof with an example
we consider next.
\end{proof}





\begin{example}
If $K_1=S^3_1$, $K_2=S^3_1$, $K_3=W^3_1$, $K_4=S^3_1$, and
$$d^{(f_4)}=h_1(2)(\alpha_{3,1}, h_2(2)(h_4(1)(\cdot_2), \cdot_3)),$$
then $d'=\sup_{K_3,K_4}(d)=  h_2(1)(\cdot_3)$.
To see property~(\ref{lemonsup5}), fix $h_1,h_2$ and $\alpha<\omega_5$
with $\alpha< (d';h_1,h_2)$. Let $(\beta_1,\dots,\beta_4) \mapsto 
\alpha(\beta_1,\dots,\beta_4)$ represent $\alpha$ with respect to 
$W_1^4$. Then $$\forall^*_{W_1^4} \beta_1,\dots,\beta_4\ 
\alpha(\beta_1,\dots,\beta_4)<(d';h_1,h_2)(\beta_1,\dots,\beta_4)=
h_2(1)(\beta_3).$$
From the definition of $h_2(1)$ it follows that 
$\forall_{W_1^4} \beta_1,\dots,\beta_4\ 
\exists 
%\gamma
\rc{mc33!}{\gamma} 
< \beta_3\ \alpha(\bar{\beta})< h_2(2)(\gamma, \beta_3)$. 
Since $\gamma(\beta_1,\dots,\beta_4)<\beta_3$, there is a function 
$g \colon \omega_1 \to \omega_1$ such that $\forall_{W_1^4} \bar{\beta}\ 
\gamma(\bar{\beta}) <g(\beta_2)$. Now, $\forall^*_{W_1^3} h_3 
\forall^*_{S_1^3} h_4\ ([h_4(1)]_{W_1^1} > [g]_{W_1^1})$.
Thus,

\begin{equation*}
\begin{split}
{\forall}^*_{W_1^3} h_3 \forall^*_{S_1^3} h_4\ \forall^*_{W_1^4} \bar{\beta}\ 
\alpha(\beta_1,\dots,\beta_4)& < h_2(2)(h_4(1)(\beta_2), \beta_3)\\ &
< h_1(2)(\alpha_{3,1}, h_2(2)(h_4(1)(\beta_2), \beta_3)) \\& 
= (d;\bar{h})(\bar{\beta}).
\end{split}
\end{equation*}



\end{example}


Note that it follows from lemma~\ref{supprop} that if 
$d \in \cD(K_1,\dots,K_t)$ then $d < d'=\sup_{K_n,\dots,K_t} d$
iff $ d \notin \cD_m(K_1,\dots,K_{n-1})$. For if 
$d=d'$ then by (\ref{sup1}) of lemma~\ref{supprop} we have 
$d \in \cD_m(K_1,\dots,K_{n-1})$. Conversely, if 
$d \in \cD_m(K_1,\dots,K_{n-1})$ then $d=d'$ by (\ref{sup2}).



\begin{lem} \label{suppropa}
Let $d \in \cD_m(K_1,\dots, K_n,\dots,K_t)$ and suppose 
$k(d)> n$. Then $\sup_{K_n,\dots,K_t} (d)=\sup_{K_{n+1},\dots,K_t} (d)$.
\end{lem}



\begin{proof}
By reverse induction on $k(d)$. If $d=\alpha_{i,j}$, then $i=k(d)>n$ and
so from (1) of definition~\ref{defsup} we have 
$\sup_{K_n,\dots,K_t} (d)=\sup_{K_{n+1},\dots,K_t} (d)= \v{1}$. 
If $d=\v{r}$, then from (2) of \ref{defsup} we have 
that both supremums are equal to $\v{r}$. 
If $d=h_i^{(s)}(l+1)(f_1,\dots,f_l,f_0)$, then $i=k(d)>n$. 
So, from (3) of \ref{defsup}, if $f_0=\v{r}$ then both 
supremums are equal to $\v{r+1}$. Otherwise, 
by induction we have that 
$\sup_{K_n,\dots,K_t} (f_0)=\sup_{K_{n+1},\dots,K_t} (f_0)$ (note that $k(f_0)
> k(d)>n$) 
and we are done since from (3) of \ref{defsup}
we have $\sup_{K_n,\dots,K_t} (d)=\sup_{K_n,\dots,K_t} (f_0)$
and  $\sup_{K_{n+1},\dots,K_t} (d)=\sup_{K_{n+1},\dots,K_t} (f_0)$.
\end{proof}












The next lemma mentions one more property of the supremum.




\begin{lem} \label{suppropb}
Let $d \in \cD_m(K_1,\dots,K_{n-1})$, $q \in \cD_m(K_1,\dots,K_{n-1}, K_n, \dots, K_t)$,
and suppose $q \leq d$ (both considered as elements of $\cD_m(K_1,\dots,K_t)$). 
Then $\sup_{K_n,\dots, K_t} (q) \leq d$. 
\end{lem}

\begin{proof}
If $q=d$ then $q \in \cD_m(K_1,\dots, K_{n-1})$ and
from (\ref{lemonsup2}) of lemma~\ref{supprop} it follows that $\sup_{K_n,\dots,K_t}(q)=d$. 
So, assume $q<d$. If $d < \sup_{K_n,\dots,K_t} (q)$, then from 
(\ref{lemonsup5}) of lemma~\ref{supprop} with 
$\alpha= (d;h_1,\dots,h_{n-1})$ we would have that 
$\forall^* h_1,\dots, h_t\ (d;\bar{h})< (q;\bar{h})$, that is, $d <q$, a contradiction. 



We can also prove the lemma in a purely syntactic manner. We give this proof. 
Let $q'=\sup_{K_n,\dots,K_t} (q)$. 
We may assume $q \notin \cD_m(K_1,\dots,K_{n-1})$ by (\ref{lemonsup2}) of lemma~\ref{supprop}. 
We proceed by reverse induction on $k=\min\{ k(q), k(d)\}$. 
If $k(q)<k(d)$, then $q$ must be of the form 
$q=h_k^{(s)}(l+1)(d_1,\dots,d_l,d_0)$ ($q$ cannot be of the form $\alpha_{k,j}$
as then $q \in \cD_m(K_1,\dots,K_{n-1})$). From I.2.\ of lemma~\ref{lemreform} 
we have that $d_0 < d$. By induction, $d'_0=\sup_{K_n,\dots,K_t} d_0\leq d$. 
If $d_0 \notin \cD_m(K_1,\dots,K_{n-1})$ then 
from (\ref{sup4}) of definition~\ref{defsup} we have that 
$q'= d'_0 \leq d$. If $d_0 \in \cD_m(K_1,\dots,K_{n-1})$, then let 
$i>0$ be least such that $d'_i=\sup_{K_n,\dots,K_t}(d_i) > d_i$, that is, 
$d_i \notin \cD_m(K_1,\dots,K_{n-1})$. From (\ref{sup4}) of definition~\ref{defsup}
we have that $q'= h_k^s(i+1)(d_1,\dots, d_{i-1},d'_i,d_0)$ or else 
$q'=h_k(i)(d_1,\dots,d_{i-1}, d_0)$. In either case, from 
I.2.\ of lemma~\ref{lemreform} we have that $q' <d$.

If $k(q) >k(d)=k$, then $d$ is of the form 
$d=h_k^{(s)}(l+1)(d_1,\dots, d_l,d_0)$. From II.2.\ of lemma~\ref{lemreform}
we have $q \leq d_0$. By induction, $q' \leq d_0$. Since $d_0 < d$, $q' < d$.

Finally, suppose $k=k(q)=k(d)$. So, $q=h_k^{(s)}(l_q+1)(q_1,\dots,q_{l_q},q_0)$
and $d=h_k^{(s)}(l_d+1)(d_1,\dots,d_{l_d},d_0)$. First assume that there is an
$l \leq \min\{ l_q,l_d\}$ such that $q_l \neq d_l$, and let $l$ be the least such. 
In particular, for $i<l$ we have $q_i \in \cD_m(K_1,\dots,K_{n-1})$. 
If $q_l \in \cD_m(K_1,\dots,K_{n-1})$ then $q'$ is of the form 
$h_k(l+1)(q_1,\dots,q_l,q_0)$ or $h_k^{(s)}(r+1)(q_1,\dots,q_l, \dots, q_r,q_0)$ for some $r>l$
(from (\ref{sup4}) of definition~\ref{defsup}). In either case,
$q'<d$ from III.2.a of lemma~\ref{lemreform}. So, suppose 
$q_l \notin \cD_m(K_1,\dots,K_{n-1})$. Then from (\ref{sup4}) of 
definition~\ref{defsup}  we have $q'= h_k^s(l+1)(q_1,\dots,q'_l,q_0)$
where $q'_l=\sup_{K_n,\dots,K_t} q_l$. We use here the fact that by induction, 
$q'_l\leq  d_l <d_0=q_0$.
From 
%III.2.b. and III.2.d. 
\rc{mc34!}{III.2.b. and III.2.d.} 
we have $q'\leq d$. Next assume that for all 
$l \leq \min\{ l_q,l_d\}$ we have $q_l=d_l$. We must have $l_q >l_d$
as otherwise $q \in \cD_m(K_1,\dots,K_{n-1})$. 
From 
%III.2.c. 
\rc{mc34!}{III.2.c.} 
it follows that $d$ does not have the symbol $s$. 
From (\ref{sup4}) of definition~\ref{defsup} we have 
that $q'$ is either of the form $h_k^s(r+1)(q_1,\dots,q_r,q_0)$ for $r > l_d$
or $h_k(r+1)(q_1,\dots,q_r,q_0)$ for $r\geq l_d$. If $r=l_d$ then $q'=d$, and in the
other cases $q'<d$ from 
%III.2.c.\ 
\rc{mc34!}{III.2.c.}\ 
again.
\end{proof}



\begin{defn}[Cofinality  of $d$] \label{cofdef}
Let   $\bar{d}=(d;\bar{K})$ where $\bar{K}=K_1,\dots,K_t$ and $d \in \cD_m(K_1,\dots,K_t)$. 
We say $\bar{d}$ has cofinality $\kappa$ ($= \omega$,
$\omega_1$, or $\omega_2$) if $\forall^\ast h_1, \dots, h_t$
$\cof (d;h_1, \dots,h_t)  =\kappa$.
\end{defn}


The next lemma reformulates the cofinality of a description in a
purely syntactic manner. For the purposes of this paper, the reader can 
take the following lemma as the definition of $\cof(\bar{d})$. 


\begin{lem} \label{lemoncof}
Let  $\bar{d}=(d;\bar{K})$ where $\bar{K}=K_1,\dots,K_t$ and $d \in \cD_m(K_1,\dots,K_t)$. 
Then $\cof(\bar{d})$ is determined as follows.

\begin{enumerate}
\item \label{lemoncof1}
If $d=\alpha_{i,j}$, then $\cof(\bar{d})=\omega$.
\item  \label{lemoncof2}
If $d=\cdot_r$, then $\cof(\bar{d})=\omega_1$ if $r=1$, and $\cof(d)=
\omega_2$ if $r>1$.
\item  \label{lemoncof3}
If $d=h_i(l+1)(d_1, \dots, d_l,d_0)$, and $K_i=S^r_1$, then
$\cof(\bar{d})=\omega$ if $l=r-1$, and if $l< r-1$ then
$\cof(\bar{d})= \cf{ d_0}$.
\item  \label{lemoncof4}
If $d=h^s_i(l+1)(d_1, \dots, d_l,d_0)$, then
$\cof(\bar{d})= \cof (d_l)$.
\end{enumerate}
\end{lem}


Note that $\cof(\bar{d})$ depends on the measure sequence $\bar{K}$ 
as well as $d$. However, it is clear form the definition that if 
$d \in \cD_m(\bar{K})$ and $\bar{K}'$ is a measure sequence extending 
$\bar{K}$, then $\cof(d;\bar{K})=\cof(d;\bar{K}')$. Later we will be considering
extensions of measure sequences, but we will not be changing the individual measures.
Thus, we will generally just write $\cof(d)$. 




Again, we do not give a detailed proof of lemma~\ref{lemoncof} but illustrate the 
proof with an example.


\begin{example}
\rc{MajorC2.3 !}{We use the same measure sequence} $\bar{K}$ and description $d^{(f_4)}$
from the previous example. According to lemma~\ref{lemoncof} we should have 
$\cof(d)=\cof(h_2(2)(h_4(1)(\cdot_2), \cdot_3))=\cof(\cdot_3)=\omega_2$.
To see this, consider four functions $h_1,\dots,h_4$
of the appropriate types corresponding to $K_1,\dots.K_4$. We show that 
the ordinal $(d;h_1,\dots,h_4)$ has cofinality $\omega_2$. Suppose 
$\alpha<(d;\bar{h})$. Representing $\alpha$ with respect 
%tp
\rc{mc35}{to} 
$W_1^4$ we have 
$$
\forall^*_{W_1^4} \beta_1,\dots,\beta_4\ 
\alpha(\beta_1,\dots,\beta_4)< (d;\bar{h})(\bar{\beta})=
h_1(2)(\alpha_{3,1}, h_2(2)(h_4(1)(\beta_2), \beta_3)).$$
Since $K_1=S_1^3$ (i.e., $K_1=S_1^r$ with $r=3$), and 
this description begins with $h_1(2)$, that is, begins
with an invariant smaller than the $r$ value, we have that there is a function $g$ 
with $[g]_{W_1^4}< (h_2(2)(h_4(1)(\cdot_2), \cdot_3); \bar{h})$
such that 
$$
\forall^*_{W_1^4} \beta_1,\dots,\beta_4\ 
\alpha(\beta_1,\dots,\beta_4)< 
h_1(3)(\alpha_{3,1}, g(\bar{\beta}), h_2(2)(h_4(1)(\beta_2), \beta_3)).$$
Since $K_2=S_1^3$, and we use $h_2(2)$ in the bound for $[g$], we have that there is a function 
$g_2$ with $[g_2]_{W_1^4}< (\cdot_3;\bar{h})$ such that
$$
\forall^*_{W_1^4} \bar{\beta}\ 
\alpha(\bar{\beta})< 
h_1(3)(\alpha_{3,1}, h_2(3)(h_4(1)(\beta_2), g_2(\bar{\beta}), \beta_3), 
h_2(2)(h_4(1)(\beta_2), \beta_3)).$$
Finally, $(\cdot_3;\bar{h})$ has cofinality $\omega_2$, since if 
$[g_2]_{W_1^4} <(\cdot_3;\bar{h})$ then $\forall_{W_1^4} \bar{\beta}\ 
g_2(\bar{\beta}) <\beta_3$
%
\rc{mc36}{.}
 So, there is a function $h \colon \omega_1
\to \omega_1$ such that $\forall_{W_1^4} \bar{\beta}\ 
g_2(\bar{\beta})<h(\beta_2)$. The map sending $[h]_{W_1^1}$ to
the $[\bar{\beta} \mapsto h(\beta_2)]_{W_1^4}$ is thus cofinal
from $\omega_2$ to $(\cdot_3;\bar{h})$ (this last part is just the argument that 
$\omega_4=(\cdot_3;\bar{h})$ has cofinality $\omega_2$). Altogether,
we have produced a cofinal map from $\omega_2$ into $(d;\bar{h})$,
so $(d;\bar{h})$ has cofinality $\omega_2$.
\end{example}





\begin{prop}\label{prop1}
Let $\bar{p}=(p;\bar{S})=(p;S_1,\dots,S_t)$ where $p \in \cD_m(\bar{S})$. 
Let $\bar{\bar{p}}=(p;\bar{S},K)$ where 
$K=S^n_1$ if
$\cof(p)=\omega_2$, and $K=W^n_1$ if $\cof(p)=\omega_1$.
Let $j \leq k(p)$. 
If $\cof(p)=\omega_2$, then
$k_{t+1}$, which represents the function corresponding to $K$, occurs in the
functional representation of $\sL^j(\bar{\bar{p}})$, that is, 
$\sL^j(\bar{\bar{p}}) \notin \cD_m(K_1,\dots,K_t)$. 
If $\cof(p)=\omega_1$, then
$\gamma_{t+1,n}$, which represents the largest ordinal  corresponding to $K$,
occurs in the  functional representation of $\sL^j(\bar{\bar{p}})$. 
\end{prop}



\begin{proof}
By reverse induction on $k(p)$. 
We suppose $\cof(p)=\omega_2$, the other case being similar.
Then $K=S^n_1$ for some $n\geq 1$.


% % % % % % % % % % % % % % % % % % % % % % % % % % % % % % % % %
%\rc{}{\enlargethispage{0.5\baselineskip}}


%If $k(p)=\infty$, then
\rc{MajorC2.4?!}{\enlargethispage{0.5\baselineskip}If $k(p)=\infty$, then} 
$p=\cdot_r$. Then $r>1$ as $\cof(p)=\omega_2$ and then $k_{t+1}(1)(\cdot_{r-1})$ is a
subdescription of $\sL^j(\bar{\bar{p}})$ from cases I and II.3.a.\ in 
definition~\ref{defl} of the $\sL$ operation.






If $k(p)<\infty$, then $p$ must be of the form
$p=h_i^{(s)}(l+1)(q_1,\ldots,q_l,q_0)$ (if $p=\alpha_{i,j}$ then $\cof(p)=\omega$).
Suppose first that $p$ has the symbol $s$, so $p$ is of the form 
$p=h^s_i(l+1)(q_1, \ldots,q_l, q_0)$. So, $j \leq k(p)=i$. 
Since $\cof(p)=\omega_2$,
we have $\cof(q_l)=\omega_2$. By induction, $k_{t+1}$ appears
in the functional representation of $\sL^{i+1}(q_l)$ 
($\sL^{i+1}(q_l)$ here is computed with respect to $\bar{S}\conc K$). 
Since $q_l$ is strictly greater than all $q_1,\dots,q_{l-1}$,
and $k(q_1),\dots,k(q_l) \geq i+1$, it follows from lemma~\ref{llemma} that 
$\sL^{i+1}(q_l)$ is greater than or equal to  all $q_1,\dots,q_{l-1}$.
Since $\sL^{i+1}(q_l)$ has $k_{t+1}$ in its functional representation,
and the others do not,  $\sL^{i+1}(q_l)$ is strictly greater than $q_1,\dots,q_{l-1}$.
Thus, $$\sL^i(\bar{\bar{p}})=h_i(l+1)(q_1
%
\rc{mc37}{,}
\dots, q_{l-1}, \sL^{i+1}(q_l), q_0)$$
from case II.1.d.\ in definition~\ref{defl} 
(we are assuming here that $l \geq 2$, the case $l=1$
is easier). 
So, $k_{t+1}$ occurs in the functional representation of $\sL^i(\bar{\bar{p}})$, and
it follows easily that 
%is 
\rc{mc38}{it} 
also therefore appears in the functional representation
of $\sL^j(\bar{\bar{p}})$. 



Suppose now that $p=h_i(l+1)(q_1,\dots,q_l,q_0)$. 
Then $h_i(l+1)$ is a proper invariant of $h_i$ (i.e., $K_i=S_1^{r_i}$ and 
$l+1< r_i$), as otherwise
$\cof(p)=\omega$. Also, $\cof(q_0)= \omega_2$, and so by induction $k_{t+1}$ appears in
the functional representation of $\sL^{i+1}(q_0)$. Arguing as in
the previous case, we have that $$\sL^i(\bar{\bar{p}})=
h_i(l+2)(q_1,\dots, q_l, \sL^{i+1}(q_0), q_0),$$ using case II.1.c.\ of 
definition~\ref{defl}, and we are done as before.
\end{proof}


The converse of proposition~\ref{prop1} is also true. We state this next.


\begin{prop} \label{prop1b}
Let $\bar{p}=(p;\bar{S})=(p;S_1,\dots,S_t)$ where $p \in \cD_m(\bar{S})$. 
If $\cof(p)=\omega$ then for any sequence $\bar{K}$ and any $j \leq k(p)$, 
if $\bar{\bar{p}}=(p;\bar{S},\bar{K})$, then $\sL^j(\bar{\bar{p}})$ (if defined) 
is in $\cD_m(\bar{S})$ (that is, $\sL^j(\bar{\bar{p}})$ does not involve any
of the measures from $\bar{K}$).
\end{prop}


\begin{proof}
\rc{mc}{Assume} 
$\cof(\bar{p})=\omega$. We proceed by reverse induction on $j$. 
If $j=\infty$ then  $p=\cdot_r$, and this case cannot occur from 
(\ref{lemoncof2}) of lemma~\ref{lemoncof}. Assume now $1 \leq j \leq t$. 
If $j<k(p)$,
then by induction $\sL^{j+1}(\bar{\bar{p}}) \in \cD_m(\bar{S})$
and then from II.2.\ and II.3.\ of definition~\ref{defl} it follows that 
$\sL^j(\bar{\bar{p}}) \in \cD_m(\bar{S})$ as well. 
Assume now that $j=k(p)$. 
If $p=\alpha_{i,j}$,
then from II.1.a.\ of \ref{defl}
we see that $\sL^j(\bar{\bar{p}}) \in \cD_m(\bar{S})$. 
Suppose $p=h_k(l+1)(d_1,\dots,d_l,d_0)$ with $K_k=S_1^r$.  
If $l=r-1$ then $\sL^j(\bar{\bar{p}})=h_k^s(l+1)(d_1,\dots,d_l,d_0)$
and the result is immediate. 
If $l<r-1$ then 
$\cof(p)=\cof(d_0)$. By induction, $\sL^{j+1}(d_0)$ (if defined;
here and below we mean computed with respect to the sequence $\bar{S},\bar{K}$)
is in $\cD_m(\bar{S})$. From II.1.c.\ of \ref{defl} 
we have that $\sL^j(\bar{\bar{p}})$ is 
either of the form $h_k^s(l+1)(d_1,\dots,d_l,d_0)$, or of the form 
$h_k(l+2)(d_1,\dots,d_l, \sL^{j+1}(d_0),d_0)$, or equal to $d_0$. In all cases,
these descriptions lie in $\cD_m(\bar{S})$. 
Suppose now that $p=h_k^s(l+1)(d_1,\dots,d_l,d_0)$. 
In this case $\cof(p)=\cof(d_l)$. By induction $\sL^{j+1}(d_l)$ 
if defined, lies in $\cD_m(\bar{S})$. From 
II.1.d.\ of \ref{defl} we have that $\sL^{j+1}(\bar{\bar{p}})$
is of the form $h_k(l+1)(d_1,\dots, \sL^{j+1}(d_l),d_0)$ or 
$h_k^s((l)(d_1,\dots,d_{l-1},d_0)$ or $d_0$. The result follows in all
cases. 
\end{proof}














\section{Representation of cardinals below $\bd^1_5$} \label{mainsec}

We state our main result.

\begin{thm} \label{mainthm}
Let $m >0$, $S_1, \dots, S_t \in \bigcup_i (W^i_1 \cup S^i_1)$
be a sequence of canonical measures. Let $d=d^{(f_m)} \in
\cD_m(S_1, \dots, S_t)$ be defined and satisfy condition D with respect to 
$S_1, \dots, S_t$. Then, $(\id;d;W^m;\bar{S})$ is a cardinal,
where $\id \colon  \bd^1_3 \to  \bd^1_3$ is the identity function.
\end{thm}



\begin{rem} \label{j1rem}
As mentioned previously, the converse is also true, that is,
every successor cardinal below the predecessor of $\bd^1_5$ is of 
the form $(\id;d;W^m;\bar{S})$. In \cite{J1} this 
%for 
\rc{mc39}{was} 
shown for the measures 
$W_3^m$, however the argument given there works also for the $W^m$. 
Namely, it was shown in \cite{J1} that if 
$g$ is strictly greater than the identity function (almost
everywhere with respect to an appropriate measure), then one can
show that $(g;d;W_3^m
%
\rc{mc40}{;}
\bar{S})$ is not a cardinal. This was, in fact,
the ``main theorem'' on descriptions from \cite{J1}. This argument 
carries over exactly for the measure $W^m$ (the measures $W^m$ or $W_3^m$ 
do not really play a role in the proof). 
\end{rem}









For the remainder of this paper, $\bar{d}$ (or $\bar{p}, \bar{q}$, etc.) will denote
a tuple $\bar{d}=(d;\bar{S})$, where  $\bar{S}$ is a sequence of measures 
each of which is in $\bigcup_i (W^i_1 \cup S^i_1)$, $m >0$ is an integer, and 
$d \in \cD_m(\bar{S})$.
The strategy of our proof is as follows.
First we will consider  for  $\bar{d}=(d;\bar{S})$
the list of descriptions $d=q_1>q_2> \cdots>q_n$ 
in $\cD_m(\bar{S})$ which satisfy condition~D. 
Each $\bar{p}=(p;\bar{S},\bar{K})$ where 
$p \in \cD_m(\bar{S},\bar{K})$ for some sequence of measures 
$\bar{K}$ (each of which is in $\bigcup_i (W^i_1 \cup S^i_1)$)
will be  naturally associated to one of the $q_i$. The set of $\bar{p}$
associated to a certain $q_i$ will be called the {\em block} of $q_i$. 
We will assign ordinals to the $\bar{p}$ which will in turn assign 
an ordinal to each block which we will call the {\em depth}
of the block $q_i$.  
Being added in a proper way these depths will give an ordinal $\xi_{\bar{d}}$.
Then we will show that
$(\id;d;W^m;\bar S)=\al{\omega+\xi_{\bar{d}}}+1$.



We recall some terminology associated with the upper-bound arguments
of \cite{J1}. 




\begin{defn}
Let $\bar{p}=(p; \bar{T})$, $\bar{q}=(q;\bar{U})$, where 
$p \in \cD_m(\bar{T})$ and $q \in \cD_m(\bar{U})$ for some $m>0$. 
Then we define $\bar{q} \prec' \bar{p}$ to hold iff 
$q=\sL(p)$ and $\bar{U}= \bar{T} \conc K$ for some 
$K \in  \bigcup_i (W^i_1 \cup S^i_1)$. Here $\sL(p)$ denotes the $\sL$
operation with respect to the measures $\bar{T}$.
\end{defn}


\begin{rem}
From remark~\ref{remonl} we are justified in writing $\prec'$
instead of $\prec'_m$.
\end{rem}


\begin{defn}
Given $\bar{d}=(d;\bar{S})$ we let $T_{\bar{d}}$ be the tree with root node 
$\bar{d}$ and successor relation given by $\prec'$.  We further require all the nodes
of $T_{\bar{d}}$ to have descriptions which satisfy condition~D. 
We let $\prec$ be the tree relation in $T_{\bar{d}}$. 
\end{defn}


Note that restricting to nodes satisfying condition~D has the effect of
restricting to a subtree. That is, if $\bar{q} \prec' \bar{p}$ and
$\bar{q}$ satisfies condition~D then so does $\bar{p}$. 



So, $\prec$ on $T_{\bar{d}}$ is the transitive closure of the $\prec'$
relation. $T_{\bar{d}}$ is the tree of finite descending chains in $\prec'$
starting from $\bar{d}$. As we move along a branch of $T_{\bar{d}}$
we successively apply the $\sL$ operation and then add a new measure
to the sequence. 
As in \cite{J1}, we define the rank function on the nodes of the tree $T_{\bar{d}}$
in the slightly non-standard manner
by $|\bar p| = (\sup_{\bar q \prec'\bar p}|\bar q|) + 1$, and $|T_{\bar{d}}|=|\bar d|$
(so the rank of all nodes is a successor ordinal).




The significance of the tree $T_{\bar{d}}$ lies in one of the main results 
of \cite{J1}. It was shown there that the ordinal $(\id;d;W^m;\bar{S})$
is at most the cardinal successor of the supremum of the 
ordinals $(\id;p;W^m;\bar{S},\bar{K})$ where $\bar{p}=(p;\bar{S},\bar{K})$ is an immediate
successor of $\bar{d}$ in $T_{\bar{d}}$. This is how the upper-bound
for $(\id;d;W^m;\bar{S})$ was obtained in \cite{J1} (we will be more precise later). 
The reader of this paper  does not need to be 
\rc{mc}{familiar} 
with the 
proofs of these results. 









Given a fixed $\bar{d}=(d;\bar{S})$, we employ a notational convention 
for $\bar{q}$ of the form $\bar{q}=(q;\bar{S},\bar{K})$. When writing the 
functional representation of such a $q$, 
we will use the symbols $h_i(j),
h^s_i(j)$, $\alpha_{i,j}$ when 
\rc{mc}{referring} 
to the measures in
$\bar{S}$, and $k_i(j)$, $k^s_i(j)$, $\gamma_{i,j}$ when
to the measures in $\bar{K}$. For example, if $\bar{S}=
(S^3_1,S^4_1,W^3_1)$, $\bar{K}= (S^4_1,W^4_1)$, then a
functional representation for $q=q^{(f_4)}$ might look like
$h_1(3)(\alpha_{3,1}, h_2(1)(k_4(2)(\gamma_{5,2}, \cdot_3)),
\cdot_4)$. 


Our main technical definition to follow is that of the o--sequence
of $\bar{q}=(q;\bar{S},\bar{K})$ relative to $\bar{d}=(d;\bar{S})$. 
This will be a sequence of formal terms of the form $k_i(\cdot_r)$
or $\gamma_{i,j}$. Such terms are 
essentially the functional
representations of certain descriptions (technically, $k_i(1)(\cdot_r)$
is a description, but this is a trivial notational difference). The ordering 
$<$ on descriptions thus orders these terms as well. Although it is just a
specialization of the general definition, we explicitly
give the definition next. We are justified in using the same symbol $<$
to denote this ordering. 



\begin{defn} \label{oord}
The ordering $<$ on terms of the form $\gamma_{i,j}$, $k_i(\cdot_r)$
is given as follows.

\begin{enumerate}
\item
$\gamma_{i,j}< \gamma_{k,l}$ iff $(i,j) <_{\lex} (k,l)$
\item
$\gamma_{i,j} <  k_l(\v r)$ for all $i,j,l,r$
\item
$k_i(\v r) <  k_j(\v s) \iff (r,i)<_{\lex}(s,j)$
\end{enumerate}
\end{defn}




The next definition is made for the ordering of terms of 
definition~\ref{oord}, but the definition is completely general and can be made for any 
linear order.


\begin{defn} \label{cis}
Let $\bar{v}=(v(0),v(1),\dots, v(l))$ be a sequence of terms of the form 
$k_i(\cdot_r)$ or $\gamma_{i,j}$. The {\em canonical increasing subsequence}
$\bar{v}'$ of $\bar{v}$ is defined by: $v'(i)=v(k_i)$, where $k_0=0$ and in general  
$k_{i+1}$ is the least $k> k_i$ such that 
$v(k)> v(k_i)$ (using the order of \ref{oord}). 
If such a $k$ does not 
%exists 
\rc{mc41}{exist}
then the definition stops at $v'(i)$. 
\end{defn}















\begin{defn} [The o--sequence of $q$, $\od{q}{\bar d}$]
\label{oseq} \ \newline
Let $\bar{d}=(d;\bar{S})$ where $d \in \cD_m(\bar{S})$, and 
let $q \in \cD_m(\bar{S};\bar{K})$. 
\rc{MajorC2.5!}{Then $\od{q}{\bar{d}}$} 
is the sequence
of terms of the form $k_i(\cdot_r)$ or $\gamma_{i,j}$
defined inductively as follows. If $q= \cdot_r$ or $q=\alpha_{i,j}$
we define $\od{q}{\bar{d}}=\emptyset$. If $q=\gamma_{i,j}$
we set $\od{q}{\bar{d}}=\gamma_{i,j}$. In the other cases $q$
is of the form $q=g(d_1,d_2,\dots,d_l,d_0)$ 
where $g$ stands for an invariant of some $h$ or some $k$ function
(with or without the symbol $s$). 
Note that each subdescription $d_i$ is defined relative to the same
sequence of measures $\bar S$, $\bar K$.
We define 


$$
\od{q}{\bar{d}}=
\begin{cases}
{[{\od{d_0}{\bar d}}^\smallfrown{\od{d_1}{\bar d}}^\smallfrown
\dots{}^\smallfrown \od{d_l}{\bar d}]}'
 & \text{if } g=h^{(s)}_i(l+1) \\
\od{d_0}{\bar d}  & \text{if } g=k^{(s)}_i(l+1) \text{ and } d_0\neq \cdot_r  \\
k_i(\cdot_r)  & \text{if } g=k^{(s)}_i(l+1) \text{ and } d_0= \cdot_r  \\
\end{cases}
$$
Here $'$ denotes the operation of definition~\ref{cis}.
\end{defn}













We define also a variation of $\od{q}{\bar d}$ which we denote
$\ods{q}{\bar d}$. This is defined exactly as $\od{q}{\bar d}$,
except that in the first case we do not apply the 
operation $'$ to the concatenated sequence. Now, each term $t=\gamma_{i,j}$
or $t=k_i(\cdot_r)$ may appear several times in the sequence.
For each such term $t$ we will attach superscripts
to the 
%
%occurences 
\rc{mc*2}{occurrences} 
%
of this term in $\ods{q}{\bar d}$. The 
%
%occurences 
\rc{}{occurrences} 
%
of this
term will thus be of the form $t^1, t^2,\dots, t^c$. 
The attachment of the superscripts is defined (inductively) as
follows. If $t^a, t^b$ both correspond to subdescriptions of
$p=g(p_1, \dots, p_l,p_0)$ (where $p$ is a subdescription of $q$)
then $a<b$ if $t^a$ corresponds to a subdescription of $p_i$ which
appears to the left of the subdescription $p_j$ corresponding to
$t^b$. If $t^a,t^b$ both correspond to subdescriptions of $p_i$,
the ordering of $a,b$ is given by induction. We officially consider the attached superscripts
to be part of the definition of $\ods{q}{\bar{d}}$.

\begin{example}
For 
$$q=h_1(3)(h_2(3)(k_2(\cdot_2), k_3(\cdot_2), \cdot_3),
\ h_2(3)(k_2(\cdot_2),  k_4(\cdot_3),\cdot_4),
\ h_2(3)(k_2(\cdot_2),  k_4(\cdot_3), \cdot_5)),$$
\begin{equation*}
\begin{split}
\od{q}{\bar d} =& (k_2(\cdot_2), k_4(\cdot_3)),\text{ and }\\ 
\ods{q}{\bar d}=&(k^3_2(\cdot_2), k^2_4(\cdot_3), k^1_2(\cdot_2),
k^1_3(\cdot_2), k^2_2(\cdot_2), k^1_4(\cdot_3)).
\end{split}
\end{equation*}
\end{example}

Note that we can recover $\od{q}{\bar{d}}$ from $\ods{q}{\bar{d}}$
by removing the superscripts and then taking the canonical increasing
subsequence.


Note also that $\od{q}{\bar d}$ and  $\ods{q}{\bar d}$ don't really depend on $d$
but only on the specification of which measures in the sequence for $\bar{q}$
are to be considered the $\bar{S}$ measures and which the $\bar{K}$ measures. 
Nevertheless, it is suggestive to write $\od{q}{\bar{d}}$ since 
we will be applying this definition to various $\bar{q} \in T_{\bar{d}}$, and
$\bar{d}=(d;\bar{S})$ will determine the initial segment of measures 
for $\bar{q}$.  We may write $\od{q}{\bar{d}}$ and $\od{\bar{q}}{\bar{d}}$
interchangeably. 



The o--sequence is the main technical tool in our main result, and perhaps
some comments about its intuition are in order. 
The intuitive idea behind the o--sequence is to ``linearize'' the description
with respect to the $\bar{K}$ measures. By linear we mean avoiding 
functional composition among the $k$ functions. Also in the intuition is the idea that
the $h$ functions (and ordinals $\alpha_{i,j}$) are fixed while the $k$ functions
(and $\gamma_{i,j}$ ordinals) are ``variable.'' Roughly speaking, this means that 
we think of a function $k_i$ as coming from an arbitrarily large measure $S_1^r$. 
The idea is that we should be able to replace terms involving composition
such as $k_i(k_j(\cdot_r))$ by just $k_j(\cdot_r)$ since as we take the supremum
over $j$ the ranks of two such descriptions should give the same value (this value
should not depend on $j$ either, but we need to keep the largest $k$ function
in the composition as an aid to comparing different terms). Moreover, 
it is important in our main 
%
%enbedding 
\rc{mc}{embedding} 
argument (lemma \ref{mainlem}) that 
compositions of the $k$ functions do not occur. 



The following easy proposition should help orient the reader.


\begin{prop} \label{propeasy}
Let $\bar{d}=(d;S_1,\dots, S_t)$ where $d \in \cD_m(\bar{S})$ 
and let $q \in \cD_m(\bar{S},\bar{K})$. Then the following are equivalent:

\begin{enumerate}
\item
$q \notin \cD_m(\bar{S})$.
\item
The functional representation of $q$ contains a $k^{(s)}_i$ or $\gamma_{i,j}$.
\item
$\od{q}{\bar{d}} \neq \emptyset$.
\end{enumerate}
\end{prop}



\begin{proof}
If $q \notin \cD_m(\bar{S})$, then $q$ must contain a subdescription involving
a measure in the $\bar{K}$ sequence. That is, it must either contain 
a subdescription beginning with $k_i^{(s)}$ or else contain a subdescription
of the form $\gamma_{i,j}$. Clearly if $q$ contains either of these terms in
its functional representation then $q \notin \cD_m(\bar{S})$. 
So, (1) is equivalent to (2). The fact that (3) implies (2) is obvious. 
To see that (1) implies (3) it is enough to show that 
$\ods{q}{\bar{d}}\neq \emptyset$, for  then $\od{q}{\bar{d}}$
is non-empty as well. We show this 
implication by reverse induction on $k(q)$. We cannot have 
$k(q)=\infty$ as then $q=\cdot_r \in \cD_m(\bar{S})$ which contradicts
(1). Suppose that $k(q) \leq t$. If $q=\alpha_{i,j}$
then we again violate (1). If $q=h_i^{(s)}(l+1)(q_1,\dots,q_l,q_0)$,
then for some $i$ we must have $q_i \notin \cD_m(\bar{S})$
and so by induction $\ods{q_i}{\bar{d}}\neq \emptyset$ and thus 
$\ods{q}{\bar{d}}\neq \emptyset$ as well. Assume now $k(q)>t$. 
If $q=\gamma_{i,j}$, then $\ods{q}{\bar{d}}=\gamma_{i,j} \neq \emptyset$. 
So suppose $q=k_i^{(s)}(l+1)(q_1,\dots,q_l,q_0)$. If $q_0=\cdot_r$
then $\ods{q}{\bar{d}}=k_i(\cdot_r) \neq \emptyset$. Otherwise, 
$\ods{q}{\bar{d}}=\ods{q_0}{\bar{d}}$. Since $k(q)<k(q_0)<\infty$,
$q_0 \notin \cD_m(\bar{S})$, and so by induction $\ods{q_0}{\bar{d}}
\neq \emptyset$. 
\end{proof}









For $\bar{d}=(d;\bar{S})$, $\bar{q}=(q;\bar{S},\bar{K})$,
we define $\sup_{{\bar d}} (q)= \sup_{\bar K} (q)$ for notational convenience, 









For future purposes, we note that the 
ordering of definition~\ref{oord} refines to an ordering
of terms from $\ods{q}{\bar{d}}$, that is, terms of the form 
$\gamma^a_{i,j}$ or $k^a_i(\cdot_r)$. We order as in 
\ref{oord}, using the superscript only to break ties. Formally, 
this is given in the following definition. We use the same symbol
$<$ to denote this ordering. 


\begin{defn} \label{osord}
The ordering terms of the form $\gamma^a_{i,j}$, $k^a_i(\cdot_r)$
is given as follows.


\begin{enumerate}
\item
$\gamma^a_{i,j}< \gamma^b_{k,l}$ iff $(i,j,a) <_{\lex} (k,l,b)$
\item
$\gamma^a_{i,j} <  k^b_l(\v r)$ for all $i,j,a,l,r,b$
\item
$k^a_i(\v r) <  k^b_j(\v s) \iff (r,i,a)<_{\lex}(s,j,b)$
\end{enumerate}
\end{defn}


We are now ready to proceed toward the definition of the ordinal 
$\xi_{\bar{d}}$. 


\begin{defn}[Level of $p$ with respect to $\bar{d}$] \label{level} 
Let $\bar{d}=(d;\bar{S})$ where $d \in \cD_m(\bar{S})$. Let 
$p \in \cD_m(\bar{S},\bar{K})$. We define $\lv{p}{\bar{d}}$, the 
{\em level} of $p$ with 
\rc{mc}{repsect} 
to $\bar{d}$, to be the countable ordinal
defined as follows. Assume first that $\od{p}{\bar{d}} \neq 
\emptyset$ (recall here proposition~\ref{propeasy}). 
Let $w=(w(0),\dots,w(l-1))=\od{p}{\bar{d}}$,
where $l$ is the length of the o--sequence.  
Recall each term $w(a)$ in this sequence is of the form 
$w(a)=k_i(\cdot_r)$ or $w(a)=\gamma_{i,j}$. Define 
$\# (w(a))= r$ in the first case, and $\# (w(a))=0$ in the second. 
Then we set (note this ordinal product  is written in reverse order):
$$\lv{p}{\bar{d}}= \prod^0_{i=l-1} \omega^{\omega^{\#w(i)}}=
\omega^{\omega^{\# w(l-1)}} \cdots \omega^{\omega^{\# w(0)}}.$$
If $\od{p}{\bar{d}}= \emptyset$, set $\lv{p}{\bar{d}}=1$.
\end{defn}


Intuitively, $\lv{p}{\bar{d}}$ attempts to compute the supremum
over the $\bar{K}$ measures of the rank of $\bar{p}=(p;\bar{S},\bar{K})$
in the tree $T'_{\bar{d}}$ which is defined just as $T_{\bar{d}}$
except all nodes $\bar{q}$ with $q<p$ and $q \in \cD_m(\bar{S})$
are declared terminal. Roughly speaking, we are computing the supremum
over $\bar{K}$ of the rank
to the next description defined with respect to just the $\bar{S}$
measures. 



\begin{example}
If $p=h_1(3)(h_2(2)(\gamma_{3,2},\v1),h_2(2)(\gamma_{3,1},k_4(\v1)) ,\v2)$,
then $$\od{p}{\bar d}=\langle \gamma_{3,2},k_4(\v1)
\rangle.$$
So, $\lv{p}{\bar d}=\omega^{\omega^{\# k_4(\v1)}} \cdot \omega^{\omega^{\#\gamma_{3,2}}}  
=\omega^\omega\cdot \omega=\omega^{\omega+1}$.
\end{example}



\begin{lem}  \label{num_of_lev}
Fix $\bar{d}=(d;\bar{S})$, where $d \in \cD_m(\bar{S})$. 
Then the set of levels $\{ \lv{p}{\bar{d}} \mid p \in \cD_m(\bar{S},\bar{K})
\text{ for some } \bar{K}\}$ is finite.
\end{lem}




\begin{proof} 
Let $\bar{S}=S_1,\dots,S_t$, and $d \in \cD_m(\bar{S})$. 
We first show by reverse induction on $k$ 
that there is a bound on the lengths of $\od{p}{\bar{d}}$
for the $p$ with $k(p) \geq k$. If $k >t$ and $k(p) \geq k$, then $p$
is of the form $p=k_i^{(s)}(l+1)(p_1,\dots,p_l,p_0)$,
$p=\cdot_r$, or $p=\gamma_{i,j}$. In all these cases, 
$\od{p}{\bar{d}}$ has length at most $1$. Suppose now $k \leq t$. 
If $S_k=W_1^{r_k}$, then if $k(p)\geq k$ either 
$k(p)>k$ (and these cases are bounded by induction) or 
$p=\alpha_{i,j}$ in which case $\od{p}{\bar{d}}=\emptyset$.
If $S_k=S_1^{r_k}$ and $k(p) \geq k$, then either $k(p)>k$
or $p$ is of the form 
$p=h_k^{(s)}(l+1)(p_1,\dots,p_l,p_0)$. 
From the definition, $\od{p}{\bar{d}}$ has length at most 
the sum of the lengths of the $\od{p_i}{\bar{d}}$. 
By induction these lengths are bounded, and since 
$l+1 \leq r_k$, the result follows. 
%%
\rc{mc42}{\par}
Next, observe that there are only finitely many
possibilities for $\#(w(a))$ for a term $w(a)$ of 
$\od{p}{\bar{d}}$. This is because any term of the o--sequence
of the form $k_i(\cdot_r)$ must have $r \leq m$ as 
$p \in \cD_m(\bar{S},\bar{K})$. The lemma now follows immediately. 
\end{proof}







We now group the $\bar{p}=(p;\bar{S},\bar{K})$ into blocks.





\begin{defn}[Block $\bl{q}{\bar d}$, $\dep{\bl{q}{\bar d}}$] \label{block}
Fix $\bar{d}=(d;\bar{S})$, $d \in \cD_m(\bar{S})$, with $d$ 
satisfying condition~D.
For  $q \in \cD_m(\bar{S})$, $q \leq d$, and $q$ satisfying condition~D, 
we define the {\em block}, $\bl{q}{\bar d}$, as the
set of all  $\bar{p}=(p;\bar{S},\bar{K})$ with
$\sup_{\bar{K}} (p)=q$ (here $p \in \cD_m(\bar{S},\bar{K})$, and we allow the $\bar{K}$
sequence to be empty). 
We also define the {\em depth} of a block by
$\dep{\bl{q}{\bar{d}}}= \max\{ \lv{p}{\bar{d}} \colon \bar{p} \in
\bl{q}{\bar{d}} \}$.
\end{defn}


Observe that the number of
blocks is the number of descriptions
$q \in \cD_m(\bar{S})$ satisfying condition~D, which is clearly finite.
Let us enumerate them in decreasing order: $d=q_1 > q_2 > \dots > q_n$.
Therefore the number of blocks is also finite and equal to  $n$.


Note that every  $\bar{p}=(p;\bar{S},\bar{K})$ 
%
\rc{mc43?}{$\le d$}
is in one of these blocks. 
This is immediate from (\ref{lemonsup1}) of
%proposition~\ref{supprop}.
\rc{mc43}{lemma~\ref{supprop}.}
Also, every block $\bl{q}{\bar{d}}$ 
is non-empty by (\ref{lemonsup2}) of 
%proposition~\ref{supprop}.
\rc{mc43}{lemma~\ref{supprop}.}




\begin{defn}  \label{xidef}
Let $\bar{d}=(d;\bar{S})$ where $d \in \cD_m(\bar{S})$ satisfies
condition~D. 
Let $q=q_1>q_2> \cdots >q_n$ enumerate the 
$q \in \cD_m(\bar{S})$ satisfying 
%condition~D.
\rc{mc44?}{condition~D, and $d\ge q$}. 
Then we define
$$\xi_{\bar{d}}= \dep{\bl{q_n}{\bar{d}}}+\dots+
\dep{\bl{q_3}{\bar{d}}}+
\dep{\bl{q_2}{\bar{d}}}.$$
\end{defn}


Note that we do not
include the topmost block in the sum defining $\xi_{\bar{d}}$.
The intuitive reason for that is that there are no nodes 
$\bar{p}$ in $T_{\bar{d}}$ which lie in the block of $q_1=d$,
since for any $\bar{p} \in T_{\bar{d}}$
%
\rc{mc45?}{and $\bar{p}\ne\bar{d}$} 
 we must have 
$p \leq \sL(\bar{d})$. 
The reader may wish to skip to the end of \S\ref{applications} 
where we consider an example which illustrates our definitions.



We next give three  facts about the o--sequence which we will
use in the following proposition. 


\begin{fact} \label{ef1}
Let $\bar{d}=(d;\bar{S})$ where $d \in \cD_m(\bar{S})$, and 
$\bar{p}=(p;\bar{S},\bar{K})$ with $p \in \cD_m(\bar{S},\bar{K})$. 
Suppose $p< \v{r}$. Then all terms of the o--sequence 
$\od{p}{\bar{d}}$ are of the form $\gamma_{a,b}$ or 
$k_a(\v{c})$ for some $c<r$.
\end{fact}


\begin{proof}
By reverse induction on $k(p)$. If $k(p)=\infty$ (i.e., $p=\v{r'}$),
then $\od{p}{\bar{d}}=\emptyset$ and there is nothing to prove. 
If $p=\alpha_{i,j}$ or $p=\gamma_{i,j}$ the result is also 
immediate. Suppose $p=h_i^{(s)}(l+1)(f_1,\dots,f_l,f_0)$. 
Since $p <\v{r}$, from I.2.\ of lemma~\ref{lemreform} 
we have $f_0 < \v{r}$. Also, from the definition of description we have 
$f_j <f_0$ for all $1 \leq j \leq l$. So, by induction $\od{f_0}{\bar{d}}$
and all the $\od{f_j}{\bar{d}}$ contain only terms of the form 
$\gamma_{a,b}$ or $k_i(\v{c})$ for $c<r$. Since 
$\od{p}{\bar{d}}=[\od{f_0}{\bar{d}} \conc \od{f_1}{\bar{d}} \conc 
\cdots \conc \od{f_l}{\bar{d}}]'$ in this case, the result then follows immediately
for $\od{p}{\bar{d}}$. 
Suppose next that 
$p=k_i^{(s)}(l+1)(f_1,\dots,f_l,f_0)$. If $f_0=\v{c}$
for some $c$, then $\od{p}{\bar{d}}$ is the single term $k_i(\v{c})$. 
Since $p <\v{r}$, it again follows from I.2.\ of lemma~\ref{lemreform}
that $c<r$. If $f_0$ is not of the form $\v{c}$, then 
$\od{p}{\bar{d}}=\od{f_0}{\bar{d}}$. From I.2.\ of \ref{lemreform} 
we have $f_0<\v{r}$. By induction, 
$\od{f_0}{\bar{d}}$ has only terms of the form 
$\gamma_{a,b}$ or $k_i(\v{c})$ for $c<r$, and the result follows for $\od{p}{\bar{d}}$. 
\end{proof}



\begin{fact} \label{ef2}
Let $\bar{d}=(d;\bar{S})$ where $d \in \cD_m(\bar{S})$, and $\bar{S}=S_1,\dots,S_t$.
Let also $\bar{p}=(p;\bar{S},\bar{K})$ with $p \in \cD_m(\bar{S},\bar{K})$. 
Suppose $p \geq \v{r}$ and $t<k(p)<\infty$. Then $\od{p}{\bar{d}}$
consists of a single term of the form $k_i(\v{c})$ where $c \geq r$ 
and $i \geq k(p)$. 
\end{fact}



\begin{proof}
Since $t <k(p)<\infty$, clearly $p \notin \cD_m(\bar{S})$
and so by proposition~\ref{propeasy} we have $\od{p}{\bar{d}} 
\neq \emptyset$. Since $k(p)>t$ it is also clear from the definition of the
o--sequence that $\od{p}{\bar{d}}$ consists of a single term. 
We prove the fact by reverse induction on $k(p)$. 
The case $p=\gamma_{i,j}$ cannot occur, as then $p<\v{1} \leq \v{r}$. 
So, suppose $p=k_i^{(s)}(l+1)(f_1,\dots,f_l,f_0)$. If 
$f_0$ is of the form $\v{c}$, then $\od{p}{\bar{d}}=k_i(\v{c})$. 
Since $p \geq \v{r}$ we have from I.2.\ of lemma~\ref{lemreform} 
that $\v{c} \geq \v{r}$ which means $c \geq r$ from IV of lemma~\ref{lemreform}. 
Also, in this case 
$i=k(p)$. If $f_0$ is not of the form $\v{c}$, then 
$\od{p}{\bar{d}}=\od{f_0}{\bar{d}}$. From I.2.\ of \ref{lemreform}
again we have that $f_0 \geq \v{r}$. By induction we have 
$\od{f_0}{\bar{d}}$ is of the form $k_i(\v{c})$ where $c \geq r$
and $i \geq k(f_0)> k(p)$. 
\end{proof}















\begin{fact} \label{ef3}
Let $\bar{d}=(d;\bar{S})$ where $d \in \cD_m(\bar{S})$
and $\bar{S}=S_1,\dots,S_t$. Let also  
$\bar{p}_1=(p_1;\bar{S},\bar{K})$, 
$\bar{p}_2=(p_2;\bar{S},\bar{K})$, where 
$p_1,p_2 \in \cD_m(\bar{S},\bar{K})$. Suppose 
$t< k(p_1),k(p_2) <\infty$.
If $p_1 \leq p_2$ then $\od{p_1}{\bar{d}} \leq \od{p_2}{\bar{d}}$
in the ordering of terms given in definition~\ref{oord}.
\end{fact}



\begin{proof}
As in fact~\ref{ef2}, $\od{p_1}{\bar{d}}$ and $\od{p_2}{\bar{d}}$
each consist of a single term. 
We prove the 
%proposition 
\rc{mc46}{fact} 
by reverse induction on $\min \{ k(p_1),k(p_2)\}$. 
If $p_1=p_2$ the result is trivial, so assume $p_1<p_2$. 
First assume $k(p_1)<k(p_2)$. If $p_1$ is of the form $\gamma_{i,j}$ 
(so $k(p_1)=i$) the result follows since either $\od{p_2}{\bar{d}}$
is of the form $k_a(\v{b})$ which is greater that the term $\gamma_{i,j}$,
or else $\od{p_2}{\bar{d}}=\gamma_{a,b}$ where $a \geq k(p_2) > k(p_1)=i$. 
In the latter case, $\gamma_{a,b}>\gamma_{i,j}$ as $a>i$. If
$p_1$ is of the form $p_1=k_i^{(s)}(l+1)(f_1,\dots,f_l,f_0)$, 
then from I.2.\ of lemma~\ref{lemreform} we have $f_0 <p_2$. 
Suppose first in this case that $f_0$ is of the form $f_0=\v{b}$. 
Thus, $\od{p_1}{\bar{d}}=k_i(\v{b})$. From II.2.\ of lemma~\ref{lemreform}
we have $\v{b} < p_1$ and so $\v{b} < p_2$. From fact~\ref{ef2} we have
that $\od{p_2}{\bar{d}}=k_j(\v{c})$ where $c \geq b$ and $j \geq k(p_2)
> k(p_1)=i$. So, $(c,j) >_\lex (b,i)$ and so from definition~\ref{oord}
we have $\od{p_2}{\bar{d}} > \od{p_1}{\bar{d}}$. Suppose next 
in this case that $f_0$ is not of the form $\v{b}$. 
Thus, $\od{p_1}{\bar{d}}=\od{f_0}{\bar{d}}$. Since $f_0<p_2$, by induction we have that
$\od{f_0}{\bar{d}} \leq \od{p_2}{\bar{d}}$ and we are done. 



Next assume that $k(p_1)>k(p_2)$. The case $p_2=\gamma_{i,j}$ cannot occur
as we would have then that $p_1 >p_2$. So assume 
$p_2=k_i^{(s)}(l+1)(g_1,\dots,g_l,g_0)$. From II.2.\ of lemma~\ref{lemreform}
we have $p_1 \leq g_0$. Suppose first that $g_0=\v{b}$, so $\od{p_2}{\bar{d}}
=k_i(\v{b})$. We cannot have $p_1=\v{b}$ since $k(p_1)<\infty$. 
So, $p_1 <\v{b}$. From fact~\ref{ef1}, and since $\od{p_1}{\bar{d}}$
consists of a single term, $\od{p_1}{\bar{d}}$ is either of the form 
$\gamma_{i,j}$ or $k_a(\v{c})$ where $c < b$. In either case we have 
from definition~\ref{oord} that $\od{p_1}{\bar{d}} <\od{p_2}{\bar{d}}$. 
Suppose next that $g_0$ is not of the form $\v{b}$. Thus, $\od{p_2}{\bar{d}}=
\od{g_0}{\bar{d}}$. Since $p_1 \leq g_0$, by induction we have 
$\od{p_1}{\bar{d}} \leq \od{g_0}{\bar{d}}=\od{p_2}{\bar{d}}$. 








Finally,  assume $k(p_1)=k(p_2)$. If $p_1=\gamma_{i,j}$, then $p_2=\gamma_{i,a}$
for some $a$. Since $p_1<p_2$ we have $j<a$ from 
III.1\ of lemma~\ref{lemreform}. So from definition~\ref{oord}
we have $\od{p_1}{\bar{d}}=\gamma_{i,j}<\gamma_{i,a}=\od{p_2}{\bar{d}}$. 
So, we may assume that $p_1$ and $p_2$ are of the forms 
$p_1=k_i^{(s)}(l+1)(f_1,\dots,f_l,f_0)$ and 
$p_2=k_i^{(s)}(l'+1)(g_1,\dots,g_{l'},g_0)$. From III.2.a.\ 
of lemma~\ref{lemreform} we must have $f_0 \leq g_0$. 
If $f_0=\v{b}$, $g_0=\v{c}$ for some $b,c$, then from IV
of lemma~\ref{lemreform} we have $b \leq c$. 
Thus, $\od{p_1}{\bar{d}}=k_i(\v{b}) \leq k_i(\v{c}) =\od{p_2}{\bar{d}}$. 
If $f_0$ is of the form $f_0=\v{b}$ and $g_0$ is not of this form, 
then $\v{b} \leq g_0$. From fact~\ref{ef2} we have $\od{g_0}{\bar{d}}
=k_a(\v{c})$ where $c \geq b$ and $a \geq k(g_0)>k(p_2)=i$. 
So, $(c,a)>_\lex (b,i)$ and thus $\od{p_1}{\bar{d}}=k_i(\v{b})
<k_a(\v{c})=\od{g_0}{\bar{d}}=\od{p_2}{\bar{d}}$. 
If $g_0$ is of the form $g_0=\v{c}$, and $f_0$ is not of this form, 
then $f_0 \leq \v{c}$. Since $f_0$ is not of this form, 
$f_0<\v{c}$. 
So, from fact~\ref{ef1} $\od{f_0}{\bar{d}}$
is a single term of the form $\gamma_{a,j}$ or $k_a(\v{b})$
where $b <c$. In either case, $\od{p_1}{\bar{d}} < k_i(\v{c})=
\od{p_2}{\bar{d}}$. Lastly, suppose neither $f_0$ nor $g_0$ is of the form 
$\v{b}$. Since $f_0 \leq g_0$, by induction we have that 
$\od{f_0}{\bar{d}} \leq \od{g_0}{\bar{d}}$ and we are done since 
$\od{p_1}{\bar{d}}=\od{f_0}{\bar{d}}$ and 
$\od{p_2}{\bar{d}}=\od{g_0}{\bar{d}}$.
\end{proof}







\begin{comment}
\begin{fact} \label{ef4}
Let $\bar{d}=(d;\bar{S})$ where $d \in \cD_m(\bar{S})$
and $\bar{S}=S_1,\dots,S_t$. 

\begin{enumerate}
\item \label{ef41}
Let   
$\bar{p}=(p;\bar{S},\bar{K})$, where $p<\v{r}$. 
If $\sup_{\bar{K}}(p)=\v{r}$ then $\od{p}{\bar{d}}$ 
is a single term of the form $k_a(\v{r-1})$ if $r>1$ and of the form $\gamma_{i,j}$
if $r=1$. 
If $\sup_{\bar{K}}(p)<\v{r}$ 
then $\od{p}{\bar{d}}$ is a
sequence (possible empty) of terms
each of which is of the form $\gamma_{i,j}$ or $k_a(\v{b})$ for $b<r-1$. 

\item \label{ef42}
If $p_1,p_2\in \cD_m(\bar{S},\bar{K})$ with  
$\sup_{\bar{K}}(p_1)=\sup_{\bar{K}}(p_2)=\v{r}$ and $p_1<p_2$, 
then $\od{p_1}{\bar{d}} < \od{p_2}{\bar{d}}$ in the ordering of $\ref{oord}$.
\end{enumerate}
\end{fact}


\begin{proof}
We show by reverse induction on $k$ that the statements hold for 
$p,p_1,p_2$ with $k(p), k(p_1)_,k(p_2)\geq k$. If $k=\infty$ then 
$p$ is of the form $\v{b}$ for $b<r$. In this case, $\sup_{\bar{K}}(p)=\v{b} \neq \v{r}$
and $\od{p}{\bar{d}}=\emptyset$ 
so the first statement of the fact holds. 
The second  statement also holds trivially, as 
$\od{p_1}{\bar{d}}$ and $\od{p_1}{\bar{d}}$ are also both empty. 
In the remaining cases assume $k<\infty$. 



Suppose next that $k\leq t$. If $k(p)>k$ then the first statement holds by induction,
so assume $k(p)=k$. Likewise, for the second and third staements we may assume that 
$\min\{ k(p_1),k(p_2)\}=k$. 
If $S_k=W_1^{r_k}$ then $p=\alpha_{k,b}$ and $\od{p}{\bar{d}}=\emptyset$.
Also, $\sup_{\bar{K}}(\alpha_{k,b})=\alpha_{k,b}< \v{r}$ so the first statement holds. 
Likewise, the second statement holds trivially as at least one of 
$\od{p_1}{\bar{d}}$, $\od{p_2}{\bar{d}}$ is empty.
So assume $S_k=S_1^{r_k}$. Say $p=h_k^{(s)}(l+1)(f_1,\dots,f_l,f_0)$. 
From I.2.\ of lemma~\ref{lemreform} we have $f_0<\v{r}$. so, 
by lemma~\ref{suppropb} we have $\sup_{\bar{K}}(f_0) \leq \v{r}$. 
Suppose first that $\sup_{\bar{K}}(f_0)=\v{r}$. 
By induction,
$\od{f_0}{\bar{d}}$ is a single term of the form $k_a(\v{r-1})$
(we use the notation for the case $r>1$, the case $r=1$ is similar).
Since $f_j <f_0$ for all 
$1 \leq j \leq l$ we have that $\sup_{\bar{K}}(f_j) \leq \sup_{\bar{K}}(f_0)$
(this uses property~\ref{lemonsup3} of property~\ref{supprop} and lemma~\ref{suppropb}).
By induction and (\ref{ef41}), (\ref{ef42}) 
of the fact, all of the $\od{f_j}{\bar{d}}$ all only contain terms of the form 
$\gamma_{a',b'}$ or  $k_{a'}(\v{c'})$ where $c'<r-1$ or $c'=r-1$ and $a' \leq a$. 
These terms are all less than or equal to $k_a(\v{r-1})$ in the order of \ref{oord}. 
Since $\od{p}{\bar{d}}=[\od{f_0}{\bar{d}}\conc \cdots \conc \od{f_l}{\bar{d}}]'$,
all of the term except the first term $\od{f_0}{\bar{d}}=k_a(\v{r-1})$ will be cancelled
and so $\od{p}{\bar{d}}=k_a(\v{r-1})$.  Also, we cannot have that 
$f_0 \in \cD_m(\bar{S})$ as otherwise since $f_0 < p<\v{r}$ (from II.2.\ of lemma~\ref{lemreform})
we would have $\sup_{\bar{K}}(f_0)=f_0 <\v{r}$ from  (\ref{lemonsup2})
of property~\ref{supprop}. So, $\sup_{\bar{K}}(f_0)>f_0$ and thus 
$\sup_{\bar{K}}(p)=\sup_{\bar{K}}(f_0) =\v{r}$ from (\ref{sup4}) of 
definition~\ref{defsup}. Thus we have verified statement (\ref{ef41})
of the fact. 


Suppose next that $\sup_{\bar{K}}(f_0)
<\v{r}$, so by induction $\od{f_0}{\bar{d}}$ consists only of terms 
$\gamma_{i,j}$ or $k_a(\v{b})$ for $b<r-1$. Since for $j>0$ we have 
$f_j<f_0$ we have $\sup_{\bar{K}}(f_j) \leq \sup_{\bar{K}}(f_0) <\v{r}$.
So by induction, all the $\od{f_j}{\bar{d}}$ contain only terms of the form 
$\gamma_{i,j}$ or $k_a(\v{b})$ for $b<r-1$ and thus the same is true for $\od{p}{\bar{d}}$. 
Also in this case, $\sup_{\bar{K}}(p)$ is either equal to $\sup_{\bar{K}}(f_0)$, which is less tham
$\v{r}$, or is of the form $h_k^{(s)}(l'+1)(\cdots,f_0)$. Since $f_0<\v{r}$,
it follows from I.2.\ of lemma~\ref{lemreform} that $\sup_{\bar{K}}(p)<\v{r}$.
This shows the first statement in the case $k \leq t$. 


To see the second statement, consider
$p_1, p_2$ as in the second statement. Assume first that $k=k(p_1)<k(p_2)$. 
Say $p_1=h_k^{(s)}(l+1)(f_1,\dots,f_l,f_0)$. 
From the proof of the first statement, $\sup_{\bar{K}}(f_0)=\v{r}$ and 
$\od{p_1}{\bar{d}}=\od{f_0}{\bar{d}}=k_a(\v{r-1})$
for some $a$. By induction we also have $\od{p_2}{\bar{d}}=k_b(\v{r-1})$ for some $b$.
By induction applied to $f_0$ and $p_2$, since $f_0 \leq p_2$ from I.2.\ 
of lemma~\ref{lemreform} we have $a \leq b$ which verifes the second 
statement of the fact for $p_1$ and $p_2$. Assume next that 
$k(p_1)>k(p_2)=k$. Say $p_2=h_k^{(s)}(l+1)(g_1,\dots,g_l,g_0)$. 
We must again have that $\sup_{\bar{K}}(g_0)=\v{r}$ and so by induction 
$\od{g_0}{\bar{d}}=k_b(\v{r-1})$ for some $b$. We have $p_1 \leq g_0$
from II.2.\ of lemma~\ref{lemreform}. By induction, $\od{p_1}{\bar{d}}=k_a(\v{r-1})$
and also $a \leq b$ by induction applied to $p_1$ and $g_0$. From the proof of the first statement,
$\od{p_2}{\bar{d}}=\od{g_0}{\bar{d}}=k_b(\v{r-1})$ and so $\od{p_1}{\bar{d}}
\leq \od{p_2}{\bar{d}}$. Assume finally that $k=k(p_1)=k(p_2)$. 
Say $p_1=h_k^{(s)}(l+1)(f_1,\dots,f_l,f_0)$, and 
$p_2=h_k^{(s)}(l'+1)(g_1,\dots,g_{l'},g_0)$. As above, we have that 
$\sup_{\bar{K}}(f_0)=\sup_{\bar{K}}(g_0)=\v{r}$. As in the proof of the first statement, 
$\od{p_1}{\bar{d}}=\od{f_0}{\bar{d}}$ and $\od{p_2}{\bar{d}}=\od{g_0}{\bar{d}}$. 
We must have $f_0 \leq g_0$ from III.2.a.\ of lemma~\ref{lemreform}. By induction,
$\od{f_0}{\bar{d}}=k_a(\v{r-1})$ and $\od{g_0}{\bar{d}}=k_b(\v{r-1})$
for some $a \leq b$ and the result follows. This verifies the second statement
in the case $k \leq t$. 





Suppose next that $k>t$. We may again assume that $k=k(p)$ and $k=\min\{ k(p_1),k(p_2)\}$. 
Suppose first that $K_k=W_1^{r_k}$. Then for $p$ as in the first statement we must
have $p=\gamma_{k,b}$. In this case $\sup_{\bar{K}}(p)=\v{1}$. Also, $\od{p}{\bar{d}}
=\gamma_{k,b}$, which verifies the first statement. For $p_1,p_2$ as in the second statement, 
at least one of $p_1$, $p_2$ must be of the form $\gamma_{k,b}$, and for this 
$p_i$ we have $\sup_{\bar{K}}(p_i)=\v{1}$. Thus, $r=1$ in the second statement. 
If $k=k(p_1)<k(p_2)$ then by induction $\od{p_2}{\bar{d}}$ consists of a single term of the form 
$\gamma_{i,j}$. We must have $i \geq k(p_2) >k$ and from definition~\ref{oord}
it follows that $\od{p_1}{\bar{d}}<\od{p_2}{\bar{d}}$. The case 
$k(p_1)>k(p_2)=k$ cannot happen as then $p_1>p_2$  from I.1.\ of lemma~\ref{lemreform}.
If $k(p_1)=k(p_2)=k$, then $p_2$ is of the form $\gamma_{k,c}$. Since $p_1 \leq p_2$
we have $b \leq c$ and so from \ref{oord}, $\od{p_1}{\bar{d}}\leq \od{p_2}{\bar{d}}$.


Suppose then that $k>t$ and $K_k=S_1^{r_k}$. To verify the first statement, 
say $p=k_k^{(s)}(l+1)(f_1,\dots,f_l,f_0)$. 
If $f_0$ is of the form $\v{b}$ for some $b$, then $\sup_{\bar{K}}(p)=
\v{b+1}$. If $b=r-1$, then by the definition of the o--sequence in this case, 
$\od{p}{\bar{d}}=k_k(\v{b})=k_k(\v{r-1})$. Also, in this case we have 
$\sup_{\bar{K}}(p)=\v{r}$ from (\ref{sup5}) of definition~\ref{defsup}.
This verifies the first statement in this case. 
If $b <r-1$, then $\sup_{\bar{K}}(p)=\v{b+1}<\v{r}$. 
Also, $\od{p}{\bar{d}}=k_k(\v{b})$ with $b<r-1$ which verifies the first statement.
If $f_0$ is not of the form $\v{b}$, then $\od{p}{\bar{d}}=\od{f_0}{\bar{d}}$. 
Since $f_0 < p$ (by II.2.\ of lemma~\ref{lemreform}), we have 
$\sup_{\bar{K}}(f_0) \leq \sup_{\bar{K}}(p)\leq \v{r}$. If $\sup_{\bar{K}}(f_0)
=\v{r}$ then by induction $\od{f_0}{\bar{d}}$ is a single term of the form 
$k_a(\v{r-1})$ (or $\gamma_{a,b}$ if $r=1$) and the result follows for $p$
since we also have $\sup_{\bar{K}}(p)=\sup_{\bar{K}}(f_0)=\v{r}$
(from (\ref{sup5}) of definition~\ref{defsup}). 
Likewise, if $\sup_{\bar{K}}(f_0)<\v{r}$, then by induction $\od{f_0}{\bar{d}}$
is a sequence of terms of the form $k_a(\v{b})$ for $b<r-1$ and $\gamma_{a,b}$
(the case $r=1$ cannot occur here as the supremum of any description 
$d'$ with $k(d')>t$ is at least $\v{1}$). We also have $\sup_{\bar{K}}(p)=
\sup_{\bar{K}}(f_0)<\v{r}$, and this verifies the first statement.



To verify the second statement of the fact, consider $p_1$, $p_2$ as in the statement. 
Assume first that $k=k(p_1)<k(p_2)$. Say, $p_1=k_k^{(s)}(l+1)(f_1,\dots,f_l,f_0)$. 
From I.2.\ of lemma~\ref{lemreform} we have $f_0 <p_2$. If $f_0$ is of the form $\v{b}$
then $\sup_{\bar{K}}(p_1)=\v{b+1}$ and so $b=r-1$. In this case, by definition of the 
o--sequence we also have $\od{p_1}{\bar{d}}=k_k(\v{b})=k_k(\v{r-1})$. 
By induction, $\od{p_2}{\bar{d}}$ is of the form $k_c(\v{r-1})$ and 
$c \geq k(p_2)>k$ and thus from definition~\ref{oord} we have $\od{p_1}{\bar{d}}
<\od{p_2}{\bar{d}}$. If $f_0$ is not of this form, then  from definition~\ref{defsup}
$\sup_{\bar{K}}(p_1)=\sup_{\bar{K}}(f_0)=\v{r}$. By induction, 
$\od{f_0}{\bar{d}}\leq \od{p_2}{\bar{d}}$ and we are done as $\od{p_1}{\bar{d}}=\od{f_0}{\bar{d}}$
in this case. Assume next that $k(p_1)>k(p_2)=k$. Say $p_2=k_k^{(s)}(l+1)(g_1,\dots,g_l,g_0)$. 
From II.2.\ of lemma~\ref{lemreform} we have $p_1 \leq g_0$. If $g_0$ is of the form 
$\v{b}$ we must have as above that $b=r-1$. But then $p_1 \leq \v{r-1}$
and so $\sup_{\bar{K}}(p_1) \leq \v{r-1}$, a contradiction. So, assume $g_0$ is not
of this form. Then $\sup_{\bar{K}}(p_2)=\sup_{\bar{K}}(g_0)=\v{r}$. By induction, 
$\od{p_1}{\bar{d}} \leq \od{g_0}{\bar{d}}$ and we are done as $\od{p_2}{\bar{d}}=
\od{g_0}{\bar{d}}$ in this case. Finally, assume $k(p_1)=k(p_2)=k$. 
Say, $p_1=k_k^{(s)}(l+1)(f_1,\dots,f_l,f_0)$ and 
$p_2=k_k^{(s)}(l'+1)(g_0,\dots,g_{l'},g_0)$. We must have $f_0 \leq g_0$ 
from III.2.a.\ of lemma~\ref{lemreform}. If $g_0$ is of the form $\v{b}$,
then $b=r-1$ as before. In this case $\od{p_2}{\bar{d}}=k_k(\v{r-1})$. 
If $f_0=\v{r-1}$, then $\od{p_1}{\bar{d}}=k_k(\v{r-1})$ as well and we are done. 
If $f_0 <\v{r-1}$, then $p_1<\v{r-1}$ from I.2.\ of lemma~\ref{lemreform}. 
But then $\sup_{\bar{K}}(p_1) \leq \v{r-1}$ from proposition~\ref{suppropb},
a contradiction. So, assume $g_0$ is not of this form.
So, $\sup_{\bar{K}}(p_2)=\sup_{\bar{K}}(g_0)=\v{r}$. If $f_0=\v{b}$, then $b=r-1$
and $\od{p_1}{\bar{d}}=k_k(\v{r-1})$. By induction, $\od{g_0}{\bar{d}}=k_b(\v{r-1})$
for some $b$, and we must have $b \geq k(g_0)>k$. So, $\od{p_1}{\bar{d}} <
\od{p_2}{\bar{d}}$ from definition~\ref{oord}. Finally, if $f_0$ is also not of the form $\v{b}$
then $\sup_{\bar{K}}(f_0)=\sup_{\bar{K}}(p_1)=\v{r}$. By induction applied to $f_0$, $g_0$
we have $\od{f_0}{\bar{d}} \leq \od{g_0}{\bar{d}}$ and we are done since 
$\od{p_1}{\bar{d}}=\od{f_0}{\bar{d}}$ and likewise $\od{p_2}{\bar{d}}=
\od{g_0}{\bar{d}}$.
\end{proof}







\begin{cor}
Let $\bar{d}=(d;\bar{S})$ where $d \in \cD_m(\bar{S})$. Let 
$q \in \cD_m(\bar{S})$ with $q \leq d$. If $q=\v{r}$
then $\dep{\bl{q}{\bar{d}}}=\omega^{\omega^{r-1}}$ and if $q <\v{r}$ then 
$\dep{\bl{q}{\bar{d}}}<\omega^{\omega^{r-1}}$.
\end{cor}
\end{comment}








\begin{prop} \label{pd}
Fix $\bar{d}=(d;\bar{S})$ with $d \in \cD_m(\bar{S})$, and let 
$\bar{p}=(p;\bar S, S^*)$ where $p=\sL(\bar{d})$.
Suppose $\bar{q} \in  \cD_m(\bar{S},S^*, \bar{K})$. 
Then $\lv{q}{\bar{p}}\leq \lv{q}{\bar{d}}$.
Moreover, if $\od{q}{\bar{d}}$ starts with the term corresponding to 
the $S^*$ measure, then strict inequality holds.  If $\od{q}{\bar{d}}$
does not start with this term, then 
%$\sup_{\bar{d}} (q)= \sup_{\bar{p}} (q)$.
\rc{mc47?}{$\sup_{S^*,\bar{K}} (q)= \sup_{\bar{K}} (q)$}.
\end{prop}


\begin{proof}
We consider the case $S^*= S^{r_*}_1$, the case $S^*=W_1^{r_*}$ being easier.
Extending our notational convention slightly, we use terms
$h_i(j), \alpha_{i,j}$ corresponding to the $\bar{S}$ measures,
$k_*$ corresponding to $S^*$, and $k_i(j), \gamma_{i,j}$
corresponding to the $\bar{K}$ measures.

We may consider the o--sequences of $q$ defined
relative to $\bar{d}$ and $\bar{p}$.
Let us fix them: $u_d=  \od{q}{\bar{d}}$ and
$u_p = \od{q}{\bar{p}}$.
We want to analyze the relationship between these two sequences.
Recall the definition of the o--sequence.
In that definition we concatenated recursively the o--sequences of the
corresponding subdescriptions (and then took the canonical increasing subsequence). 
We can repeat the same constructions
with the only difference that we stop when the subdescription is of the form 
$k_*^{(s)}(j)(\dots)$, for some $j$. Suppose that happens $a$ times.
Then
\begin{align*}
u_d&= [{u_1} \conc {\od{k_*^{(s)}(j_1)(\dots)}{\bar d}}
        \conc \dots \conc {
%u_2
\rc{mc48}{u_a}
        } \conc
        {\od{k_*^{(s)}(j_a)(\dots)}{\bar d}}\conc  u_{a+1}]'\\
u_p&= [{u_1} \conc {\od{k_*^{(s)}(j_1)(\dots)}{\bar p}}
        \conc \dots \conc {
%u_2
\rc{mc48}{u_a}
        } \conc
        {\od{k_*^{(s)}(j_a)(\dots)}{\bar p}}\conc u_{a+1}]'
\end{align*}
In other words, the difference between $u_d$ and $u_p$ is determined
only by
the o--sequences of the subdescriptions starting with an invariant of $k_*$.
Let us call these subdescriptions $s_1,\dots,s_a$, so for $1 \leq b \leq a$
we have $s_b$ is of the form $s_b=k_*^{(s)}(j_b)(f_1,\dots,f_l,f_0)$.






We first claim  that for each $b$ we either have $\od{s_b}{\bar{d}}=
\od{s_b}{\bar{p}}$ or else $\od{s_b}{\bar{d}}=k_*(\v{r})$ and 
$\od{s_b}{\bar{p}}$ is a sequence of terms each of which is less than $\v{r}$. 
Granting this claim, it is straightforward to show that 
$\lv{q}{\bar{p}} \leq \lv{q}{\bar{d}}$ (we use here the fact that 
ordinals of the form $\omega^{\omega^r}$ are closed under multiplication).
To see the claim, consider  $s_b=k_*^{(s)}(j_b)(f_1,\dots,f_l,f_0)$.
From the definition of the o--sequence we have that $\od{s_b}{\bar{d}}$
is of the form $k_*(\v{r})$, $k_i(\v{r})$, or $\gamma_{i,j}$.
First suppose that $f_0=\v{r}$. In this case 
$\od{s_b}{\bar{d}}=k_*(\v{r})$
From the definition of the o--sequence we
also have $\od{s_b}{\bar{p}}=
[\od{f_1}{\bar{p}} \conc \dots \conc  \od{f_l}{\bar{p}}]'$.
For each $1\leq i\leq l$, $\ f_i < \v{r}$, and so from fact~\ref{ef1} 
we have that $\od{f_i}{\bar{d}}$ has only terms which are less than $\v{r}$. 
So, $\od{s_b}{\bar{p}}$
is a sequence of terms strictly less than $\v{r}$.





 







Next suppose that $f_0$ is not of the form $\v{r}$. 
Thus $\od{s_b}{\bar{d}}=\od{f_0}{\bar{d}}$ which is a single term of the form 
$k_i(\v{r})$ or $\gamma_{i,j}$. Suppose 
$\od{f_0}{\bar{d}}=k_i(\v{r})$ for some $i$. 
Then $\od{s_b}{\bar{p}}=[k_i(\v{r})\conc  \od{f_1}{\bar{p}}
\dots \conc  \od{f_l}{\bar{p}}]'$. 
From the definition of description we have that 
for all $1\leq e\leq l$ that $f_e<f_0$. From fact~\ref{ef3}
it 
\rc{mc}{follows} 
that that either $\od{f_e}{\bar{p}}=\emptyset$
or else is a single term which is less than or equal to the 
term $k_i(\v{r})$ in the ordering of definition~\ref{oord}. 
Thus, all of the 
terms from the $\od{f_e}{\bar{p}}$ will be canceled when we compute 
$\od{s_b}{\bar{p}}$. Hence, $\od{s_b}{\bar{p}}=k_i(\v{r})=\od{s_b}{\bar{d}}$. 
The case where $\od{f_0}{\bar{d}}=\gamma_{i,j}$ is argued exactly the same way.
This completes the proof of the first statement of the proposition.



To see the second statement of the proposition, suppose $\od{q}{\bar{d}}$ starts with 
a term of the form $k_*(\v{r})$. In this case $u_1=\emptyset$ and $s_1$
is of the form $s_1=k_*^{(s)}(j_1)(\cdots,\v{r})$. We argued above that in this case we 
have $\lv{s_1}{\bar{p}} <\lv{s_1}{\bar{d}}$. Since $u_1=\emptyset$, it
now follows that $\lv{q}{\bar{p}}<\lv{q}{\bar{d}}$. This proves the second 
statement of the proposition. 






Finally, to see the third statement of the proposition 
suppose $\od{\bar q}{\bar d}$ begins with a term  of the form 
$k_i(\v{r})$ or $\gamma_{i,j}$. To prove the third statement 
it suffices to show the following claim: if $q \in \cD_m(\bar{S}, S^*,\bar{K})$
and $\od{q}{\bar{d}}$ does not begin with a $k_*$ term, then 
$\sup_{\bar{K}}(q)=\sup_{S^*,\bar{K}}(q)$.
We prove this claim by reverse induction on $k(q)$. 
If $k(q)>t+1$ (where $\bar{S}=S_1,\dots,S_t$), then 
the result follows from lemma~\ref{suppropa}. 
If $k(q)=t+1$, then $q$ is of the form $q=k_*^{(s)}(l+1)(f_1,\dots,f_l,f_0)$. 
Also, $f_0\neq \v{r}$ for any $r$ since in that case $\od{q}{\bar{d}}=k_*(\v{r})$. 
It follows that $\od{f_0}{\bar{d}} \neq \emptyset$. By lemma~\ref{suppropa} we have 
$\sup_{\bar{K}} (f_0)= \sup_{S^*, \bar{K}} (f_0)$. From the definition of supremum we then have 
$\sup_{\bar{K}} (q)= \sup_{\bar{K}} (f_0)=\sup_{S^*,\bar{K}} (f_0)= 
\sup_{S^*,\bar{K}} (q)$ (for the first equality we use  (\ref{sup4}) of definition~\ref{defsup},
and for the third equality we use (\ref{sup5}) of \ref{defsup}). Suppose finally that 
$k(q) \leq t$. If $q=\alpha_{i,j}$ the result is trivial, so assume 
$q=h_i^{(s)}(l+1)(f_1,\dots,f_l,f_0)$. Let $0 \leq j \leq l$ be least so that 
$\od{f_j}{\bar{d}} \neq \emptyset$. Recall that by proposition~\ref{propeasy}
this is equivalent to saying $f_j \notin \cD_m(\bar{S})$. By definition of the o--sequence,
$\od{q}{\bar{d}}$ starts with $\od{f_j}{\bar{d}}$, so $\od{f_j}{\bar{d}}$
does not start with a $k_*$ term. By induction, 
$\sup_{\bar{K}}(f_j)=\sup_{S^*,\bar{K}}(f_j)$. We cannot have $f_j \in \cD_m(\bar{S},S^*)$
as then $\od{f_j}{\bar{d}}$ begins with a $k_*$ term. This is because 
$f_j \notin \cD_m(\bar{S})$, and so $\od{f_j}{\bar{d}} \neq \emptyset$
from proposition~\ref{propeasy}. But since $S^*$ is the only 
measure after the $\bar{S}$ sequence, $\od{f_j}{\bar{d}}$ can 
only contain terms of the form $k_*(\v{r})$. 
Therefore, by the comments
just before proposition~\ref{suppropb}, $\sup_{\bar{K}} (f_j)>f_j$ and also 
$\sup_{S^*,\bar{K}}(f_j)>f_j$. From (\ref{sup2}) of proposition~\ref{supprop}
we have that for all $j'<j$ that $\sup_{\bar{K}}(f_{j'})=
\sup_{S^*,\bar{K}}(f_{j'})=f_{j'}$ and so from (\ref{sup4}) of 
%definition~\ref{supprop}
\rc{mc49}{definition~\ref{defsup}}
 it follows that $\sup_{\bar{K}}(q)=\sup_{S^*,\bar{K}}(q)$. 
\end{proof}





\begin{lem}\label{comprank}
 Let  $\bar d=(d,\bar S)$, and  $\bar p$ be a node in $T_{\bar{d}}$ below
 $\bar d$. Then  $\xi_{\bar{p}} < \xi_{\bar{d}}$.
\end{lem}



\begin{proof} By induction on the rank of $\bar d$, we may assume that
$\bar{p}$ has description $p=\sL(\bar{d})$. So, $\bar{p}
=(\sL(\bar{d}); \bar{S},S^*)$ for some measure $S^*$. 
In keeping with the previous conventions, we denote terms
corresponding to the measure $S^*$ by $k_*$ (we use this notation even though
we could have $S^*=W_1^r$ in which case the term would be of the form $\gamma^*_{i,j}$). 


Let $\bl{q_1}{\bar{d}},\dots,\bl{q_n}{\bar{d}}$ be all the blocks of %$T_{\bar{d}}$
\rc{mc50?}{$\bar{d}$}
where $q_1=d> q_2=p > q_3 > \dots > q_n$ and ${q_i} \in \cD_m(\bar{S})$.
Each $q_i$ for $i \geq 2$ is also a description in 
$\cD_m(\bar{S},S^*)$ which is less than or equal to $p$. However, between
$q_i$ and $q_{i+1}$ there may be several descriptions in $\cD_m(\bar{S},S^*)$. 
Thus, each $\bar{d}$--block $\bl{q_i}{\bar{d}}$, $i \geq 2$, may split into several 
$\bar{p}$--blocks. Let $s^i_1>s^i_2>\cdots>s^i_e$ enumerate all the 
$s \in \cD_m(\bar{S},S^*)$ with $q_{i+1}<s \leq q_i$ (of course $e=e_i$ depends on $i$). 
So, $s^i_1=q_i> s^i_2 > \cdots >s^i_e>q_{i+1}$. Thus, the $\bar{d}$--block
$\bl{q_i}{\bar{d}}$ splits into $e_i$ many $\bar{p}$--blocks, namely the 
$\bl{s^i_j}{\bar{p}}$ for $1 \leq j \leq e_i$. 
The idea of the proof is to show that the sum of the $\dep{\bl{s^i_j}{\bar{p}}}$
for all $1 \leq j \leq e_i$ is no greater than $\dep{\bl{q_i}{\bar{d}}}$.
We state this precisely in the following claim. 






\begin{claim}
With notation as above:
\begin{enumerate}
\item \label{rc1}
For any $i \geq 2$, $\sum_{j=e_i}^1 \dep{\bl{s^i_j}{\bar{p}}} \leq \dep{\bl{q_i}{\bar{d}}}$.
\item \label{rc2}
If $i=2$ (that is, $s^i_1=q_i=p)$, then
$\sum_{j=e_i}^2 \dep{\bl{s^i_j}{\bar{p}}} < \dep{\bl{q_i}{\bar{d}}}$.
\end{enumerate}
\end{claim}






\begin{proof}
Consider $q_i$ for $i \geq 2$ and the 
\rc{mc}{corresponding} 
block $\bl{q_i}{\bar{d}}$. 
Again let $s^i_1=q_i>s^i_2>\cdots>s^i_e>q_{i+1}$ as above so $\bl{q_i}{\bar{d}}$
splits into the $\bar{p}$--blocks $\bl{s^i_j}{\bar{p}}$ for $1 \leq j \leq e=e_i$. 


First suppose that $e=1$. In this case, $\bl{q_i}{\bar{d}}=\bl{q_i}{\bar{p}}$. 
For any $q \in \cD_m(\bar{S},S^*,\bar{K})$, for any sequence of measures $\bar{K}$,
proposition~\ref{pd} gives that $\lv{q}{\bar{p}}\leq \lv{q}{\bar{d}}$. 
Then (\ref{rc1}) of the claim follows immediately since the left-hand
side of the inequality is equal to $\lv{q}{\bar{p}}$ and the right-hand side
to $\lv{q}{\bar{d}}$ for some $q \in \bl{q_i}{\bar{d}}$. 
Still assuming $e=1$, suppose now that $i=2$. In this case the left-hand side
of the inequality in (\ref{rc2}) is empty, 
%that is gives 
\rc{mc51}{that gives} 
value $0$, while the 
right-hand side is at least $1$ by definition of $\dep{\bl{q_i}{\bar{d}}}$. 

Suppose next that $e>1$, so  $\bl{q_i}{\bar{d}}$ splits into $e$ many 
blocks $\bl{s^i_j}{\bar{p}}$ for $1 \leq j \leq e$. Let $2 \leq j' \leq e$
be such that $\dep{\bl{s^i_{j'}}{\bar{p}}}$ is maximal among 
 $\dep{\bl{s^i_{2}}{\bar{p}}}, \dots$,  $\dep{\bl{s^i_{e}}{\bar{p}}}$. 
Let $q' \in \bl{s^i_{j'}}{\bar{p}}$ be such that $\lv{q'}{\bar{p}}
=\dep{\bl{s^i_{j'}}{\bar{p}}}$. Say $q' \in \cD_m(\bar{S},S^*,\bar{K})$. 
From the last statement of proposition~\ref{pd} we must have that 
$\od{q'}{\bar{p}}$ begins with a $k_*$ term as otherwise 
$\sup_{S^*,\bar{K}} (q')=\sup_{\bar{K}} (q')$ which is impossible as 
$\sup_{\bar{K}} (q')= s^i_{j'}$ while $\sup_{S^*,\bar{K}} (q')=q_i$
and $s^i_{j'} \neq q_i$ as $j' \geq 2$. From the second statement 
of proposition~\ref{pd} we have $\lv{q'}{\bar{p}} < \lv{q'}{\bar{d}}$. 
Also, since $\od{q'}{\bar{d}}\neq \emptyset$ we must that 
$\lv{q'}{\bar{d}}>1$ and is an ordinal of the form $\omega^\alpha$
for some $\alpha \geq 1$. Since ordinals of this form are 
closed under addition, we have 
$
%\sum_{j=e_i}^2 
\sum_{j=\rc{mc52}{e}}^2 
\dep{\bl{s^i_{j}}{\bar{p}}} 
\leq \lv{q'}{\bar{p}} \cdot (e-1)< \lv{q'}{\bar{d}}\leq \dep{\bl{q_i}{\bar{d}}}$.
This gives (\ref{rc2}) of the claim.
If $\dep{\bl{s^i_1}{\bar{p}}} \leq \lv{q'}{\bar{p}}$, then 
the left-hand side of the inequality of (\ref{rc1}) of the claim is at most 
$ \lv{q'}{\bar{p}} \cdot e$ which is 
\rc{mc}{still} 
less than the right-hand side. 
If  $\dep{\bl{s^i_1}{\bar{p}}} > \lv{q'}{\bar{p}}$, then the left-hand
side of (\ref{rc1}) is equal to  $\dep{\bl{s^i_1}{\bar{p}}}$ as this ordinal
is closed under addition. But,  $\dep{\bl{s^i_1}{\bar{p}}}
\leq \dep{\bl{q_i}{\bar{d}}}$ from proposition~\ref{pd}. This verifies
(\ref{rc1}) of the claim. 
\end{proof}









Lemma~\ref{comprank} is an immediate consequence of the last claim:
\begin{equation*}
\begin{split}
\xi_{\bar p}
&= \sum_{i=n}^3 \left(  \sum_{j=e_i}^1 \dep{ \bl{s^i_{j}}{\bar{p}} } \right)
+ \sum_{j=e_2}^2 \dep{ \bl{s^2_{j}}{\bar{p}} }
\\ & <  \sum_{i=n}^3  \dep{ \bl{q_i}{\bar{d}} } +\dep{\bl{q_2}{\bar{d}}}=\xi_{\bar{d}}.
\end{split}
\end{equation*}
\end{proof}




\begin{cor} \label{lb}
Let $d \in \cD_m(\bar{S})$, and satisfy condition~D.
Then we have $(\id;d;W^m;\bar{S}) \leq  \al{\omega+\xi_{\bar{d}}+1}$.
\end{cor}



\begin{proof}
Lemma~\ref{comprank} and a trivial induction show that
the rank of the tree $T_{\bar{d}}$ (in the usual sense of rank)
is at most $\xi_{\bar{d}}$. However, $|T_{\bar{d}}|$ is at most 
one more than the usual rank (it is exactly one more if the rank
is infinite). So, 
$|T_{\bar{d}}|\leq\xi_{\bar{d}}+1$.
By the results of \cite{J1} (see remark~\ref{j1rem}), $(\id;d;W^m;\bar S) \leq
\al{\omega+|T_{\bar{d}}|}$.
So $(\id;d;W^m;\bar S)\leq \al{\omega+\xi_{\bar{d}}+1}$.
Note here that if $d$ is the minimal description in $\cD_m(\bar{S})$
then $\xi_{\bar{d}}=0$ (as the sum defining $\xi_{\bar{d}}$ is empty)
and the upper bound becomes $\al{\omega+1}=\bd^1_3$.
\end{proof}



We now head towards our main result which is that
the lower bound for $(\id;d;W^m;\bar{S})$
is also $\al{\omega+\xi_{\bar{d}}+1}$.


We recall the following fact.
\begin{thm}[Martin]\label{mt2}
Assume $\kappa\rightarrow\kappa^\kappa$. Then for any measure $\nu$
on $\kappa$, the ultrapower $j_\kappa(\kappa)$ is a cardinal.
\end{thm}
\begin{proof} See \cite{J1}.
\end{proof}


Our 
\rc{mc}{strategy} 
for the rest of the proof is to embed the ultrapower
of $\bd^1_3$ by a certain measure corresponding to $\xi_{\bar d}$ (made
precise below) into $(\id;\bar{d};W^m; \bar{S})$. Using theorem~\ref{mt2},
this will give the lower bound. We require first some embedding lemmas.


\begin{defn}[Strong embedding] \label{stremb}
Let $(D_i,<_{D_i})$, $(E_i,<_{E_i})$, $1 \leq i \leq n$ be well-orderings
of length $< \bd^1_3$, and  $M_i$, $N_i$ measures on $D_i$, $E_i$.
Let $D=D_1 \oplus \dots \oplus D_l$, $E=E_1 \oplus \dots \oplus E_l$,
the sum of the order types.
We say $(D, \{ M_i \} )$ {\em strongly embeds} into $(E, \{ N_i \} )$
if there is a measure
$\mu$ on $\kappa < \bd^1_3$, and a function $H$ with the
following properties:
\begin{enumerate}
\item \label{stremb1}
%
\rc{Major2.6!}{$\forall^\ast_\mu \theta \ H(\theta)=([\phi_1]_{M_1}, \dots,
[\phi_l]_{M_l})$, where $\phi_i:D_i \to E_i$ is order-preserving.}
%
\item  \label{stremb2}
For all $A_i \subseteq E_i$, $1 \leq i \leq n$,  of $N_i$
measure 1, $\forall^\ast_\mu \theta$
$\forall i \ \forall^\ast_{M_i} \alpha \in D_i$ $\phi_i(\alpha) \in A_i$.
\end{enumerate}
\end{defn}



If $(D_i,M_i)$ strongly embeds into $(E_i,N_i)$ for all
$1 \leq i \leq n$, then $D=\oplus D_i$ strongly embeds into
$E=\oplus E_i$. Namely, if the measures $\mu_i$ and functions $H_i$
%
%wirness 
\rc{mc}{witness}
% 
the strong embeddability of $(D_i,M_i)$ into $(E_i,N_i)$,
then the product measure $\mu=\mu_1\times \cdots \times \mu_n$ and 
the function $H(\theta_1,\dots,\theta_n)=(H_1(\theta_1),\dots, H_n(\theta_n))$
witness the strong embeddability of $D$ into $E$. 


In definition~\ref{ordmeas} below we implicitly use the fact that 
if $\mu$ is a measure on $\alpha<\bd^1_3$ 
%amd
\rc{mc}{and} 
$\beta<\bd^1_3$, then $j_\mu(\beta)
<\bd^1_3$. Recall our comments about this fact in \S\ref{preliminaries}.



\begin{defn} \label{ordmeas}
Given the ordering $D=D_1 \oplus \dots \oplus D_n$ 
%%
\rc{mc53?}{}
and measures $M_i$ on $D_i$, let $\nu_D$ denote the
measure on $n$-tuples from $\bd^1_3$ induced by the weak partition relation on
$\bd^1_3$, functions $f\colon D \to \bd^1_3$ of the correct type,
and the $M_i$. That is, 
%$A \subseteq \bd^1_3$ 
\rc{mc53}{$A \subseteq (\bd^1_3)^n$} 
has $\nu_D$ measure one iff 
there is a c.u.b.\ $C \subseteq \bd^1_3$ such that for all 
$f=(f_1 \oplus \cdots \oplus f_n)  \colon D \to C$ of the correct type, \
$([f_1]_{M_1},\dots, [f_n]_{M_n}) \in A$. 
\end{defn}








\begin{prop} \label{embprop}
If $(D,\{ M_i \})$, $1 \leq i \leq n$,  strongly embeds into
$(E,\{ N_i \})$, then
$j_{\nu_D}(\bd^1_3) \leq j_{\nu_E}(\bd^1_3)$.
\end{prop}



\begin{proof}
Let $\mu$, $H$ witness the strong embeddability.
We define an embedding $\pi$ from $j_{\nu_D}(\bd^1_3)$ to
$j_{\nu_E}(\bd^1_3)$. Define $\pi([F]_{\nu_D})=[G]_{\nu_E}$,
where for $g=(g_1 \oplus \dots \oplus g_n)\colon
E \to \bd^1_3$ of the correct type, 
$$G([g_1]_{E_1}, \dots, [g_n]_{E_n})
=[\theta \to F([g_1\circ \phi_1]_{M_1}, \dots,
[g_n \circ \phi_n]_{M_n}) ]_{\mu},$$ 
where $H(\theta)=([\phi_1]_{M_1}, \dots, [\phi_n]_{M_n})$.
Using the
properties of $H$, this is easily well-defined and an
embedding.
\end{proof}




\begin{prop} \label{basicota}
Let $\cO$ be an order-type of length $< \bd^1_3$, and let
$M$ be a measure on $\cO$.  
Let $D=\cO\oplus \cdots \oplus \cO$
be the $n$-fold sum of $\cO$, and let $M_i=M$ for $1 \leq i \leq n$. 
Let $E$ be the order-type of 
$\omega_1 \times \cO$ ordered 
%lexicgraphically, 
\rc{mc}{lexicographically}, 
and let $N=W_1^1 \times M$.
Then $(D,\{ M_i\})$ strongly embeds into $(E,N)$. 
\end{prop}



\begin{proof}
Let $\mu=W_1^n$. For $\bar{\alpha}=(\alpha_1,\dots,\alpha_n) \in (\omega_1)^n$, let 
%$H(\bar{\alpha})\colon D \to E$
\rc{mc54!?}{$H(\bar{\alpha})\colon D \to E$} 
be defined by 
$H(\bar{\alpha})(i, \beta)= (\alpha_i,\beta)$ where here we identify $D$
with pairs $(i,\beta) \in  \{ 1,\dots,n\} \times \cO$ ordered lexicographically. 
It is straightforward to check (\ref{stremb1}) and (\ref{stremb2}) of 
definition~\ref{stremb}.
\end{proof}










\begin{prop} \label{basicot}
Let $\cO$ be an order type of length $< \bd^1_3$,
and $\nu$ a measure on $\cO$. Let
$0 \leq k <l$, $m>0$. Let $D$ be lexicographic order on
$(\omega_{k+1})^m \times \cO$. Let $M$ be the product measure 
$M= (S_1^k)^m \times \nu$ if $k>0$ and let $M=(W_1^1)^m \times \nu$ if $k=0$.
Let $E$ be lexicographic ordering on 
$\omega_{l+1} \times \cO$ and let $N$ be the measure $S_1^l \times \nu$.
Then $(D,M)$ strongly embeds into $(E,N)$.
\end{prop}



\begin{proof}
We prove the result for $k >0$, the other case being similar.
Let $\mu=S^{l+m}_1$. Define $H([h]_{W^{l+m}_1})=[\phi]_{M}$,
where $\phi:D \to E$ is defined as follows. $\phi([f_1]_{W^k_1},
\dots, [f_m]_{W^k_1}, \gamma)=([g]_{W^l_1},\gamma)$, where
$$g(\delta_1, \dots, \delta_l)=h(\delta_1, \dots, \delta_k,
f_1(\delta_1, \dots, \delta_k), \dots, f_m(\delta_1, \dots, \delta_k),
\delta_{k+1}, \dots, \delta_l).$$ This is easily well-defined,
and gives a strong embedding.
\end{proof}





By a {\em sub-basic} order-type  we mean either the ordinal $1$
(i.e., the order-type of a single point $0$) or lexicographic ordering on 
$\omega_{k_1+1} \times \cdots \times \omega_{k_m+1}$ for some 
$k_1,\dots,k_m  \in \omega$. To each sub-basic order type $D$ we associate 
a measure $M$. If $D=1$ then $M$ is the principal measure on $0$, and otherwise let 
$M$ be the product measure 
$S_1^{k_1} \times \cdots \times S_1^{k_m}$, where we use here $W_1^1$
in place of $S_1^{k_i}$ whenever $k_i=0$. By a {\em basic} order-type $D$
we mean an order-type of the form $D=D_1 \oplus \cdots \oplus D_l$ where
each $D_i$ is a sub-basic order-type. We associate to $D$ the  measures 
$\{ M_i\}$ where $M_i$ is the measure associated to
the sub-basic order-type $D_i$. 








To each sub-basic order-type  $D$, we associate an ordinal $c(D)$ as follows. If $D=1$,
then $c(D)=1$. If $D= \omega_{k_1+1}\times \cdots \times \omega_{k_m +1}$, then $c(D)=
\omega^{\omega^{k_m}} \cdots \omega^{\omega^{k_2}} \cdot
\omega^{\omega^{k_1}}$. We then extend the $c$ function to basic order-types
$D=D_1 \oplus \cdots \oplus D_l$ by defining 
$c(D)=c(D_1)+ \dots + c(D_l)$.




\begin{prop}  \label{basicotc}
Let $D$ be a basic order-type and $E=D\oplus 1$. Then 
$j_{\nu_D}(\bd^1_3)< j_{\nu_E}(\bd^1_3)$.
\end{prop}


\begin{proof}
Define $\pi$ from $j_{\nu_D}(\bd^1_3)$ to  $j_{\nu_E}(\bd^1_3)$ by
$\pi([F]_{j_{\nu_{D}}}) =[G]_{j_{\nu_E}}$ where 
for $f\colon D \to \bd^1_3$ of the correct type and $\alpha <\bd^1_3$
with 
%$\alpha>\sup(f)$ 
\rc{mc55, mc56!}{$\alpha > [f]$} 
we define 
$G([f],\alpha)=F([f])$. This is easily an embedding and maps 
$j_{\nu_D}(\bd^1_3)$ to a proper initial segment of 
$ j_{\nu_E}(\bd^1_3)$, namely to those $[G]$ satisfying 
$G([f],\alpha) <\alpha$ for $\nu_E$ almost all pairs $([f],\alpha)$.
\end{proof}














\begin{lem} \label{basic}
For $D$ a basic order-type with corresponding measure
$\nu_D$ as in definition~\ref{ordmeas}, $j_{\nu_D}(\bd^1_3) \geq \al{\omega+c(D)+1}$.
\end{lem}




\begin{proof}
An easy induction on the length of $D$, $|D|$,  using
propositions~\ref{basicota}, \ref{basicot} and \ref{basicotc}.
For example, the inductive step at
$D=\omega_3$ would be: $j_{\nu_{\omega_3}}(\bd^1_3) \geq$
$\sup_n j_{\nu_{(\omega_2)^n}}(\bd^1_3) \geq$
$\sup_n \al{\omega + \omega^{\omega \cdot n} +1}$
$=\al{\omega^{\omega^2}}$. The first inequality comes from 
proposition~\ref{basicot} and the 
\rc{mc}{second} 
inequality 
is from induction.
Since $\cf{j_\nu(\bd^1_3)}>\omega$
for any measure $\nu$, we then have $j_{\nu_{\omega_3}}(\bd^1_3) \geq
\al{\omega^{\omega^2} +1}= \al{\omega+ \omega^{\omega^2} +1}=\al{\omega+c(D)+1}$.
\end{proof}






Suppose now $M=M^0_{1}\times
\dots \times M^0_{a_0} \times \dots \times
M^n_1 \times \dots \times M^n_{a_n}$ is a product measure, where
$M^i_j= W^1_1$ if $i=0$, and $M^i_j=S^i_1$ for $i>0$.
Let $M_1,\dots,M_k$ enumerate the $M^i_j$ in the order they appear in 
the product $M$ (so $k=a_0+\cdots+a_n$). 
Let $\pi=(p_1, \dots, p_k)$ be a permutation of $k$. Let $D$ be the $M$
measure one set of $(\alpha_1, \dots, \alpha_k)=(\alpha^0_1,
\dots, \alpha^0_{a_0}, \dots, \alpha^n_1, \dots, \alpha^n_{a_n})$
such that $\alpha^0_1 < \dots < \alpha^0_{a_0}$,
$\alpha^i_j > \omega_i$, and $\alpha_i(1) < \alpha_j(1)$ for
$i<j$ and $\alpha_i > \omega_1$ (recall definition~\ref{definvariants}
and the comments immediately after).
Let $<_D$ be the ordering
of $D$ defined by: $(\alpha_1, \dots, \alpha_k)<_D
(\beta_1, \dots, \beta_k)$ iff $(\alpha_{p_1}, \dots, \alpha_{p_k})
<_{\lex} (\beta_{p_1}, \dots, \beta_{p_k})$.

We define the {\em canonical subsequence} $\pi^*$ to be the canonical increasing subsequence 
of $\pi$. That is, $\pi^* = (q_1, \dots, q_l)=(p_{s_1}, \dots,
p_{s_l})$, where $s_1=1$, and $s_{i+1} > s_i$ is least
such that $p_{s_{i+1}} > p_{s_i}$. Note that $q_l=k$.
To fix notation, let $M_i=M^{r(i)}_{u(i)}$ for $1 \leq i \leq k$.
Define $N$ to be the product measure $N=M_{q_1} \times \dots
\times M_{q_l}$, and let $E$ be lexicographic ordering
on tuples $(\beta_1, \dots, \beta_l)$ with $\beta_i < \omega_{r(q_i)+1}$.

Notice that $(D,M)$ and $(E,N)$ are sub-basic order-types. 

\begin{lem} \label{techlem}
With $(D,M)$, $(E,N)$ as above, $(E,N)$ strongly embeds
into $(D,M)$.
\end{lem}

\begin{proof}
Let $\mu=M_1 \times \dots \times M_{q_1-1} \times
\prod_{j=q_1}^k M_j^+$, where $(W^1_1)^+=S^1_1$, and
$(S^r_1)^+=S^{r+1}_1$.
Fix $\bar{\theta}=(\theta_1, \dots, \theta_k) \in \dom(\mu)$. 
%
\rc{MajorC2.7!}{Let $h_i \colon \dom(<_{r(i)+1})
\to \omega_1$ represent $\theta_i$ if $r(i)>0$.}
%
Set $H(\bar{\theta})=[\phi]_{N}$, where $\phi(\alpha_1, \dots, \alpha_l)=
(\beta_1, \dots, \beta_k)$ is defined as follows. First,
$\beta_1, \dots, \beta_{q_1-1}=\theta_1, \dots, \theta_{q_1-1}$.
Next, suppose  $q_i \leq j < q_{i+1}$.
If $r(j)=0$, set $\beta_j= h_j(
%\alpha_{q_i}
\rc{mc57}{\alpha_i}
)$.
%
If $r(j)>0$ and $r(q_i)=0$, set $\beta_j =[g_j]$, where
$g_j(\gamma_1, \dots, \gamma_{r(j)})= h_j(
%\alpha_{q_i,}
\rc{mc57}{\alpha_i},
%
\gamma_1,
\dots, \gamma_{r(j)})$. If $r(q_i)>0$, set $\beta_j=[g_j]$, where
$$g_j(\gamma_1, \dots,\gamma_{r(j)})=h_j(\gamma_1, \dots,
%%
\rc{mc58!}{}
\gamma_{r(q_i)}, f_i(\gamma_1, \dots, \gamma_{r(q_i)}), \gamma_{r(q_i)+1},
\dots, f_i(1)(\gamma_{r(j)})),$$ where $[f_i]=
%\alpha_{q_i}
\rc{mc57}{\alpha_i}
$, and
the argument $\gamma_{r(q_i)}$ of $h_j$ is omitted if $r(q_i)=r(j)$ (this
is just to give the correct number of arguments).
This is easily checked to be well-defined and a strong embedding.
\end{proof}

\begin{rem} The proof of lemma~\ref{techlem} also shows
if $\pi'$ is any subsequence of the canonical sequence $\pi^*$
of $\pi$, and $E'$, $N'$ the corresponding order and product measure,
then $(E',N')$ strongly embeds into $(D,M)$.
\end{rem}


\begin{prop}  \label{cofprop}
Let $\bar{d}=(d;\bar{S})$ and consider a block 
%%
\rc{mc59!mc60!}{\enlargethispage{0.5\baselineskip}}
$\bl{q_i}{\bar{d}}$ which is non-trivial, that is, with $\dep{\bl{q_i}{\bar{d}}}>1$.
Then there is a $\bar{p}_i=(p_i;\bar{S},\bar{K}) \in 
\bl{q_i}{\bar{d}}$  with $\lv{p_i}{\bar{d}}=
\dep{\bl{q_i}{\bar{d}}}$ where $p_i$ is of the form $p_i=\gamma_{a,b}$,
$p_i=k_a(\v{r})$. or 
$p_i=h_k^{(s)}(l+1)(f_1,\dots,f_l,f_0)$. In the last case, 
there is a $\bar{p}'_i \in \bl{q_i}{\bar{d}}$
with $\lv{p'_i}{\bar{d}}=
\dep{\bl{q_i}{\bar{d}}}$ and such that $p'_i$ is of the form 
$$p'_i=h_k^{(s)}(r)(g_1,\dots,g_{r-1},g_0)$$ where $S_k=S_1^{r}$
(that is, $p'_i$ has maximal possible length).
\end{prop}
 



\begin{proof}
Say $d \in \cD_m(\bar{S})$ and recall that $q_i \in \cD_m(\bar{S})$ as well. 
Note first that $\cof(q_i)>\omega$. For if $\cof(q_i)=\omega$,  
\rc{mc61}{then}
%
the block 
%
\rc{MajorC2.8!}
{$\bl{q_i}{\bar{d}}$ would be empty and so 
$\dep{\bl{q_i}{\bar{d}}}=1$. For suppose $\bl{q_i}{\bar{d}}\neq \emptyset$
and let $\bar{p}=(p;\bar{S},\bar{K})$} 
%
for some measure sequence $\bar{K}$
be such that $\sup_{\bar{K}}(p)=q_i
%.
\rc{mc62}{\ne p}.$ 
Note that $\sL(q_i;\bar{S},\bar{K})$
is defined and $p \leq \sL(q_i;\bar{S},\bar{K})$ by the maximality of 
$\sL(q_i;\bar{S},\bar{K})$ among those descriptions in 
$\cD_m(\bar{S},\bar{K})$ which are less than $q_i$.
By proposition~\ref{prop1b} we have 
that $\sL(q_i; \bar{S},\bar{K}) \in \cD_m(\bar{S})$. Thus, 
$\sL(q_i; \bar{S},\bar{K})=q_{i+1}=\sL(q_i;\bar{S})$. But then by  
proposition~\ref{suppropb} we have $\sup_{\bar{K}} (p) \leq 
q_{i+1}$, a contradiction. 




% % % % % % % % % % % % % % % % % % % % % % % % % % % % % % % % %
%\rc{mc}{\enlargethispage{0.5\baselineskip}}

Suppose first that $q_i$ has functional representation
$q_i=h_k(l+1)(f_1, \dots,f_l,f_0)$, and $S_k=S^{r}_1$.
We must have
$l<r-1$, as otherwise $\cof(q_i)=\omega$ from (\ref{lemoncof3})
of lemma~\ref{lemoncof}. We can also see this directly as follows. 
Since $\bl{q_i}{\bar{d}}\neq \emptyset$, there is a $\bar{p}=
(p;\bar{S},\bar{K})$ with $\sup_{\bar{K}}(p)=q_i$. 
However, inspecting definition~\ref{defsup}
we see that this is only possible if $l<r-1$
(in particular, note from (\ref{sup4})
of \ref{defsup} that if $\sup_{\bar{K}} (p)$ doesn't have the symbol 
$s$ then $l<r-1$). Note next from (\ref{sup4}) of 
definition~\ref{defsup} 
%
%that any $\bar{p}=(p;\bar{S},\bar{K}) \in \bl{q_i}{\bar{d}}$ must be of the form 
\rc{Quesn1.2!}{that any $\bar{p}=(p;\bar{S},\bar{K}) \in \bl{q_i}{\bar{d}}$ must be of the form} 
%
$$p = h_k^{(s)}(l'+1)(f_1,\dots,f_l,f_{l+1},\dots,f_{l'},f_0)$$
for some $l'>l$ and where $\sup_{\bar{K}}(f_{l+1})=f_0$.
From (\ref{lemoncof3}) of lemma~\ref{lemoncof} we have 
$\cof(q_i)=\cof(f_0)$, so $\cof(f_0)>\omega$. 
Let now $\bar{p}=(p; \bar{S},\bar{K}) \in \bl{q_i}{\bar d}$ have
maximum possible level. In particular, $p$ is of the form
displayed 
%
%immediarely 
\rc{mc}{immediately} 
%
above. 
Since replacing $h_k^s(l'+1)$ 
by $h_k(l'+1)$ in the functional representation for $p$ 
does not change either the level or the block, we may
assume $p=h_k(l'+1)(f_1,\dots,f_l,f_{l+1}\dots,f_{l'},f_0)$.
Since $\cof(f_0)>\omega$, from proposition~\ref{prop1} it follows that for some measure 
$M$ (either of the form $W_1^b$ or $S_1^b$) that 
$\sL(f_0;\bar{S},\bar{K},M)$ involves the term corresponding to the measure
$M$ in its functional representation (that is, 
$\sL(f_0;\bar{S},\bar{K},M) \notin \cD_m(\bar{S},\bar{K})$).
Consider now the pre-description 
$$p'=h_k(r)(f_1,\dots,f_l,f_{l+1}\dots,f_{l'},g_{l'+1},\dots,g_{r-1},f_0)$$
where for $l'+1 \leq j \leq r-1$ we let 
$g_j=\sL^{k+1}(f_0; \bar{S},\bar{K},M_1,\dots, M_{j-l'})$ where each $M_a=M$. 
Clearly $g_j<f_0$ for each $j$. Also, $f_{l'}\neq g_{l'+1}$ as 
$f_{l'} \in \cD_m(\bar{S},\bar{K})$ and $g_{l'+1}
\notin  \cD_m(\bar{S},\bar{K})$. Since $f_{l'}$ is also in 
$\cD_m(\bar{S},\bar{K},M_1)$ (as is $g_{l'+1}$) and $k(f_{l'}) >k$,
it follows from the 
%
%maxinmality 
\rc{mc}{maximality} 
%
of $\sL^{k+1}(f_0;\bar{S},\bar{K},M_1)=g_{l'+1}$
among the descriptions $d'$ in $\cD_m(\bar{S},\bar{K},M_1)$ which are less than $f_0$
and with $k(d')\geq k+1$ 
that $f_{l'} \leq g_{l'+1}$. Thus we actually have 
$f_{l'} <g_{l'+1}$. A similar argument shows that 
$g_{l'+1}< \cdots < g_{r-1}$. Thus, $p'$ is actually
a description. It is clear that $\lv{p'}{\bar{d}} \geq \lv{p}{\bar{d}}$
and $p'$ has maximal length as required. 







Suppose now $q_i=h_k^s(l+1)(f_1, \dots,f_l,f_0)$. Inspecting 
definition~\ref{defsup} we see that any $\bar{p} \in \bl{q_i}{\bar{d}}$
must be of the form $\bar{p}=(p;\bar{S},\bar{K})$ with 
$$p=h_k^{(s)}(l'+1)(f_1,\dots,f_{l-1}, g_l,\dots,g_{l'}, f_0)$$
and where $\sup_{\bar{K}} g_l=f_l$. 
From (\ref{lemoncof4}) of lemma~\ref{lemoncof} we have 
$\cof(q_i)=\cof(f_l)$, and so $\cof(f_l)>\omega$. 
We may assume without loss of generality that $g_{l'} \notin
\cD_m(\bar{S})$ as otherwise $\od{g_{l'}}{\bar{d}}=\emptyset$
and we remove $g_{l'}$ without affecting the level. Let $g= \sup_{\bar{K}}(g_{l'})$.
We must have $\cof(g)>\omega$ as otherwise $g_{l'}\leq 
%\sL^{k+1}(g_{l'};\bar{S},\bar{K})
\rc{mc63}{\sL^{k+1}(g_{l};\bar{S},\bar{K})}=
%\sL^{k+1}(g_{l'};\bar{S}) 
\rc{mc64}{\sL^{k+1}(g_{l};\bar{S})} 
$ 
which would imply $\sup_{\bar{K}}(g_{l'}) \leq 
%\sL^{k+1}(g_{l'};\bar{S}) 
\rc{mc64}{\sL^{k+1}(g_{l};\bar{S})} 
%<g_{l'}
\rc{mc64*3}{<g_l}
$. 
From proposition~\ref{prop1}
there is a measure $M$ such that $\sL^{k+1}(g;\bar{S},\bar{K},M)$ 
involves the term corresponding to the measure $M$. We then finish as in the previous case,
considering now 
$$p'=h_k(r)(f_1,\dots,f_{l-1}, g_l,\dots,g_{l'}, g_{l'+1},\dots, g_{r-1},f_0)$$
where for $j>l'$, $g_j=\sL^{k+1}(g;\bar{S},\bar{K},M_1,\dots,M_{j-l'})$
(where again all $M_j$ are equal to $M$). 


We cannot have $q_i=\alpha_{i,j}$ as then $q_i$ would not satisfy condition~D. 
The remaining case is when $q_i=\v{r}$ for some $r$. In this case, 
$\bar{p}_i=(p_i;\bar{S},\bar{K})$ is in $\bl{q_i}{\bar{d}}$ for any sequence of measures
$\bar{K}$ such that 
%$K_a=S_1^{r_i}$ 
\rc{mc65}{$K_a=S_1^{r}$} 
and where $p_i=k_a(\v{r-1})$. 
This follows from (\ref{sup5}) of definition~\ref{defsup}. 
Also $\lv{k_a(\v{r-1})}{\bar{d}}=\omega^{\omega^{r-1}}$ from 
definition~\ref{level}. On the other hand, for any $\bar{p}'_i$
in $\bl{q_i}{\bar{d}}$ we have $p'_i< q_i=\v{r}$. 
So, from fact~\ref{ef1} we have that $\od{p'_i}{\bar{d}}$ is a sequence of terms
of the form $\gamma_{a,b}$ or $k_a(\v{c})$ where $c<r$. 











\end{proof}


\begin{comment}
We make more precise a technical device that was implicitly used in the
proof of lemma~\ref{cofprop}. Suppose $(\bar{S},\bar{K})$ and 
$(\bar{S},\bar{M},\bar{K})$ are two measure sequences. Note that we are using
the same $\bar{S}$ and $\bar{K}$ in both sequences. Suppose 
$d \in \cD_m(\bar{S},\bar{K})$ and $d' \in \cD_m(\bar{S},\bar{M},\bar{K})$. 
We say $d$ and $d'$ are $K$--isomorphic, $d'\cong_{\bar{K}} d$, if $d'$ is obtained from $d$ by 
translating all terms $k_i^{(s)}$ or $\gamma_{i,j}$ in the functional
representation of $d$ to the corresponding terms in the second measure sequence. 
The precise inductive definition follows.




\begin{defn}
Let $d \in \cD_m(\bar{S},\bar{K})$ and $d' \in \cD_m(\bar{S},\bar{M},\bar{K})$.
Say $\bar{S}=S_1,\dots,S_t$, $\bar{M}=M_1,\dots,M_u$, $\bar{K}=K_1,\dots,K_v$. 
Then $d$, $d'$ are $K$--isomorphic provided one of the following holds (to avoid confusion
here we use only the terms $\alpha_{i,j}$, $h_i^{(s)}$ in the functional notation,
that is, we do not use the notation $\gamma_{i,j}$, $k_i^{(s)}$).

\begin{enumerate}
\item
$d=\alpha_{i,j}$, $i \leq t$, and $d'=d$.
\item
$d=\alpha_{i,j}$, where $i=t+k$ for some $k>0$, and $d'=\alpha_{i+u,j}$.
\item
$d=\cdot_r$ and $d'=\cdot_r$.
\item
$d=h_i^{(s)}(l+1)(d_1,\dots,d_l,d_0)$, $i \leq t$, and 
$d'=h_i^{(s)}(l+1)(d'_1,\dots,d'_l,d'_0)$ where $d'_1 \cong_{\bar{K}} d_1, \dots$,
$d'_l \cong_{\bar{K}} d_l$. 
\item
$d=h_i^{(s)}(l+1)(d_1,\dots,d_l,d_0)$, where $i =t+k$ for some $k>0$ and 
$d'=h_{i+u}^{(s)}(l+1)(d'_1,\dots,d'_l,d'_0)$ where $d'_1 \cong_{\bar{K}} d_1, \dots$,
$d'_l \cong_{\bar{K}} d_l$. 
\end{enumerate}
\end{defn}




Using this notion we can prove a strengthening of lemma~\ref{cofprop}. 
Although we could get by without the following lemma in the proof
of the main result  to follow, this lemma provides a simplification
which will help clarify the proof.



\begin{prop}  \label{cofpropb}
For every block $\bl{q_i}{\bar{d}}$ as in proposition~\ref{cofprop}
there is a $\bar{p_i}=(p;\bar{S},\bar{K})$ as in proposition~\ref{cofprop}
with the additional property that all terms in $\ods{p_i}{\bar{d}}$
are distinct. In fact, if $k_i(\cdot_r)$ and $k_j(\cdot_s)$ both occur 
in  $\ods{p_i}{\bar{d}}$, then $i \neq j$. 
\end{prop}
\end{comment}

















We now prove our main lemma.




\begin{lem} \label{mainlem}
Fix $\bar{d}=(d;\bar{S})$ where $d \in \cD_m(\bar{S})$, and
satisfies condition D. Then $(\id;d;W^m;\bar{S}) \geq
\al{\omega+ \xi_{\bar{d}}+1}$.
\end{lem}


Let $d=q_1 > q_2 > \dots >q_n$ enumerate the $q \in
\cD_m(\bar{S})$ below $d$, so the number of $\bar{d}$--blocks is also %$n$.
\rc{mc66}{$n$, and $d$ satisfies condition $D$}.


For $2 \leq i \leq n$ such that $\dep{\bl{q_i}{\bar d}}>1$,
let $\bar{p_i}$ be as in proposition~\ref{cofprop}.
We refer to these blocks as the
{\em non-trivial} blocks. For the trivial blocks, let $\bar{p}_i=\bar{q}_i$.
For each non-trivial block $\bl{q_i}{\bar{d}}$, let $\bar{p_i}=(p_i;\bar{S},\bar{K}^i)$,
where $\bar{K}^i=(K_1^i, \dots, K_{u_i}^i)$.


For each non-trivial block $\bl{q_i}{\bar{d}}$, $2\leq i \leq n$, 
let $w_i=\od{p_i}{\bar{d}}$ and $w^*_i=\ods{p_i}{\bar{d}}$. 
Recall that $w^*_i$ is a sequence of terms of the form $\gamma^a_{i,j}$
and $k^a_i(\cdot_r)$ and that these terms are ordered by 
definition~\ref{osord}. Recall also that $w_i$ is obtained from $w^*_i$
by first removing the superscripts from the terms, and then taking the canonical
increasing subsequence (using definition~\ref{oord}). The ordinal 
$\lv{p_i}{\bar{d}}$ (which is equal to $\dep{\bl{q_i}{\bar{d}}}$) 
was then derived from $w_i$ (definition~\ref{level}).
Let $l_i=\lh(w_i)-1$ and $l^*_i=\lh(w^*_i)-1$.


For each $2 \leq i \leq n$ we define two sub-basic order-measures
$(D_i,M_i)$, $(E_i,N_i)$ as follows. First assume that 
the block $\bl{q_i}{\bar{d}}$ is non-trivial, that is, the 
sequence $\od{p_i}{\bar{d}} \neq \emptyset$.




To define $(D_i,M_i)$, consider the sequence of terms 
$(w^*_i(0),\dots,w^*_i(l^*_i))$ from $w^*_i=\ods{p_i}{\bar{d}}$. 
Let $(t^i_0,\dots,t^i_{l^*_i})$ be the same set of terms but written in increasing
order according to the ordering of definition~\ref{osord}. Let $\pi_i$
be the permutation of $\{ 0,1,\dots.l^*_i\}$ which gives this rearrangement, that is, 
$w^*_i(j)=t^i_{\pi_i(j)}$. To each term 
%$t^i_j$ 
\rc{mc67}{$t=t^i_j$} 
of this sequence we associate a 
measure $M^i_j$ as follows. If  $t=\gamma_{i,j}$, then we associate the measure 
$M^i_j=W_1^1$. If $t=k^a_b(\cdot_r)$ then we set $M^i_j=S_1^r$. 
Let $M^*_i=M^i_0 \times \cdots \times M^i_{l^*_i}$. For convenience,
we consider the domain of $M^*_i$ to be the (measure one) set of
tuples $(\beta^i_0,\dots,\beta^i_{l^*_i})$ such that $\beta^i_j \in \dom(M^i_j)$,
$\beta^i_j <\beta^i_{j+1}$, and if $M^i_j$, $M^i_{j+1}$ are both of the form
$S_1^r$ (for possibly different values of $r$), then $[\beta^i_j(1)]_{W_1^1}< 
%[\beta^i_+{j+1}]_{W_1^1}
\rc{mc68}{[\beta^i_{j+1}]_{W_1^1}}
$. Let $D_i$ be the set of these tuples 
$(\beta^i_0,\dots,\beta^i_{l^*_i})$ ordered by:
$$
(\beta^i_0,\dots,\beta^i_{l^*_i}) < (\eta^i_0,\dots,\eta^i_{l^*_i})
\leftrightarrow 
(\beta^i_{\pi(0)},\dots,\beta^i_{\pi(l^*_i)})
<_\lex 
(\eta^i_{\pi(0)},\dots,\eta^i_{\pi(l^*_i)}).
$$






We define $(E_i,N_i)$ as follows. Let $(u^i_0,\dots,u^i_{l^*_i})$
be the same sequence of terms as $(t^i_0,\dots,t^i_{l^*_i})$
except that we have removed the superscripts 
%fronm 
\rc{mc69}{from} 
the terms. Let 
$j_0,j_1,\dots, j_e$ be the canonical increasing subsequence 
of $u^i_{\pi(0)}, \dots, u^i_{\pi(l^*_i)}$ using the ordering of
definition~\ref{oord}. That is, $j_0=\pi(0)$, 
%and $j_{l+1}$ is the least integer greater than $j_l$ such that $u^i_{\pi(l+1)} > u^i_{\pi(l)}$ 
\rc{mc70?}{$j_l=\pi(k_l)$, and $j_{l+1}=\pi(k)$, where $k$ is the least integer $k'>k_l$ with $u^i_{\pi(k'} > u^i_{j_l}=u^i_{\pi(k_l)}$} 
(in the order of \ref{oord}). 
Thus, $u^i_{j_0},\dots,u^i_{j_e}$ enumerates the o--sequence 
$\od{p_i}{\bar{d}}$. Let $N_i$ be the product measure 
$M^i_{j_0} \times \cdots \times M^i_{j_e}$. Let $E_i$ be lexicographic
ordering on the tuples $(\beta^i_{j_0},\dots,\beta^i_{j_e})$ 
where again $\beta^i_{j_k} \in \dom(M^i_{j_k})$ (and we restrict to the
analogous measure one set as in the definition of $D_i$). 









For trivial blocks $\bl{q_i}{\bar{d}}$ we let 
$D_i=E_i=1$ and $M_i=N_i=$ the principal measure on $\{ \emptyset\}$. 

\rc{mc}{\newpage}
Finally, we set 
$E=E_n \oplus \cdots \oplus 
%E_1
\rc{mc71}{E_2}
$ 
and 
$D=D_n \oplus \cdots \oplus 
%D_1
\rc{mc71*2}{D_2}
$. 
So, we have defined the 
basic types $(E, \{ N_i\})$ and $(D,\{ M_i\})$. From 
definition~\ref{ordmeas} we have also defined the order measures 
$\nu_D$ and $\nu_E$ on $
%(\bd^1_3)^n
\rc{mc71}{(\bd^1_3)^{n-1}}
$. 


\begin{example}
Suppose $\ods{p_i}{\bar{d}}=(\gamma^1_{1,1},k^1_5(\cdot_1), k^2_5(\cdot_1), \gamma^1_{1,2},
k_3^1(\cdot_2), k^1_1(\cdot_2)
%%
\rc{mc72}{)}
$. Then 
$(t^i_0,\dots,t^i_5)=(\gamma^1_{1,1}, \gamma^1_{1,2}, k^1_5(\cdot_1),
k^2_5(\cdot_1), k^1_1(\cdot_2), k^1_3(\cdot_2))$. The measure $M_i$ is equal to 
$M_i=W_1^1 \times W_1^1 \times S_1^1 \times S_1^1 \times S_1^2 \times S_1^2$. 
The permutation
$\pi_i$ is equal to $(0,2,3,1,5,4)$. 
Also, $(u^i_0,\dots,u^i_5)=(\gamma_{1,1}, \gamma_{1,2}, k_5(\cdot_1),
k_5(\cdot_1), k_1(\cdot_2), k_3(\cdot_2))$. 
So, $j_0=0$, $j_1=2$, and $j_3=5$. 
Thus, $N_i=W_1^1 \times S_1^1 \times S_1^2$.
\end{example}





Notice that for all non-trivial blocks $i$, $(E_i,N_i)$
is the order type and measure corresponding to a subsequence of the
canonical sequence of $\pi_i$ (it may be a proper subsequence 
since in the $t^i_j$ we keep the superscripts on the terms while
for the $u^i_j$ we do not). In the example just considered, the canonical
increasing subsequence of $\pi_i$ would be $(0,2,3,5)$,
while  $(j_0,j_1,j_2)$ is the proper subsequence $(0,2,5)$.



From lemma~\ref{techlem} we have that 
$j_{\nu_E}(\bd^1_3) \leq 
%j_{\nu_E}
\rc{mc73}{j_{\nu_D}}
(\bd^1_3)$. 
From lemma~\ref{basic}
we have that $j_{\nu_E}(\bd^1_3) \geq \aleph_{\omega+c(E)+1}$.
The ordinal $c(E)$ is just the ordinal $\xi_{\bar{d}}$ and so 
$j_{\nu_E}(\bd^1_3) \geq \aleph_{\omega+ \xi_{\bar{d}}+1}$.
From corollary~\ref{lb} we have that 
$(\id;d;W^m;\bar{S}) \leq  \aleph_{\omega+ \xi_{\bar{d}}+1}$.
Putting this together we have:


$$j_{\nu_D}(\bd^1_3)\ \geq \ j_{\nu_E}(\bd^1_3)\
 \ge\  
%\al{\omega  + \xi_{\bar{d}}   
\rc{mc74}{\al{\omega  + \xi_{\bar{d}}+1 }  
  } \geq (\id;d;W^m;\bar{S})
$$


In the remainder of the proof we show that 
that $j_{\nu_D}(\bd^1_3) \leq (\id;d;W^m;\bar{S})$, which
shows that equality holds in the above inequalities, and
completes the proof of lemma~\ref{mainlem}.



We define an embedding $\phi\colon  j_{\nu_D}(\bd^1_3) \to
(\id;d;W^m;\bar{S})$. Fix $[G]_{\nu_D}$, where $G\colon 
%\bd^1_3 
\rc{mc75}{(\bd^1_3)^{n-1}} 
\to
\bd^1_3$. $\phi([G]_{\nu_D})$ will be represented with respect to
$W^m$, $S_1, \dots, S_t$ (as in the definition of $(\id;d;W^m;\bar{S})$)
by $\phi([G]_{\nu_D})(f,h_1,\dots,h_t)$. That is, 
$\phi([g]_{\nu_D})$ is represented in the ultrapower by the measure 
$W^m$ by the function $
[f]_{S_1^m} 
\rc{mc75!}{}
\mapsto  \phi([G]_{\nu_D})([f])$.
The value $\phi([G]_{\nu_D})([f])$ is then represented with respect to the measure 
$S_1$ by the function $[h_1] \mapsto \phi([G]_{\nu_D})([f],[h_1]))$, etc.,
and where $$\phi([G]_{\nu_D})(f,h_1,\dots,h_s)=G([g]),$$ where
$g\colon  D \to \bd^1_3$ is defined as follows. As usual, $[h_1]$
here means $[h_1]_{W_1^r}$ if $S_1=S_1^r$, and if $S_1=W_1^r$
then $[h_1]$ simply means $h_1$ (in this case $h_1 \in (\omega_1)^r$). 
We have also suppressed writing the equivalence class notation. 


It remains to define $g$, and for this it 
suffices to define
$g_i=g \res D_i$ for each $i$. If 
%$i$ is a trivial block, 
\rc{mc76?}{$i$ is a trivial block}, 
that is,
$D_i=1$, then set $g_i(0)= (\id;p_i;f;h_1,\dots,h_t)$.
Recall that $p_i=q_i$ in this case. 
Fix a 
%non-trivial block $i$.  
\rc{mc76?}{non-trivial block $i$}.  
To ease notation, let $t^*=(t_0, \dots,t_{l^*})$
be the terms of $\ods{p_i}{\bar{d}}$ written in increasing order
(in the order \ref{osord}), and write $K_{t+1}, \dots, K_u$ for
$K_{t+1}(i), \dots, K_{u_i}(i)$ (so $p_i \in \cD_m(\bar{S},\bar{K})$). 
Recall each term $t_l$ of $\ods{p_i}{\bar{d}}$
is of the form $t_l=\gamma^{a}_{i,j}$ or
$t_l=k^{a}_{i}(\cdot_{r})$.






We must define $g_i(\beta_0, \dots,
\beta_{l^*})$ where $\bar{\beta}$ is as in the definition of $D_i$.
Fix such $\beta_0,\dots,\beta_{l^*}$, and for $\beta_l > \omega_1$,
let $\beta_l=[\tau_l]_{W^{r_l}_1}$, where $\tau_l\colon  \dom(<_{r_l}) \to
\omega_1$ is of the correct type and $t_l=k^a_i(\cdot_{r_l})$. 



Finally, define $g_i(\beta_0,
\dots, \beta_{l^* })=
(\id;p_i;f;h_1, \dots, h_t;\beta_0, \dots, \beta_{l^* })^*$.
Roughly speaking, this is defined as $(\id; p_i;f;h_1, \dots, h_t;
k_{t+1}, \dots, k_u)$, except that for subdescriptions $q$ of $p_i$ corresponding
to terms $t_l$ of $\ods{p_i}{\bar{d}}$, the interpretation of the
description, $(q;\bar{h},\bar{k})$, is replaced by
$\beta_l$. The key point is that 
since no functional composition of the $k_i$ functions is involved
in interpreting any term from the o--sequence, the 
evaluation of these terms is well-defined with respect to the 
ordinal product measure $K_{t+1}\times \cdots \times K_u$
(where $\bar{K}=K_{t+1},\dots,K_u$). 
This, of course, is not true for the measures $\bar{S}=S_1,\dots,S_t$. 
This is where we use the fact that we have ``linearized'' the description with
respect to the $\bar{K}$ measures. A more precise definition follows. 
In this definition we recall for convenience our notation. 



\begin{defn}  \label{modint}
Let $\bar{d}=(d;\bar{S})$ where $d \in \cD_m(\bar{S})$ and
$\bar{S}=S_1,\dots,S_t$.  
Let $p_i \in \cD_m(\bar{S},\bar{K})$ and let 
$(t_0,\dots,t_{l^*})$ be the sequence of terms from $\ods{p_i}{\bar{d}}$
written in increasing order using the ordering of \ref{osord}. 
Let  $h_1,\dots, h_t$ be functions in the function space measures 
$\sS_1,\dots,\sS_t$ (i.e., if $S_i=W_1^r$ then $h_i \in (\omega_1)^r$ and
if $S_i=S_1^r$ then $h_i \colon \dom(<_r) \to \omega_1$ is of the correct type).
Let $(\beta_0,\dots,\beta_{l^*})$ be an increasing sequence of ordinals 
with $\beta_j <\omega_{r_j+1}$ where $r_j=0$ if $t_j=\gamma^a_{b,c}$
and $r_j=r$ if $t_j=k^a_b(\cdot_r)$. We assume that the sequence of functions and ordinals
$h_1,\dots,h_t,\beta_0,\dots,\beta_{l^*}$ is in ``general position,''
that is: (1) if $\vec\gamma \in (\omega_1)^p$ occurs before 
$\vec\delta \in (\omega_1)^q$ in the sequence, then $\gamma_p< \delta_1$,
(2) If $f$ occurs before $g$ in the sequence then $[f(1)]<[g(1)]$
where $f$ and $g$ are either functions from some $\dom(<_r)$ to $\omega_1$
or ordinals below $ \omega^{r+1}$ (recall here definition~\ref{definvariants}
and the remarks immediately following), and (3) each $h_i$ takes values in a 
c.u.b.\ set closed under the $h_j(1)$ for all $j<i$. Likewise each 
$\beta_i$ can be  represented by $\tau_i$ which takes values in a c.u.b.\
set closed under all the 
%$h_j(1)$. 
\rc{mc77?}{$h_j(1)$ and all the $\tau_j(1)$}. 
Let $f \colon \omega_{m+1}
\to \bd^1_3$ be of continuous type. 




Then we define $(\id;p_i,f;\bar{h},\bar{\beta})^*= 
f((p_i;\bar{h},\bar{\beta})^*)$. Finally, $(p_i;\bar{h},\bar{\beta})^*$
is defined as follows. More generally, we define $(q;\bar{h},\bar{\beta})^*$ 
for any subdescription $q$ of $p_i$ (including $p_i$) of the form 
$\alpha_{i,j}$, $\cdot_r$, $h_i^{(s)}(\cdots)$, or 
$q=k_i^{(s)}(\cdots)$ or $q=\gamma_{i,j}$ and where in these last two cases 
we assume that $\od{q}{\bar{d}}$, which is of the form $k_b(\cdot_r)$ or $\gamma_{b,c}$, 
contributes a  term 
$t_e=k_b^a(\cdot_r)$ or $t_e=\gamma^a_{b,c}$
to  $\ods{p_i}{\bar{d}}$. 
We let $(q;\bar{h},\bar{\beta})^*$ be represented with respect to $W_1^m$
by the function $(\alpha_1,\dots,\alpha_m) \mapsto 
(q;\bar{h},\bar{\beta})^*(\bar{\alpha})$ which is defined inductively 
as follows (for the subdescriptions $q_i$ of $q$
we abbreviate $(q_i; \bar h; \bar \beta)^*(\bar \alpha)$ 
by writing just $(q_i)^*(\bar \alpha)$). 




\begin{enumerate}
\item
If $q=\alpha_{i,j}$ then 
$(q;\bar{h},\bar{\beta})^*(\bar{\alpha})=\alpha_{i,j}$.
\item
If $q=\cdot_r$ then $(q;\bar{h},\bar{\beta})^*(\alpha_1,\dots,\alpha_m)=\alpha_r$.
\item
If $q=h_i(l+1)(q_1, \dots,q_l,q_0)$, then 
$$(q;\bar h; \bar \beta)^*(\bar{\alpha})= 
h_i(l+1)((q_1)^*(\bar{\alpha}),
\dots, (q_l)^*(\bar{\alpha}), 
(q_0)^*(\bar{\alpha})).$$

\item
If $q=h_i^s(l+1)(q_1, \dots,q_l,q_0)$, then
$$(q;\bar h;\bar \beta)^*(\bar{\alpha})= 
h_i^s(l+1)((q_1)^*(\bar{\alpha}),
\dots, (q_{l})^*(\bar{\alpha}), (q_0)^*(\bar{\alpha})).$$

\item
If $q=\gamma_{i,j}$, and corresponds to $t_e=\gamma^a_{i,j}$, then
$$(q;\bar h; \bar \beta)^*(\bar{\alpha})= \beta_e < \omega_1.$$
\item \label{lastcase}
If $q= k^{(s)}_i(l+1)(q_1,\dots,q_l,q_0)$,  
then by assumption $\od{q}{\bar{d}}$ 
corresponds to a term, say $t_e=k_b^a(\cdot_{r})$ or $t_e=\gamma^a_{b,c}$, 
of $\ods{p_i}{\bar{d}}$.
Then 
$$(q;\bar h; \bar \beta)^*(\bar{\alpha})=\tau_e(\alpha_1, \dots, \alpha_{r})
%.
\rc{mc78:.}{}
$$
in the first case and $(q;\bar h; \bar \beta)^*(\bar{\alpha})= \beta_e$
in the second case.
\end{enumerate}
\end{defn}











Note that the definition in case (\ref{lastcase}) makes sense
since if $t_e=k_b^a(\cdot_r)$, then 
$\beta_e$ is represented by $\tau_e \colon 
\dom(<_r) \to \omega_1$. 


\begin{rem} \label{subrem}
We have not necessarily defined $(q;\bar{h},\bar{\beta})^*$ 
for all subdescriptions of $p_i$. If we start from $p_i$ and descend
along a branch of the ``tree of subdescriptions'' of $p_i$,
we have defined $(q;\bar{h},\bar{\beta})^*$ up to and including the 
first point where $q$ is of the form $q=\gamma_{i,j}$
or $q=k_i^{(s)}(l+1)(q_1,\dots,q_l,q_0)$. In the latter case, for example, we have
not necessarily defined the $(q_j;\bar{h},\bar{\beta})^*$. This is enough, however,
to give a definition of $(p_i;\bar{h},\bar{\beta})^*$. 
\end{rem}




We first note that for fixed $h_1,\dots,h_t, \beta_0,\dots,\beta_{l^*}$
in general position that $(p_i;\bar{h},\bar{\beta})^*$ is well-defined. 
This uses two facts. First, the definition of 
each $(q;\bar{h},\bar{\beta})^*$ depends only on the $\beta_j$
and not on the functions $\tau_j$ chosen to represent them. This is
clear from the definition. Second, when $q=h_i^{(s)}(l+1)(q_1,\dots,q_l,q_0)$
then for almost all $\bar{\alpha}$ we have that 
$(q_1)^*(\bar{\alpha})< \cdots < (q_l)^*(\bar{\alpha})<(q_0)^*(\bar{\alpha})$. 
Recall from the definition of $\ods{p_i}{\bar{d}}$ that 
if a term $k^a_b(\cdot_r)$ of $\ods{p_i}{\bar{d}}$ comes from 
$\ods{q_j}{\bar{d}}$ and the term $k^{a'}_b(\cdot_r)$ comes from 
$\ods{q_{j'}}{\bar{d}}$ and $q_{j'}$ is to the right of $q_j$ 
(i.e., $0<j<j'$ or $j'=0$) then $a<a'$. This is how we attached the superscripts
in the definition of $\ods{p_i}{\bar{d}}$. So, to prove this second fact it suffices to show
the following claim.



\begin{claim} \label{claimmta}
Suppose $q<q'$ are subdecriptions of $p_i$ for which we have defined
$(q;\bar{h},\bar{\beta})^*$ and $(q';\bar{h},\bar{\beta})^*$ (see remark~\ref{subrem}).
Suppose also that if $k^a_b(\cdot_r)$ and $k^{a'}_b(\cdot_r)$ are terms
of $\ods{p_i}{\bar{d}}$ coming from $q$ and $q'$ respectively, then $a<a'$
(and similarly for terms $\gamma^a_{b,c}$, $\gamma^{a'}_{b,c}$).
Then $(q;\bar{h},\bar{\beta})^* < (q';\bar{h},\bar{\beta})^*$.
\end{claim}


\begin{proof}
By reverse induction on $\min\{ k(q),k(q')\}$. Suppose first that 
$q=h_k^{(s)}(l+1)(q_1,\dots,q_l,q_0)$ and $k(q')>k$. Since $q<q'$
we have from I.2.\ of 
%lemma~\ref{llemma} 
\rc{mc79}{lemma~\ref{lemreform}} 
that $q_0<q'$. By induction, 
for almost all $\bar{\alpha}$ we have 
$(q_0;\bar{h},\bar{\beta})^*(\bar{\alpha})<
(q';\bar{h},\bar{\beta})^*(\bar{\alpha})$ and using the fact that 
the $h_l$ for $l>k$ and the $\tau_e$ have (almost everywhere) range
in a set closed under $h_k(1)$ (and $k(q')>k$) it easily follows that 
$(q;\bar{h},\bar{\beta})^*(\bar{\alpha})<
(q';\bar{h},\bar{\beta})^*(\bar{\alpha})$. The remaining cases where 
one of $q$ or $q'$ has the form $h_k^{(s)}(\dots)$ are handled by induction
in a similar fashion. The cases where one of $q$, $q'$ has the form 
%$\alpha_{i,j}$ 
\rc{mc80}{$\alpha_{i,j}$ or $\gamma_{i,j}$} 
are essentially trivial. So, suppose 
$q=k_i^{(s)}(l+1)(q_1,\dots,q_l,q_0)$ and $q'=k_{i'}^{(s)}(l'+1)
(q'_1,\dots,q'_{l'},q'_0)$ (the cases where one of $q$, $q'$
is equal to $\cdot_r$ are easy). In this case $\od{q}{\bar{d}}$
and $\od{q'}{\bar{d}}$ both consist of a single term. We consider the case where 
$\od{q}{\bar{d}}=k_b(\cdot_r)$ and $\od{q'}{\bar{d}}=k_{b'}(\cdot_r)$,
the other cases being similar. From the definition of the o--sequence and I and II of 
%lemma~\ref{llemma} 
\rc{mc79}{lemma~\ref{lemreform}} 
it follows that we must have $b \leq b'$. 
If $b<b'$, then $\od{q}{\bar{d}}$ contributes  the term $k^a_b(\cdot_r)=t_e$
to $\ods{p_i}{\bar{d}}$ for some $a$, and $\od{q'}{\bar{d}}$
contributes the term $k^{a'}_{b'}(\cdot_r)=t_{e'}$. By the ordering on these terms
(definition~\ref{osord}) we have $e<e'$. We therefore have that 
$[\tau_e(1)]<[\tau_{e'}(1)]$. In particular, 
%$\tau_3 <\tau_{e'}$
\rc{mc81}{$\tau_e <\tau_{e'}$}
almost everywhere and we are done by case~\ref{lastcase}
of definition~\ref{modint}. If $b=b'$ then $\od{q}{\bar{d}}$
contributes the term $k^a_b(\cdot_r)=t_e$ and $\od{q'}{\bar{d}}$
contributes the term $k^{a'}_b(\cdot_r)=t_{e'}$ for some $a<a'$. 
Again it follows that $e<e'$ and we are done (we use now our hypothesis
on the superscripts stated in the claim). 
\end{proof}



It follows that we have shown that for fixed $f$ and $h_1,\dots, h_t$,
$\beta_0,\dots,\beta_{l^*}$ (in general position)
that $(\id;p_i;f;\bar{h},\bar{\beta})^*$ is well-defined. 
In thus follows that for fixed $G$, $f$, $h_1,\dots, h_t$, 
$\beta_0,\dots,\beta_{l^*}$ that the function $g_i \colon 
D_i \to \bd^1_3$ defined above is well-defined. 







Next, we claim that for fixed $G$, that
$\forall^\ast_{W^m} [f]$, if $[f]=[f']$ then $\forall^\ast_{S_1} [h_1]$, if
$[h_1]=[h'_1], \dots$, $\forall^\ast_{S_t} [h_t]$ if $[h_t]=
[h'_t]$ then: 
$$\forall 1 \leq i \leq n \ \forall^\ast_{M_i}
\beta_0, \dots, \beta_{l^*_i}\ g_i(\bar{\beta})=
f((p_i; \bar{h}, \bar{\beta})^*)=  f'((p_i;\bar{h'}, \bar{\beta})^*)=
g'_i(\bar{\beta}).$$
To see this, note that we may assume that for $i<j \leq t$ that 
$h_j$, $h'_j$ have range in the limit points $C'_i$ of a c.u.b.\ set $C_i$ on which 
$h_i$, $h'_i$ agree. We may also assume that all $h_i$ have range in 
the limit points $C'$ of 
a c.u.b.\ set $C$
defining measure one sets $S_1^\pi$ on which $f$ and $f'$ agree. We may also assume
that the $h_1,\dots,h_t$ are in general position. Then for 
$M_i$ almost all $(\beta_0,\dots,\beta_{l^*})$ this sequence is in general position
and for each $e$ the function $\tau_e$ representing $\beta_e$ 
(or $\beta_e$ itself if $M^u_e=W_1^1$) has range in $C' \cap \bigcap_{i\leq t} C'_i$.
From this and the claim above it follows that 
$(p_i; \bar{h}, \bar{\beta})^*=(p_i;\bar{h'}, \bar{\beta})^*$ and
that $\forall_{W_1^m} \bar{\alpha}\ (p_i; \bar{h}, \bar{\beta})^*(\bar{\alpha})
\in C'$. From this and the fact that $p_i$ satisfies condition~D it follows that 
$f((p_i; \bar{h}, \bar{\beta})^*)= f'((p_i; \bar{h}, \bar{\beta})^*)$.
We use here the fact that if $h \colon (\omega_1)^m \to C'$ satisfies 
$h(\alpha_1,\dots,\alpha_m) >\alpha_m$ almost everywhere, then 
$[h]_{W_1^m}$ is a limit of ordinals of the form $[h']_{W_1^m}$
where $h' \colon (\omega_1)^m \to C$ is of type $\pi$ for some 
permutation $\pi$.
Note again that since we are using the product measure $M_i$
to quantify over the $\beta_0,\dots,\beta_{l^*}$ it is important
that no composition of the $k_i$ functions occur in the definition of
$(p_i;\bar{h},\bar{\beta})$. 
We have now shown that for fixed $G$, the ordinal $\phi(G)$ is well-defined. 













The proofs that $\phi$ depends only on $[G]_{\nu_D}$, and that $\phi$
is one-to-one are similar. So, suppose $[G_1]_{\nu_D}=[G_2]_{\nu_D}$.  
Let $C \subseteq \bd^1_3$ be c.u.b.\ such that if $g\colon \dom(<_D) \to C$
is of the correct type, then $G_1([g])=G_2([g])$.  Let $C'=\{ \alpha
\in C\colon \alpha$ is the $\alpha^{th}$ element of $C \}$.  Consider
$f$, $h_1, \dots,h_t$ such that $f$ has range in $C'$ and  the $h_i$ are
of the correct type and in general position.
Let $g\colon \dom(<_D) \to \bd^1_3$
be the function defined in the definition of $\phi$.  Since $f$ has
range in $C'$, so does $g$. It remains to show that $g$ is order-preserving
and of uniform cofinality $\omega$. 
%
\rc{MajorC2.9!}{The domain of $D$ may be regarded
as lexicographic order} 
%
on the sequences 
$(i, \beta_{\pi_i(0)},\beta_{\pi_i(1)},\dots,\beta_{\pi_i(l^*_i)})$
where $1\leq i \leq 
%n
\rc{mc82}{n-1}
%
$ ($n$ is the number of blocks), 
$(\beta_0,\dots,
%\beta_{l^*}
\rc{mc83}{\beta_{l_i^*}}
) \in \dom(M_i)$, and $\pi_i$ is the permutation of
$\{ 0,\dots, l^*_i\}$ defined in the definition of $D_i$
(so $\beta_{\pi_i(0)}, 
%\dots,,
\rc{mc84:,}{\dots,}
\beta_{\pi_i(l^*_i)}$ 
corresponds to the 
order of appearance of the terms in $\ods{p_i}{\bar{d}}$). To show $g$
is order-preserving, we show that for fixed $f$, $h_1,\dots,h_t$, that 
if 
$$
(i, \beta_{\pi_i(0)},\beta_{\pi_i(1)},\dots,\beta_{\pi_i(l^*_i)})
<_\lex
(i', \beta'_{\pi_i(0)},\beta'_{\pi_i(1)},\dots,\beta'_{\pi_i(l^*_i)})
$$
then $(p_i;\bar{h},\bar{\beta})^* < (p'_i;\bar{h},\bar{\beta'})^*$.
Suppose first that $i<i'$. 
Then, $p_i \leq q_i <p_{i'}$ (if $p_{i'} \leq q_i$ then $q_{i'}=
%\sup_{\bar{K}(i')}\leq q_1
\rc{mc85}{\sup_{\bar{K}(i')}p_{i'}\leq q_i}
$ by \ref{lemonsup2} of lemma~\ref{supprop}). 
It suffices to prove that $(p_i;\bar{h},\bar{\beta}(i)) \leq 
(q_i;\bar{h}) <(p_{i'};\bar{h},\bar{\beta}(i'))$ (here $\bar{\beta}(i)$, 
$\bar{\beta}(i')$ refer to elements of $\dom(M_i)$, 
%$\dom(M_{i'})$
\rc{mc:,}{$\dom(M_{i'})$,} 
respectively).  
If $p_i=q_i$ (that is, 
$\bl{q_i}{\bar{d}}$ is trivial) then $(p_i;\bar{h},\bar{\beta}(i))
=(p_i;\bar{h})=(q_i;\bar{h})$. 
%
\rc{mc86!?}{}
So, it suffices to show the following. 


\begin{claim} \label{claimmtb}
Suppose $q$ is a subdescription of $p_i$ for which $(q;\bar{h},\bar{\beta})^*$ is defined. 
Suppose $q' \in \cD_m(\bar{S})$. If $q<q' $ (respectively 
%$q'>q$
\rc{mc87}{$q>q'$}
) then 
$(q;\bar{h},\bar{\beta})^*< (q';\bar{h})$ (respectively 
$(q;\bar{h},\bar{\beta})^*> (q';\bar{h})
%
\rc{mc88}{)}
$ for all $\bar{h}$, $\bar{\beta}$
in general position.
\end{claim}


\begin{proof}
The proof is by reverse induction on $\min\{ k(q), k(q')\}$ and is 
similar to that of 
%claim~\ref{claimmta}. 
\rc{mc89:p37}{claim~\ref{claimmta}}. 
The cases where $\min \{ k(q),k(q')\}
\leq 
%t
\rc{\qquad\qquad mc90}{\infty}
$ (recall $\bar{S}=S_1,\dots,S_t$) follow in a straightforward manner by induction.
The remaining case is when $q$ is of the form $q=k_i^{(s)}(l+1)(q_1,\dots,q_l,q_0)$
and $q'=\cdot_r$. Then $\ods{q}{\bar{d}}$ consists of a single
term $k^a_b(\cdot_{r'})$ or $\gamma^a_{b,c}$. In the first case, since $q<q'$
it follows that $r'<r$. In this case we have that for almost all
$\bar{\alpha} \in (\omega_1)^m$ that 
$(q;\bar{h},\bar{\beta})^*(\bar{\alpha})= \tau_e(\alpha_1,\dots,\alpha_{r'})<\alpha_r$,
where the term $k^a_b(\cdot_{r'})$ corresponds to the factor $M^i_e$ of $M_i$
(that is, $t^i_e=k^a_b(\cdot_{r'})$). The case where $q=\gamma^a_{b,c}$
is clear as then $(q;\bar{h},\bar{\beta})^*<\omega_1$. 
\end{proof}




Suppose next that $i=i'$. Let $p \leq l^*_i$ be least such that 
$\beta_{\pi_i(p)} \neq \beta'_{\pi_i(p)}$, so we have 
$\beta_{\pi_i(p)} < \beta'_{\pi_i(p)}$. Say $v=k^a_b(\cdot_r)$
or $v=\gamma^a_{b,c}$ is the corresponding term of $\ods{p_i}{\bar{d}}$,
that is, $t^i_{\pi_i(p)}=v$. It suffices now to prove the following claim. 



\begin{claim} \label{claimmtc}
Suppose $q$ is a subdescription of $p_i$ with $(q;\bar{h},\bar{\beta})^*$ defined. 
Suppose that $\ods{q}{\bar{d}}$ contains the term $v$. Then 
$(q;\bar{h},\bar{\beta}) <(q;\bar{h},\bar{\beta}')$.
\end{claim}

\begin{proof}
The proof is again by reverse induction on $k(q)$. Suppose first
$q= h_i^{(s)}(l+1)(q_1,\dots,q_l,q_0)$. Let $\bar{l}\leq l$ be the unique integer
such that $v$ corresponds to a term of $\ods{q_{\bar{l}}}{\bar{d}}$. 
For $j<\bar{l}$ we must have that $(q_j;\bar{h},\bar{\beta})^*=
(q_j;\bar{h},\bar{\beta}')^*$ as the definitions of these ordinals use only 
$\beta_{\pi_i(0)},\dots, \beta_{\pi_i(b-1)}$ 
where $t^i_{\pi_i(b)}=v$
(recall that 
$(t^i_{\pi_i(0)},\dots,t^i_{\pi_i(l^*_i)})$ enumerates $\ods{p_i}{\bar{d}}$
and $\ods{p_i}{\bar{d}}$ is the concatenation of $\ods{q_0}{\bar{d}}$,
$\ods{q_1}{\bar{d}}$, up through $\ods{q_l}{\bar{d}}$ with appropriately
labeled superscripts). By induction, 
$(q_{\bar{l}};\bar{h},\bar{\beta})^* < (q_{\bar{l}};\bar{h},\bar{\beta}')^*$. 
That is, for $W_1^m$ almost all $\bar{\alpha}$ we have that 
$(q_j;\bar{h},\bar{\beta})^*(\bar{\alpha})=
(q_j;\bar{h},\bar{\beta'})^*(\bar{\alpha})$ for $j<\bar{l}$ and 
$(q_{\bar{l}};\bar{h},\bar{\beta})^*(\bar{\alpha}) < 
(q_{\bar{l}};\bar{h},\bar{\beta}')^*(\bar{\alpha})$. Since $h_i
\colon \dom(<_{r_i})\to \omega_1$ is order-preserving, it then follows that 
$(q;\bar{h},\bar{\beta})^*(\bar{\alpha})<
(q;\bar{h},\bar{\beta'})^*(\bar{\alpha})$. The remaining cases are when 
$q=\gamma_{i,j}$ or $q=k_i^{(s)}(l+1)(q_1,\dots, q_l,q_0)$. In these cases 
$q$ contributes a single term to $\ods{p_i}{\bar{d}}$ which by assumption
must be $v=t^i_{\pi_i(b)}$. The result then follows immediately from 
$\beta_{\pi_i(b)} < \beta'_{\pi_i(b)}$. 
\end{proof}






We have shown that for fixed $f, h_1,\dots, h_t$ that the function 
$g\colon \dom(<_D) \to \bd^1_3$ is order-preserving when restricted to a $M_i$
measure one set. Clearly $g$ has range in $C'$ since $f$ does. Finally, $g$ has uniform 
cofinality $\omega$. To see this, consider one of the subfunctions $g_i$. 
If $i$ is a trivial block, then $p_i=q_i$ has cofinality $\omega$. 
Then $g_i(0)$ (recall the domain of $g_i$ is the single point $0$ in this case)
is equal to $f((p_i;\bar{h}))$ which has cofinality $\omega$ since $\cof((p_i;\bar{h}))=
\omega$ and $f$ is continuous. 
Suppose $i$ is a non-trivial block. We cannot have $p_i=\alpha_{a,b}$ or $p_i=\gamma_{a,b}$
as then $q_i=\sup_{\bar{K}}(p_i) \leq \v{1}$ and then $q_i$ does not satisfy 
condition~D. If $p_i=k_i^{(s)}(l+1)(q_1,\dots,q_l,q_0)$ then $\od{p_i}{\bar{d}}$
consists of a single term of the form $k_a(\v{r})$ or $\gamma_{a,b}$ and also $q_i=\sup_{\bar{K}}(p_i)=
\v{r+1}$. Since $q_i$ satisfies condition~D we must have $r=m$. However,
this violates $q_i\in \cD_m(\bar{S})$. 
So, $p_i$ has the form $p_i=h_j(l+1)(q_1, \dots, q_l,q_0)$
and from proposition~\ref{cofprop} we  may assume that 
$p_i$ has maximal length, that is,  $S_j=S^{l+1}_1$. 
Since $f$ is continuous, to show $g_i$ has uniform cofinality 
$\omega$ it suffices to show that the function 
$(\beta^i_0,\dots,\beta^i_{l^*_i}) \mapsto (p_i;\bar{h},\bar{\beta})^*$
has uniform cofinality $\omega$. This follows from the fact that $h_j$
has uniform cofinality $\omega$. Namely, if $h'_j \colon 
\dom(<_l) \times \omega \to \omega_1$ induces $h_j$ (i.e., 
$h_j(\bar{\alpha})=\sup_n h'_j(\bar{\alpha},n)$) then 
$$(p_i;\bar{h},\bar{\beta})^*(\bar{\alpha})= 
h_j((q_1;\bar{h},\bar{\beta})^*(\bar{\alpha}),\dots, (q_0;\bar{h},\bar{\beta})
%%
\rc{mc91:1,..,l,0}{^*(\bar{\alpha})}
)$$
is the supremum over $n \in \omega$ of 
$$h'_j((q_1;\bar{h},\bar{\beta})^*(\bar{\alpha}),\dots, (q_0;\bar{h},\bar{\beta})
%%
\rc{mc91:1,..,l,0}{^*(\bar{\alpha})}
,n).$$
Thus, restricted to an $M_i$ measure one set, the function $g$ is order-preserving,
of uniform cofinality $\omega$, and has range in $C'$. 
An easy argument now shows that there is a $g'$ such
that $[g']_{M_i}=[g]_{M_i}$, and $g'$ is everywhere order-preserving and of 
uniform cofinality $\omega$ and with range in $C$. 
Thus, $\phi(G)=\phi(G')$. 
This shows $\phi$ is well-defined and one-to-one. 



Lastly, we observe that $\phi([G])<(\id; d;W^m;\bar{S})$. 
This follows from the fact that only the $q_i$ for $i \geq 2$ 
were used in defining $<_D$ while $q_1=d>q_2$. Namely, it follows from 
%claim~\ref{claimmtb}  
\rc{mc92:p39top}{claim~\ref{claimmtb}}  
that for almost all $f,h_1,\dots, h_t$ that 
$$\sup_{\bar{\beta}}g_2(\bar{\beta}) 
\leq f((q_2;\bar{h}))<f((d;\bar{h}))=(\id;d;f;\bar{h}).$$
We may assume without loss of generality that $f$ takes range in a 
c.u.b.\ $D \subseteq \bd^1_3$ closed under the function $G$ and also
closed under ultrapowers by the measures $M_i$. For such $f$
we then have $(\id;d;f;\bar{h})> G([g])$.








This completes the proof of lemma~\ref{mainlem}, and of
theorem~\ref{mainthm}. As we remarked in the proof of lemma~\ref{mainlem},
we have actually shown the following.


\begin{thm}\label{thm:d_is_c}
Let $d \in \cD_m(K_1, \dots, K_t)$ satisfy condition~D.
Then (where $\xi_{\bar d}$ is defined after definition~\ref{block}):
$$(\id;d;W^m;\bar{K}) = \al{\omega+ \xi_{\bar d}+1}.$$
\end{thm}

\begin{cor}
The successor cardinals  $\bd^1_3 \leq \al{\alpha+1} < \bd^1_5$,
are exactly the
ordinals of the form $(\id;d;W^m;\bar{K})$ for some
$d \in \cD_m(K_1, \dots, K_t)$ satisfying condition~D.
\end{cor}

\begin{proof}
From \cite{J1} all successor cardinals in this range are of necessarily of the form 
$(\id;d;W^m;\bar{K})$ (the results of \cite{J1} are stated for 
the ordinals $(\id;d; W_3^m;\bar{K})$ and $f \colon \omega_{m+1}\to \bd^1_3$
of the correct type, but they immediately show the current claim as well). 
Theorem \ref{thm:d_is_c} gives the converse. 
\end{proof}


\begin{rem}
As mentioned, our definitions are slightly different from
those of \cite{J1}. However, a minor variation of our embedding
argument shows that the ordinals $(\id;d;W_3^m;\bar{K})$ as
defined in \cite{J1} are also cardinals. For the readers familiar
with \cite{J1}, we briefly mention the changes necessary.
Given $(d)$ or $(d)^s$ in $\bar{\cD}_m(\bar{S})$ satisfying ``condition~D'' of \cite{J1}
(which is different from that of this paper) 
one considers now the blocks
$(d)^{(s)}=q_1>q_2 > \cdots > q_n$ where each 
$q_i$ is now of the form $(d')$ or $(d')^s$ and 
satisfies  condition~D of \cite{J1}. The $q_i$ of the form 
$q_i=(d')$ are regarded as trivial blocks in the definition of the ordering 
$D$ used to define $\nu_D$. The blocks of the form $q_i=(d')^s$
are 
%trated 
\rc{mc93}{treated} 
as in the current paper (so these may or may not be trivial). 
One can show that the $p_i$ as in proposition~\ref{cofprop}
can be chosen so that $(p_i)^s$ satisfies condition~D of \cite{J1}. 
The argument then proceeds as in the current paper.
\end{rem}




\section{Applications} \label{applications}


Recall from \S\ref{mainsec} the definitions of a basic order type,
$D$, the ordinal $c(D)$, and the associated measure $\nu_D$. Recall also
lemma~\ref{basic}, which says $j_{\nu_D}(\bd^1_3) \geq
\al{\omega + c(D) +1}$.

We show now that equality holds here, thereby providing another
representation for the successor cardinals $\bd^1_3  <
\al{\alpha+1} < \bd^1_5$.

\begin{thm} \label{basicthm}
For $D$ a basic order type, and associated measure $\nu_D$, we have
$j_{\nu_D}(\bd^1_3)=\al{\omega + c(D) +1}$.
\end{thm}

\begin{proof}
Let $\kappa= \al{\omega+c(D)+1}$.
From Martin's theorem (theorem~\ref{mt2}), $j_{\nu_D}(\bd^1_3)$
is a cardinal, and since $\cf( j_{\nu_D}(\bd^1_3) )>\omega$,
it is a successor cardinal. From \cite{J1}, every successor
$\bd^1_3 < \al{\alpha+1}< \bd^1_5$ is of the form $(\id;d;W^m;\bar{S})$
for some $d \in \cD_m(\bar{S})$. From the equality proved
in lemma~\ref{mainlem}, $$\kappa=(\id;d;W^m;\bar{S})=
j_{\nu_E}(\bd^1_3)= \al{\omega+c(E)+1}$$ for some basic order type $E$.
It thus suffices to show that if $D$, $E$ are basic order types
with $c(D)=c(E)$, then $j_{\nu_D}(\bd^1_3)=j_{\nu_E}(\bd^1_3)$.



This follows from two observations. 
First, if $A$, $B$ are sub-basic order types,
with $c(A)<c(B)$, then $A \oplus B$ strongly embeds into $B$.
This follows from a variation of
proposition~\ref{basicot}. Consider the case 
$A=(\omega_{k+1})^m=\omega_{k+1}\times \cdots \times \omega_{k+1}$
and $B=\omega_{l+1}$ where $k<l$ (in fact, using 
propositions~\ref{basicota} and \ref{basicot} and 
%
\rc{MajorC2.10!}{the observation of the following paragraph,} 
%
the general case can be reduced to this one). 
Let $\mu=S_1^{m+1} \times S_1^1$. For $(\theta_1,\theta_2)
\in \dom(\mu)$ represented by functions 
$h_1\colon \dom(<_{m+1}) \to \omega_1$ and $h_2 \colon \omega_1 \to \omega_1$ of the correct type
with $[h_2(1)]_{W_1^1} >[h_1(1)]_{W_1^1}$, define 
$H(\theta_1,\theta_2)\colon A \oplus B \to B$ as follows. If 
$\bar{\alpha}=(\alpha_1,\dots,\alpha_m) \in A$, with 
$\alpha_i=[f_i]_{W_1^k}$, then define $H(\theta_1,\theta_2)(\bar{\alpha})
=[g]_{W_1^l}$ where $g$ is given by:
$$g(\delta_1,\dots,\delta_l)=
h_1(f_1(\delta_1,\dots,\delta_k),\dots, f_m(\delta_1,\dots,\delta_k), \delta_l).$$
If $\alpha=[f]_{W_1^l} \in B$, then set $H(\theta_1,\theta_2)(\alpha)=[g]_{W_1^l}$
where 
$$g(\delta_1,\dots,\delta_l)=h_2(f(\delta_1,\dots,\delta_l)).$$
As in propositions~\ref{basicota} and \ref{basicot} it can be checked
that $H$ is well-defined and gives a strong embedding.



The second observation is that 
if $D$ is the order type of lexicographic 
ordering on $\omega_{l+1} \times \omega_{k+1}$ 
where $k<l$, and $E$ is the order type $\omega_{l+1}$, then $D$ strongly
embeds into $E$. The proof of this is also similar to that of 
proposition~\ref{basicot}. Let $\mu$ be the measure $S_1^2$. 
For $\theta \in \dom(\mu)=\omega_3$ represented 
by $h \colon \dom(<_2) \to \omega_1$, define $H(\theta) 
\colon D \to E$ as follows. Fix $f_1 \colon \dom(<_l) \to \omega_1$, and 
$f_2 \colon \dom(<_k) \to \omega_1$ representing 
$(\alpha_1, \alpha_2) \in D$. Then $H(\theta)(\alpha_1,\alpha_2)=[g]_{W_1^l}$
where 
$$
g(\delta_1,\dots,\delta_l)=h(f_2(\delta_1,\dots,\delta_k), f_1(\delta_1,\dots,\delta_l)).
$$
It is readily checked that $H$ is well-defined and gives a 
strong embedding from $D$ to $E$.
\end{proof}




\begin{figure}[!h]
\begin{align*}
{}&q_1=h_1(1)(\cdot_2) \\
{}&q_2=h_1(2)(h_2(1)(\cdot_1), \cdot_2) \\
{}&\text{\makebox[30pt]{}} p_2=h_1(3)(h_2(1)(\cdot_1),
        k_3(1)(\cdot_1), \cdot_2)\\
{}&\text{\makebox[30pt]{}}r_2=\omega^\omega \\
{}&q_3=h^s_1(2)(h_2(1)(\cdot_1),\cdot_2)\\
{}&\text{\makebox[30pt]{}} p_3=h_1(3)(h_2(2)(\gamma_{4,1},\cdot_1),
    k_5(1)(\cdot_1), \cdot_2)\\
{}&\text{\makebox[30pt]{}} r_3=\omega^\omega \cdot \omega=\omega^{\omega+1} \\
{}&q_4=h_1(2)(\cdot_1,\cdot_2)\\
{}&\text{\makebox[30pt]{}} p_4=h_1(3)(\cdot_1, k_6(1)(\cdot_1), \cdot_2)\\
{}&\text{\makebox[30pt]{}} r_4=\omega^\omega \\
{}&q_5= h_1(3)(\cdot_1, h_2(1)(\cdot_1),\cdot_2)\\
{}&\text{\makebox[30pt]{}} r_5=1\\
{}&q_6=h^s_1(3)(\cdot_1, h_2(1)(\cdot_1), \cdot_2)\\
{}&\text{\makebox[30pt]{}} p_6=h_1(3)(\cdot_1, h_2(2)(\gamma_{8,1}, \cdot_2)\\
{}&\text{\makebox[30pt]{}} r_6=\omega\\
{}&q_7=h^s_1(1)(\cdot_1, \cdot_2)\\
{}&\text{\makebox[30pt]{}} p_7=h_1(3)(\gamma_{9,1}, k_{10}(1)(\cdot_1), \cdot_2)\\
{}&\text{\makebox[30pt]{}} r_7=\omega^\omega \cdot \omega =\omega^{\omega+1}
\end{align*}
\caption{\label{fig:1}}
\end{figure}






We thus have two ways of representing the successor cardinals
below $\bd^1_5$, and the results of this paper give an algorithm
for converting from one representation to the other. Questions about
the cardinals below $\bd^1_5$ may thus be approached in either manner.
To illustrate this, we compute the cofinality of the  successor
cardinals below $\bd^1_5$.


\begin{thm} \label{cofcomp}
Suppose $\bd^1_3=\al{\omega+1} < \al{\alpha+1} < \al{\omega^{\omega^\omega}
+1}= \bd^1_5$. Let $\alpha=\omega^{\beta_1}+ \dots +
\omega^{\beta_n}$, where $\omega^\omega> \beta_1  \geq  \dots \geq
\beta_n$ be the normal form for $\alpha$. Then:
\begin{itemize}
\item
If $\beta_n=0$, then $\cf(\kappa)=\bd^1_4=\al{\omega+2}$.
\item
If $\beta_n>0$, and is a successor ordinal, then $\cf(\kappa)
=\al{\omega \cdot 2+1}$.
\item
If $\beta_n >0$ and is a limit ordinal, then $\cf(\kappa)=
\al{\omega^\omega +1}$.
\end{itemize}
\end{thm}

We note that $\al{\omega+2}$, $\al{\omega \cdot 2 +1}$, and
$\al{\omega^\omega +1}$ are the three regular cardinals strictly
between $\bd^1_3$ and $\bd^1_5$, and are the ultrapowers of
$\bd^1_3$ by the three normal measures on $\bd^1_3$ (generated by the
c.u.b.\ filter and the possible cofinalities $\omega$, $\omega_1$, $\omega_2$).
This is proved in \cite{J1}.

\begin{proof}[Sketch of proof.]
The proof in all cases is similar, so suppose $\beta_n >0$ and
is a limit. Thus, $\beta_n=\omega^{m_l}+ \omega^{m_{l-1}} + \dots
+\omega^{m_1}$, where $m_l \geq m_{l-1} \geq \dots \geq m_1 >0$.
For $1 \leq i \leq n$, let $D_i$ be the sub-basic order type
corresponding to $\beta_i$, that is, $c(D_i)=\omega^{\beta_i}$.

Let $D=D_1 \oplus \dots \oplus D_n$. Thus,
$D_n$ is lexicographic ordering on 
$\omega_{m_1+1}\times \cdots \times \omega_{m_l+1}$. 
Also, $\al{\alpha+1}=j_{\nu_D}(\bd^1_3)$ from theorem~\ref{basicthm}.
Let $\nu_2$ be the
$\omega_2$--cofinal normal measure on $\bd^1_3$. We embed
$j_{\nu_2}(\bd^1_3)$ cofinally into $j_{\nu_D}(\bd^1_3)$.
Given $[F]_{\nu_2}$, let $\pi([F]_{\nu_2})=[G]_{\nu_D}$, where
for $g=(
%g_l, 
\rc{mc94}{g_n}, 
\dots, g_1)\colon <_D \to \bd^1_3$ of the correct type, 
$G([
%g_l 
\rc{mc94}{g_n} 
],
\dots,[g_1])= F(\sup{g_1})$. Easily, $\pi$ is well-defined and
strictly increasing. An easy partition argument using the weak partition
relation on $\bd^1_3$ shows that $\pi$ is also cofinal.
\end{proof}








Finally, we close by considering an example which illustrates
the arguments of this paper. Let $\bar{S}= (S^3_1,S^2_1)$, $m=2$, and
$d \in \cD_m(\bar{S})$ with functional representation
$d= h_1(1)(\cdot_2)$. Let $\kappa=(\id;d;W^2;\bar{S})$.
The table in Figure \ref{fig:1} lists the descriptions $q_1, \dots, q_7$
determining the blocks $B_2, \dots, B_7$, the $p_i$ giving the depth of
each (non-trivial) block, and the depth  $r_i =\dep{\bl{q_i}{\bar{d}}}$
of eack block.











Thus, $\kappa=\al{\omega^{\omega+1}+\omega+1+
\omega^\omega+\omega^{\omega+1}+\omega^\omega+1}$
$=\al{\omega^{\omega+1} \cdot 2 +\omega^\omega +1}$.
From theorem~\ref{cofcomp}, $\cf(\kappa)=\al{\omega^\omega +1}$.



\begin{thebibliography}{1}

\bibitem{J3}
Steve Jackson.
\newblock Structural consequences of {A}{D}.
\newblock In {\em Handbook of Set Theory},
M.\,Foreman, A.\,Kanamori, M.\,Magidor (\textit{eds.}), to appear.


\bibitem{J2}
Steve Jackson.
\newblock {A}{D} and the projective ordinals.
\newblock In {\em Cabal Seminar 81--85}, volume 1333 of {\em Lecture Notes in
  Math.}, pages 117--220. Springer, Berlin, 1988.

\bibitem{J1}
Steve Jackson.
\newblock A computation of ${\boldsymbol\delta}^1_5$.
\newblock {\em Memoirs of the A.M.S.}, 140(670):1--94, 1999.

\bibitem{Ke1}
Alexander~S. Kechris.
\newblock {A}{D} and the projective ordinals.
\newblock In {\em Cabal seminar 76--77}, volume 689 of {\em Lecture Notes in
  Math.}, pages 91--132. Springer, Berlin, 1978.

\bibitem{Ke}
Alexander~S. Kechris.
\newblock {\em Classical Descriptive Set Theory}, volume 156 of {\em Graduate
  Texts in Mathematics}.
\newblock Springer-Verlag, 1994.

\bibitem{Mo}
Yiannis~N. Moschovakis.
\newblock {\em Descriptive set theory}, volume 100 of {\em Studies in logic}.
\newblock North--Holland, 1980.

\end{thebibliography}















\end{document}








Source for bibliography follows, from file desc.bib.
\bibliographystyle{plain}
\bibliography{desc}


@article{J1,
   AUTHOR = {Jackson, Steve},
   TITLE  = {A computation of ${\boldsymbol\delta}^1_5$},
   JOURNAL   = {Memoirs of the A.M.S.},
  FJOURNAL= {Memoirs of the American Mathematical Society},
   VOLUME = {140},
   NUMBER ={670},
     YEAR ={1999},
    PAGES={1--94},
}

@incollection {J2,
   AUTHOR = {Jackson, Steve},
    TITLE = {{A}{D} and the projective ordinals},
BOOKTITLE = {Cabal Seminar 81--85},
    PAGES = {117--220},
   SERIES = {Lecture Notes in Math.},
   VOLUME = {1333},
PUBLISHER = {Springer},
  ADDRESS = {Berlin},
     YEAR = {1988},
}



@incollection {J3,
   AUTHOR = {Jackson, Steve},
    TITLE = {Structural consequences of {A}{D}},
BOOKTITLE = {Handbook of Set Theory},
    COMMENT= {To appear}
}




@book {Ke,
 AUTHOR={Kechris, Alexander S.},
TITLE={Classical Descriptive Set Theory},
SERIES={Graduate Texts in Mathematics},
VOLUME={156},
PUBLISHER={Springer-Verlag},
YEAR={1994}
}




@incollection {Ke1,
   AUTHOR = {Kechris, Alexander S.},
    TITLE = {{A}{D} and the projective ordinals},
BOOKTITLE = {Cabal seminar 76--77},
    PAGES = {91--132},
   SERIES = {Lecture Notes in Math.},
   VOLUME = {689},
PUBLISHER = {Springer},
  ADDRESS = {Berlin},
     YEAR = {1978},
}



@incollection {Ke2,
   AUTHOR = {Kechris, Alexander S.},
    TITLE = {Homogeneous trees and projective scales},
BOOKTITLE = {Cabal seminar 77--79},
    PAGES = {33--73},
   SERIES = {Lecture Notes in Math.},
   VOLUME = {839},
PUBLISHER = {Springer},
  ADDRESS = {Berlin},
     YEAR = {1981},
}



@book{Mo,
   AUTHOR = {Moschovakis, Yiannis N.},
   TITLE  = {Descriptive set theory},
   PUBLISHER = {North--Holland},
   YEAR   = {1980},
   SERIES = {Studies in logic},
   VOLUME = {100},
}






%55555555555555555555555555555555555555555555555

%\title{Descriptions and Computation of Ultrapowers in $\LR$ }
%\author{Farid T. Khafizov}
\email{farid.khafizov@gmail.com}
\keywords{Cardinals, Descriptions, Determinacy, Projective Ordinals}
\subjclass[2000]{\bf 03E60}


\begin{abstract}
Assuming $\ad$, we provide exact computation of ultrapowers by measures on cardinals
$\aleph_n, n\in \omega$, in $\LR$, and a proof that ordinals in $\LR$ below 
$\bd^1_5$ represented by descriptions and the identity function with respect to sequences of measures are cardinals. 
\end{abstract}

%\maketitle
\section{Ultrapowers of $\aleph_\alpha$ for $\alpha<\omega_1$ by measures on $\aleph_n, n<\omega$}
\subsection{Introduction}
{\sloppy
We use $\pi_m$ to denote a partial permutation $(m,i_1,\dots,i_{m-1})$ of $(1,2,\dots,m)$. Given any such permutation $\pi$, we define the corresponding wellorder, $<^\pi$, on tuples $\bar{\alpha}:=(\alpha_1,\dots,\alpha_m)\in \aleph_1^m$, with $\alpha_1<\dots<\alpha_m$, by
$$\bar{\alpha}<^\pi \bar{\beta} \iff \pi(\bar{\alpha}) <^{lex} \pi(\bar{\beta})
$$
}
\begin{defn}
$h:<^{\pi_m}\to\aleph_1$ is of \textbf{c-correct type (c-c.t.)}, $0\le c\le m$, if (1) $h$ is strictly increasing, (2) everywhere discontinuous, and (3) $\forall \bar{\alpha}$, uniformly $\cof h(\bar{\alpha})=\alpha_c$, where $\alpha_0=\omega$.
\end{defn}
Sometimes we write c.t. for 0-c.t.

\begin{defn}[Basic Measure $S^{\pi_m,c}$]
If $A\subseteq \aleph_{m+1}$, then
$$
A\in S^{\pi_m,c} \iff \exists \textrm{ c.u.b. } C\subset \aleph_1 \textrm{ s.t. }\forall h:<^{\pi_m}\to C \textrm{ of c.t. }, [h]_{\mu^m}\in A 
$$
\end{defn}

To define a \emph{general measure}, 
fix a function $\sF_m: U\to \cup_{m,\pi_m,0\le c\le m} \{ S^{\pi_m,c} \}$ 
with finite domain $U\subset \omega^{\le m}$ such that $\forall u\in U$ with
$\lh(u)=k,\, \sF_m(u)=S^{\pi_k,c}$ \rc{c!}{}. 
It is assumed that $U$ contains at least one sequence of length $m$.

On tuples $<\ns u,\lambda\ns >\in (\omega\times \aleph_1)^{\le m}$, where $u\in U$ and 
$\lambda(1)<\dots<\lambda(k)$, we define a well-order $<_*$ as follows.
Let $u, u'\in U, \, k=\lh(u),\, k'=\lh(u')$; assume  $\sF_m(u)=S^{\pi_k,c}, \,
\sF_m(u')=S^{\pi_{k'},c'}$ with $\pi=(k,i_2,\dots,i_k)$,   $\pi'=(k',j_2,\dots,j_{k}')$. Then

\begin{equation*}
\begin{split} 
<\ns u,\lambda\ns >\; & <_* \; <\ns u',\lambda'\ns >  \quad \iff \quad \\
& (\lambda(k),u(1),\lambda(i_2),u(2),\dots,\lambda(i_k),u(k)) \, <^{lex} \, \\
& \qquad (\lambda'(k'),u'(1),\lambda'(j_2),u'(2),\dots,\lambda'(j_k),u'(k))
\end{split}
\end{equation*}

Let $\sH:<_*\to\aleph_1$ be order preserving and discontinuous. For each $u\in U$ we define $h_u(\lambda):=\sH(u,\lambda)$. We say $\sH$ is of the type \emph{matching with} $\sF_m$ if $\forall u\in U$
$$
h_u \textrm{ is of $c$-c.t. } \quad \iff \quad \sF_m(u)=S^{\pi_k,c}
$$
In other words, the function $h_u$ is one of those on which the measure $\sF_m$ concentrates.

Let us enumerate $U=\{u_i\}^p_{i=1}$ in lexicographically increasing order (here $p=|U|$). We set
\begin{align*}
\delta_i & := [h_{u_i}]_{\mu^{k_i}},\quad \textrm{for all } u_i\in U,\, k_i=\lh(u_i)  \\
\delta(\Cal{H}) & := \textrm{ot}(\delta_1,\dots,\delta_p)\in \aleph_{m+1}
\end{align*}

Notice that given the function $\sF_m$ with the domain $U$, for each function $\sH:<_*\to\aleph_1$ of the type matching with $\sF_m$, the ordinal $\delta=\delta(\sH)$ is determined uniquely. One thinks of the measure $\sG_{\sF_m}$ being concentrated on such functions $\sH$. We summarize all these in the formal
\begin{defn} [General Measure $\sG_{\sF_m}$]
Let $\sA\subseteq \aleph_{m+1}$.
We set 
$ \sA \in\sG_{\sF_m}$ iff $\exists \textbf{ c.u.b. } C\subset \aleph_1,
\, \forall\sH:<_*\to C \, \textrm{ of the type matching with }  \sF_m, \delta(\sH)\in\sA
$
\end{defn}

Observe that every basic measure $S^{\pi,c}=\sG_\sF$, where the domain of $\sF$ is a singleton $\{u\}$ and $\sF(u)=S^{\pi,c}$.

A measure $\sG_{\sF_m}$ (or simply $\sG$) can be represented by a finitely splitting tree of height $m+1$. The root of the tree is the empty set, and every maximal path is an element, $u$, of the domain of $\sF_m$ with the terminal node $(\pi_k,c)$ if and only if $\sF(u)=S^{\pi_k,c}$. Moreover, paths are ordered lexicographically, so that the left most maximal path is precisely $u_1$. An example of such a tree is shown in Figure~\ref{fig:f1}.

%\begin{figure}[h!]
%  \centering
%    \includegraphics[width=0.5\textwidth]{AD_and_POs_pics}
%  \caption{The tree corresponding to $\sG_{\sF_3}$. $\sF_3$ has domain $U=\{\, (0,0,0),\, (0,0,1),\, (1),\, (2,0)\,  \}$}
%  \label{fig:f1}
%\end{figure}

\section{Stuff}


Throughout the dissertation we work in the context of $\ad$ inside of $\LR$. In this section we give basic definitions and facts needed for what follows, then we define {\emph{general}} measures and state the theorem which gives the formula for ultrapowers.
Throughout this chapter \emph{c.u.b. set} means closed and unbounded subset of $\aleph_1$; \emph{measure} means a $\sigma$-complete ultrafilter; $\mu$ is the normal measure on $\aleph_1$; and $\mu^m$ is its $m$-fold product measure on $\aleph_1^m$.

\subsection{Technical Lemmas}
\subsection{Ultrapowers by Basic Measures}
\subsection{Ultrapowers by General Measures}


\section{Descriptions and Representation of Cardinals Below $\bd^1_5$}
\subsection{Descriptions and their interpretation}
\subsection{The Lowering Operator}
\subsection{Representation of Cardinals below $\bd^1_5$}



\newpage
